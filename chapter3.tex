
The concepts of bosons and fermions lie at the heart of
microscopic physics. They are described in terms of creation and
annihilation operators of the corresponding particle algebra: \bea
c_i c_j \mp c_j c_i & = & 0 \\
c_i c^*_j \mp c^*_j c_i & = & \delta_{ij} \eea where the upper
sign is for the boson algebra $BA(d)$ and the lower sign is for
the fermion algebra $FA(d)$.

It has been realized that quantum algebras play an important role
in the description of physical phenomena. Some classical physical
systems which are invariant under a classical Lie group, when
quantized, are invariant under a quantum group \cite{fadeev}. The
quantum groups thus considered turn out to be $q$-deformations of
the classical semisimple groups. On the other hand, inhomogeneous
quantum groups \cite{sww,pw} are perhaps more interesting since
classical inhomogeneous groups such as the Poincar\'e group are
more important in physics.

In this paper we will investigate an important class of
inhomogeneous quantum groups which are related to the boson
algebra $BA(d)$ and the fermion algebra $FA(d)$. Although $BA(d)$
and $FA(d)$ themselves are not quantum groups, by considering
quantum group versions of symmetry transformations acting on these
algebras, one can arrive at these inhomogeneous quantum groups.
Mathematically speaking we are thus interested in constructing
left modules of these algebras such that these modules have Hopf
algebra structure.

Traditionally the boson algebra has the symmetry group $ISp(2d,
\IR)$, the inhomogeneous symplectic group, which transforms
creation and annihilation operators as: \beq c_i \rightarrow
\alpha_{ij} c_j + \beta_{ij} c^*_j + \gamma_i \quad . \eeq In this
transformation $\alpha_{ij}, \beta_{ij}, \gamma_i$ are complex
numbers satisfying the constraints required by the group $ISp(2d,
\IR)$. One should note that this symmetry group is also the group
of linear canonical transformations of a classical dynamical
system. An important physical application of this transformation
is the Bogoliubov transformation which is crucial in the
explanation of many quantum mechanical effects such as the Unruh
Effect \cite{unruh}
 and Hawking Radiation \cite{hawking}.
In the case of the Hawking Radiation, the physical
reinterpretation of the transformed operators imply that the
future vacuum state is annihilated by the transformed annihilation
operator, which is related to the initial creation and
annihilation operators by a Bogoliubov transformation.

Similar to the boson algebra, the fermion algebra has the
classical symmetry group $O(2d)$ with the transformation law: \beq
c_i \rightarrow \alpha_{ij} c_j + \beta_{ij} c^*_j \quad . \eeq
however, unlike its bosonic counterpart this algebra is not
inhomogeneous. This fact is the primary motivation for the
generalization that we are going to offer. By relaxing the
conditions on the transformation coefficients such as
commutativity, one can come up with inhomogeneous invariance
(quantum)groups for fermions and for bosons alike. The explicit
$R$-matrices utilizing the quantum group properties of these
structures have already been presented \cite{agy, ab}. In this
paper, after a brief definition of these quantum groups \FIO, the
fermionic inhomogeneous orthogonal quantum group, and \BISp, the
bosonic inhomogeneous symplectic quantum group, in Section 1, we
will investigate their sub(quantum)groups and also study the
(quantum)groups obtained by their contractions. In the last
section, $FIO(2d + 1, \IR)$, the fermionic inhomogeneous quantum
orthogonal group in odd number of dimensions, will also be defined
and its properties examined.

\section{The Bosonic Inhomogeneous Symplectic Quantum Group \BISp}

\section{The Fermionic Inhomogenous Group \FIO}
A general transformation of a particle algebra can be described in
the following way: \beq \left(
\begin{array}{c}
c' \\
{c^{*}}' \\
1
\end{array}
\right) = \left(
\begin{array}{ccc}
\alpha & \beta & \gamma \\
\beta^* & \alpha^* & \gamma^* \\
0 & 0 & 1
\end{array}
\right) \dot{\otimes} \left(
\begin{array}{c}
c \\
c^* \\
1
\end{array}
\right) \eeq where $c, c^*, \gamma, \gamma^*$ are column matrices
and $\alpha, \beta, \alpha^*, \beta^*$ are $d\times d$ matrices.
Thus, in index notation the transformation is given by: \bea
c'_i &=& \alpha_{ij} \otimes c_j + \beta_{ij} \otimes c^*_j + \gamma_i \otimes 1 \quad , \\
{c^{*}}'_i &=& \alpha^*_{ij} \otimes c^*_j + \beta^*_{ij} \otimes
c_j + \gamma^*_i \otimes 1 \quad . \eea

Given this transformation, we look for an algebra $\mathcal{A}$
generated by these matrix elements such that the particle algebra
remains invariant. Thus, we first write the transformation matrix
in the above equation in the following way: \beq M = \left(
\begin{array}{cc|c}
\alpha & \beta & \gamma \\
\beta^* & \alpha^* & \gamma^* \\
\hline 0 & 0 & 1
\end{array}
\right)
 =
\left(
\begin{array}{c|c}
A & \Gamma \\
\hline 0 & 1
\end{array}
\right) \quad . \label{M-matrix} \eeq

We assume that $\alpha_{ij}, \beta_{ij}, \gamma_i$ belong to a
possibly noncommutative algebra on which a hermitian conjugation
denoted by $*$ is defined. We also assume that the matrix elements
of $A$ form a commutative subalgebra.

Applying this transformation and requiring that the
bosonic/fermionic particle algebra remains invariant after the
transformation, we arrive at the following relations that the
transformation parameters should obey: \bea
\gamma_i \gamma^*_j \mp \gamma^*_j \gamma_i &=& \delta_{ij} - \alpha_{ik}\alpha^*_{jk} \pm \beta_{ik} \beta^*_{jk} \label{rel1} \\
\gamma_i \gamma_j \mp \gamma_j \gamma_i &=& \pm \beta_{ik} \alpha_{jk} - \alpha_{ik} \beta_{jk} \label{rel2} \\
\alpha_{ij} \gamma_k \mp \gamma_k \alpha_{ij} & = & 0 \label{rel3} \\
\beta_{ij} \gamma_k \mp \gamma_k \beta_{ij} & = & 0 \label{rel4} \\
\alpha_{ij} \gamma^*_k \mp \gamma^*_k \alpha_{ij} & = & 0 \label{rel5} \\
\beta_{ij} \gamma^*_k \mp \gamma^*_k \beta_{ij} & = & 0
\label{rel6} \eea together with the $*$-conjugates of these
relations. In these relations the upper and lower signs are for
the transformation of bosons and fermions, respectively.
Furthermore according to our assumption above, the set
$\alpha_{ij}, \beta_{i,j}, \alpha^*_{ij}, \beta^*_{ij}$ forms a
commutative algebra.

The set of matrices $M$ obeying the above relations form the group
of inhomogeneous transformations of bosons and fermions. For
bosons, we name this group \BISp and for fermions \FIO. These
symmetry groups, however, are not classical groups but are in fact
quantum groups with a Hopf algebra structure. As shown in
\cite{ab}, this Hopf algebra has an explicit $R$-matrix
representation and the coproduct, counit and antipode are defined
as: \bea
\Delta(M) & = & M \dot{\otimes} M \label{coproduct} \\
\epsilon(M) & = & I \label{counit} \\
S(M) & = & M^{-1} \label{antipode} \quad . \eea In Equation
\ref{coproduct}, the symbol $\dot{\otimes}$ denotes the usual
matrix multiplication where when elements of the matrices are
multiplied, tensor multiplication is used.

The inverse of the matrix $M$ can be defined as: \beq M^{-1} =
\left(
\begin{array}{cc}
A^{-1} & -A^{-1} \Gamma \\
0 & 1
\end{array}
\right) \eeq where $A^{-1}$ is defined in the standard way since
matrix elements of $A$ were assumed to be commutative.

\section{Subgroups and Contractions}
After having shown that the inhomogeneous transformations form the
symmetry groups \FIO and \BISp and that they are quantum groups,
one important question is what sub(quantum)groups do these quantum
groups have. For example, we know that the group $ISp(2d, \IR)$ is
an important special subgroup of \BISp and other
sub(quantum)groups could turn out to have similarly important
physical applications. While searching for sub(quantum)groups, we
would also like to find new (quantum)groups allowed by suitable
contractions \cite{inonu} of these quantum groups as well.

The sub(quantum)groups of these algebras are obtained by imposing
additional relations on the matrix elements of $M$ which obey the
relations (\ref{rel1}) - (\ref{rel6}). The additional relations
that we will impose are:
\begin{enumerate}
\renewcommand{\labelenumi}{\bf(\alph{enumi})}
\item $\delta_{ij} - \alpha_{ik}\alpha^*_{jk} \pm \beta_{ik} \beta^*_{jk} = \pm \beta_{ik} \alpha_{jk} - \alpha_{ik} \beta_{jk} = 0$
\item $\gamma_i = 0$
\item $\beta_{ij} = 0$
\item $\alpha_{ij} = 0$
\end{enumerate}

We would like to study the implication of each relation one by one
in the bosonic and the fermionic case.

\subsection{Inhomogeneous Sub(super)groups}
The relation {\bf(a)}:
\[
\delta_{ij} - \alpha_{ik}\alpha^*_{jk} \pm \beta_{ik} \beta^*_{jk}
= \pm \beta_{ik} \alpha_{jk} - \alpha_{ik} \beta_{jk} = 0
\]
by virtue of (\ref{rel1}) and (\ref{rel2}) implies that $\gamma_i
\gamma^*_j \mp \gamma^*_j \gamma_i = 0$ and $\gamma_i \gamma_j \mp
\gamma_j \gamma_i = 0$, i.e. that the inhomogeneous transformation
parameters are commutative or anticommutative variables.

In the case of the fermionic particles, we end up with an
inhomogeneous orthogonal algebra where the inhomogeneous
parameters are grassmanian variables thus giving us the group
$GrIO(2d, \IR)$ as the resulting subgroup of \FIO. This group can
be considered to be an inhomogeneous supergroup.

More generally, in the fermionic case, $\alpha_{ij}$,
$\beta_{ij}$, $\alpha^*_{ij}$, $\beta^*_{ij}$ anticommute with
$\gamma_i$, $\gamma^*_i$ and the \FIO matrices $M$ are multiplied
with each other using the standard tensor product. It can
additionally be shown that $\alpha_{ij}$, $\beta_{ij}$,
$\alpha^*_{ij}$, $\beta^*_{ij}$ can be taken to commute with
$\gamma_i$, $\gamma^*_i$ provided that the matrices $M$ are
multiplied with a braided \cite{majid} tensor product, eg. \beq (A
\otimes B)(C \otimes D) = - AC \otimes BD \eeq whenever $B$ and
$C$ are both fermionic. This also corresponds to the usual
superalgebra approach.

For bosonic particles, applying the relation {\bf(a)} gives us the
classical symmetry group of the bosonic particle algebra $ISp(2d,
\IR)$ in which all the parameters of the inhomogeneous
transformation are commutative.

\subsection{Homogeneous Subgroups}
The relation {\bf(b)}:
\[
\gamma_i = 0
\]
practically gets rid of the inhomogeneous part of the
transformation and also implies the relation {\bf(a)} considered
in the previous subsection.

For the fermionic particles, this gives us the subgroup in the
previous subsection without the inhomogeneous part which is the
classical orthogonal group $O(2d, \IR)$ and similarly, for bosonic
particles, gives us the classical symplectic group $Sp(2d, \IR)$.

\subsection{Fermionic and Bosonic Inhomogeneous Unitary Quantum Groups}
The relation {\bf(c)}:
\[
\beta_{ij} = 0
\]
applied to the transformation gets rid of the off-diagonal members
of the homogeneous part of it and leaves us with the following
relation: \beq \gamma_i\gamma^*_j \mp \gamma^*_j\gamma_i =
\delta_{ij} - \alpha_{ik}\alpha^*_{jk} \eeq For the homogeneous
part of the transformation, this equation implies $ \delta_{ij} =
\alpha_{ik}\alpha^*_{jk} $ which tells us that the submatrices
$\alpha$ and $\alpha^*$ in equation (\ref{M-matrix}) are both
members of $U(d)$. The subgroup we have arrived at thus is an
inhomogeneous quantum group whose homogeneous part is $U(d)$. For
fermions we will name this group $FIU(d)$, the fermionic
inhomogeneous quantum group, and for bosons we will similarly name
it $BIU(d)$, the bosonic inhomogeneous quantum group.

\subsection{Fermion and Boson Algebra}
The relation {\bf(d)}:
\[
\alpha_{ij} = 0
\]
applied alone onto the transformation gets rid of the diagonal
members of the homogeneous part and prevents such transformations
from forming a (quantum)group since the homogeneous parts of these
set of transformations can never include the identity
transformation.

However, if this relation is applied together with the previous
one, relation {\bf(c)}, the resulting relation gets rid of the
whole homogeneous part of the transformation leaving only the
inhomogeneous part and gives us a single relation: \beq
\gamma_i\gamma^*_j \mp \gamma^*_j\gamma_i = \delta_{ij} \eeq which
gives us back the fermion algebra, $FA(d)$, for the fermionic case
and the boson algebra, $BA(d)$, for the bosonic case.

We should note, however, that after this condition is applied, the
resulting set of matrices $M$, which form $FA(d)$ or $BA(d)$, are
no longer quantum or classical groups since the antipode defined
in equation (\ref{antipode}) no longer exits. For this reason,
these algebras can be considered to be a boundary for the
subgroups of their corresponding quantum groups.

\subsection{Sub(quantum)group Diagram}
As a result of the above discussion, we get the sub(quantum)group
diagram:
\[
\begin{CD}
{FIO(2d, \IR)} @>{\bf (a)}>> {GrIO(2d, \IR)} @>{\bf (b)}>> {O(2d, \IR)} \\
@V{\bf (c)}VV @V{\bf (c)}VV @V{\bf (c)}VV \\
{FIU(d)}  @>{\bf (a)}>> {GrIU(d)} @>{\bf (b)}>> {U(d)} \\
@V{\bf (d)}VV \\
{FA(d)}
\end{CD}
\]
for the fermionic case and the diagram:
\[
\begin{CD}
{BISp(2d, \IR)} @>{\bf (a)}>> {ISp(2d, \IR)} @>{\bf (b)}>> {Sp(2d, \IR)} \\
@V{\bf (c)}VV @V{\bf (c)}VV @V{\bf (c)}VV \\
{BIU(d)} @>{\bf (a)}>> {IU(d)} @>{\bf (b)}>> {U(d)} \\
@V{\bf (d)}VV \\
{BA(d)}
\end{CD}
\]
for the bosonic case.

\subsection{Contractions of \FIO and \BISp}
In order to explore the new (quantum)groups that will come about
as the result of a contraction, we replace $\gamma_i$ by
$\gamma_i/\hbar$ so that we may consider the case $\hbar
\rightarrow 0$. After this replacement, the equations (\ref{rel1})
and (\ref{rel2}) become: \bea
\gamma_i \gamma^*_j \mp \gamma^*_j \gamma_i &=& \hbar(\delta_{ij} - \alpha_{ik}\alpha^*_{jk} \pm \beta_{ik} \beta^*_{jk}) \\
\gamma_i \gamma_j \mp \gamma_j \gamma_i &=& \hbar(\pm \beta_{ik}
\alpha_{jk} - \alpha_{ik} \beta_{jk}) \eea When we consider the
case $\hbar \rightarrow 0$, we get the relations: \bea
\gamma_i \gamma^*_j \mp \gamma^*_j \gamma_i &=& 0 \\
\gamma_i \gamma_j \mp \gamma_j \gamma_i &=& 0 \eea which imply
that the inhomogeneous part of the transformation form grassmanian
variables for the fermionic case and ordinary complex numbers for
the bosonic case. What makes this case different from the previous
case of sub(super)groups is that the homogeneous part of this
transformation forms a matrix $A$ with non-zero determinant. We
can transform such a matrix $A$ with a similarity transformation
given by the unitary matrix: \beq U = \frac{1}{\sqrt{2}} \left(
\begin{array}{cc}
1 & 1 \\
i & -i
\end{array}
\right) \eeq to put it in a real form. The transformation gives:
\bea
A' & = & U A U^\dagger \\
& = & \frac12 \left(
\begin{array}{cc}
1 & 1 \\
i & -i
\end{array}
\right) \left(
\begin{array}{cc}
\alpha & \beta \\
\beta^* & \alpha^*
\end{array}
\right) \left(
\begin{array}{cc}
1 & -i \\
1 & i
\end{array}
\right) \\
& = & \left(
\begin{array}{cc}
Re(\alpha) + Re(\beta) & Im(\alpha) - Im(\beta) \\
- Im(\alpha) - Im(\beta) & Re(\alpha) - Re(\beta)
\end{array}
\right) \eea which is a real matrix that is a member of the
general linear group $GL(2d, \IR)$. Thus for the fermionic case we
have the group $GrIGL(2d, \IR)$, the grassmanian inhomogeneous
general linear group, and for the bosonic case we have $IGL(2d,
\IR)$, the inhomogeneous general linear group.

If we consider the contraction of the subgroups as well then we
should examine the $\hbar \rightarrow 0$ limit after the relations
{\bf(c)} and {\bf(d)} are applied.

After we apply relation {\bf(c)}, we get the subgroups, $FIU(d)$
and $BIU(d)$ as discussed previously. After, the contraction,
again the inhomogeneous part of these groups become grassmanian
variables and complex numbers for the fermionic and bosonic cases
respectively. However, if we apply the previous similarity
transformation on the homogeneous part, we get: \bea
A' & = & U A U^\dagger \\
& = & \frac12 \left(
\begin{array}{cc}
1 & 1 \\
i & -i
\end{array}
\right) \left(
\begin{array}{cc}
\alpha & 0 \\
0 & \alpha^*
\end{array}
\right) \left(
\begin{array}{cc}
1 & -i \\
1 & i
\end{array}
\right) \\
& = & \left(
\begin{array}{cc}
Re(\alpha) & Im(\alpha) \\
- Im(\alpha) & Re(\alpha)
\end{array}
\right) \\
& = & Re(\alpha) \I1
 +
Im(\alpha) \II \eea where $\I1$ stands for the identity matrix and
$\II$ stands for the matrix the square of which is minus the
identity matrix. This way we can see that the matrix $A'$ is a
actually member of $GL(d, \IC)$. This gives us $GrIGL(d, \IC)$ as
the group for the contraction in the fermionic case and $IGL(d,
\IC)$ for the bosonic case.

We have previously shown that we get the fermionic and bosonic
algebras after applying both of the relations {\bf(c)} and
{\bf(d)}. We have also discussed that in this case only the
inhomogeneous part of the transformation survives and after
applying the contraction the inhomogeneous part of the
transformation turns into grassmanian or complex variables. Thus
in this case, the contraction of $FA(d)$ gives us $Gr(d, \IC)$ and
the contraction of $BA(d)$ gives us $\IC^d$.


As a summary, for the contraction considered combined with the
remaining subgroup relations we get the tables:
\[
\begin{CD}
{FIO(2d, \IR)} @>{\bf (e)}>> {GrIGL(2d, \IR)} \\
@V{\bf (c)}VV @V{\bf (c)}VV \\
{FIU(d)} @>{\bf (e)}>> {GrIGL(d, \IC)} \\
@V{\bf (d)}VV @V{\bf (d)}VV \\
{FA(d)} @>{\bf (e)}>> {Gr(d, \IC)}
\end{CD}
\]
for the fermionic case and:
\[
\begin{CD}
{BISp(2d, \IR)} @>{\bf (e)}>> {IGL(2d, \IR)} \\
@V{\bf (c)}VV @V{\bf (c)}VV \\
{BIU(d)} @>{\bf (e)}>> {IGL(d, \IC)} \\
@V{\bf (d)}VV @V{\bf (d)}VV \\
{BA(d)} @>{\bf (e)}>> {\IC^d}
\end{CD}
\]
for the bosonic case.

\section{The Fermionic Inhomogeneous
Orthogonal Quantum Group of Odd Dimension}

When we consider a unitary transformation of the \FIO matrix as:
\[
M \rightarrow UMU^{-1}
\]
using the unitary matrix: \beq U = \left(
\begin{tabular}{cc|c}
$\frac{1}{\sqrt{2}}$ & $\frac{1}{\sqrt{2}}$ & $0$ \\
$\frac{i}{\sqrt{2}}$ & $\frac{-i}{\sqrt{2}}$ & $0$ \\
\hline $0$ & $0$ & $1$
\end{tabular}
\right) \eeq we can see that we can put $M$ in the real form: \beq
\left(
\begin{tabular}{cc|c}
$Re(\alpha + \beta)$ & $Im(\alpha - \beta)$ & $\sqrt{2}Re(\gamma)$ \\
$Im(-\alpha - \beta)$ & $Re(\alpha - \beta)$ & $\sqrt{2}Im(\gamma)$ \\
\hline $0$ & $0$ & $1$
\end{tabular}
\right) = \left(
\begin{tabular}{c|c}
$A$ & $\Gamma$ \\
\hline $0$ & $1$
\end{tabular}
\right) \eeq where $A$ and $\Gamma$ matrices are defined as in
(\ref{M-matrix}), and $Re$ and $Im$ denote the hermitian and
anti-hermitian parts.

Using this form it is found that for \FIO, the transformation
relations (\ref{rel1}) and (\ref{rel2}) together become:

\beq [\Gamma_i, \Gamma_j]_+ = \delta_{ij} - A_{ik} A_{jk}
\quad\quad, i, j = 1, 2, \ldots , 2d . \eeq By extending the range
of the indices in this relation to odd-dimensions it is possible
to define $FIO(2d + 1, \IR)$, the fermionic inhomogeneous
orthogonal algebra of odd dimension.

Similar to the analysis that went into finding the
sub(quantum)groups of \FIO, we can also investigate the
sub(quantum)groups of $FIO(2d + 1, \IR)$. The sub(quantum)group
relations in this case, however, are more restricted owing to the
fact that the algebra is not described anymore by submatrices of
the $A$ matrix but is rather described by the whole matrix itself.
Thus, we cannot set $\alpha$ or $\beta$ to zero on their own, we
can only restrict the algebra by setting the whole of $A$ to zero.
Thus the resulting sub(quantum)algebra relations are:

\begin{enumerate}
\renewcommand{\labelenumi}{\bf(\alph{enumi})}
\item $\delta_{ij} - A_{ik}A_{jk} = 0$
\item $\Gamma_i = 0$
\item $A_{ij} = 0$
\end{enumerate}
which, through a similar analysis to the even dimensional case,
gives us the following sub(quantum)group diagram:

\[
\begin{CD}
{FIO(2d + 1, \IR)} @>{\bf (a)}>> {GrIO(2d + 1, \IR)} @>{\bf (b)}>> {O(2d + 1, \IR)} \\
@V{\bf (c)}VV\\
{Clif(2d + 1)}
\end{CD}
\]

