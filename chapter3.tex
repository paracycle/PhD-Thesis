
The concepts of bosons and fermions lie at the heart of
microscopic physics. They are described in terms of creation and
annihilation operators of the corresponding particle algebra: \bea
c_i c_j \mp c_j c_i & = & 0 \\
c_i c^*_j \mp c^*_j c_i & = & \delta_{ij} \eea where the upper
sign is for the boson algebra $BA(d)$ and the lower sign is for
the fermion algebra $FA(d)$.

It has been realized that quantum algebras play an important role
in the description of physical phenomena. Some classical physical
systems which are invariant under a classical Lie group, when
quantized, are invariant under a quantum group \cite{fadeev}. The
quantum groups thus considered turn out to be $q$-deformations of
the classical semisimple groups. On the other hand, inhomogeneous
quantum groups \cite{sww,pw} are perhaps more interesting since
classical inhomogeneous groups such as the Poincar\'e group are
more important in physics.

In this paper we will investigate an important class of
inhomogeneous quantum groups which are related to the boson
algebra $BA(d)$ and the fermion algebra $FA(d)$. Although $BA(d)$
and $FA(d)$ themselves are not quantum groups, by considering
quantum group versions of symmetry transformations acting on these
algebras, one can arrive at these inhomogeneous quantum groups.
Mathematically speaking we are thus interested in constructing
left modules of these algebras such that these modules have Hopf
algebra structure.

Traditionally the boson algebra has the symmetry group $ISp(2d,
\IR)$, the inhomogeneous symplectic group, which transforms
creation and annihilation operators as: \beq c_i \rightarrow
\alpha_{ij} c_j + \beta_{ij} c^*_j + \gamma_i \quad . \eeq In this
transformation $\alpha_{ij}, \beta_{ij}, \gamma_i$ are complex
numbers satisfying the constraints required by the group $ISp(2d,
\IR)$. One should note that this symmetry group is also the group
of linear canonical transformations of a classical dynamical
system. An important physical application of this transformation
is the Bogoliubov transformation which is crucial in the
explanation of many quantum mechanical effects such as the Unruh
Effect \cite{unruh}
 and Hawking Radiation \cite{hawking}.
In the case of the Hawking Radiation, the physical
reinterpretation of the transformed operators imply that the
future vacuum state is annihilated by the transformed annihilation
operator, which is related to the initial creation and
annihilation operators by a Bogoliubov transformation.

Similar to the boson algebra, the fermion algebra has the
classical symmetry group $O(2d)$ with the transformation law: \beq
c_i \rightarrow \alpha_{ij} c_j + \beta_{ij} c^*_j \quad . \eeq
however, unlike its bosonic counterpart this algebra is not
inhomogeneous. This fact is the primary motivation for the
generalization that we are going to offer. By relaxing the
conditions on the transformation coefficients such as
commutativity, one can come up with inhomogeneous invariance
(quantum)groups for fermions and for bosons alike. The explicit
$R$-matrices utilizing the quantum group properties of these
structures have already been presented \cite{agy, ab}. In this
paper, after a brief definition of these quantum groups \FIO, the
fermionic inhomogeneous orthogonal quantum group, and \BISp, the
bosonic inhomogeneous symplectic quantum group, in Section 1, we
will investigate their sub(quantum)groups and also study the
(quantum)groups obtained by their contractions. In the last
section, $FIO(2d + 1, \IR)$, the fermionic inhomogeneous quantum
orthogonal group in odd number of dimensions, will also be defined
and its properties examined.

A general transformation of a particle algebra can be described in
the following way: \beq \left(
\begin{array}{c}
c' \\
{c^{*}}' \\
1
\end{array}
\right) = \left(
\begin{array}{ccc}
\alpha & \beta & \gamma \\
\beta^* & \alpha^* & \gamma^* \\
0 & 0 & 1
\end{array}
\right) \dot{\otimes} \left(
\begin{array}{c}
c \\
c^* \\
1
\end{array}
\right) \eeq where $c, c^*, \gamma, \gamma^*$ are column matrices
and $\alpha, \beta, \alpha^*, \beta^*$ are $d\times d$ matrices.
Thus, in index notation the transformation is given by: \bea
c'_i &=& \alpha_{ij} \otimes c_j + \beta_{ij} \otimes c^*_j + \gamma_i \otimes 1 \quad , \\
{c^{*}}'_i &=& \alpha^*_{ij} \otimes c^*_j + \beta^*_{ij} \otimes
c_j + \gamma^*_i \otimes 1 \quad . \eea

Given this transformation, we look for an algebra $\mathcal{A}$
generated by these matrix elements such that the particle algebra
remains invariant. Thus, we first write the transformation matrix
in the above equation in the following way: \beq M = \left(
\begin{array}{cc|c}
\alpha & \beta & \gamma \\
\beta^* & \alpha^* & \gamma^* \\
\hline 0 & 0 & 1
\end{array}
\right)
 =
\left(
\begin{array}{c|c}
A & \Gamma \\
\hline 0 & 1
\end{array}
\right) \quad . \label{M-matrix} \eeq

We assume that $\alpha_{ij}, \beta_{ij}, \gamma_i$ belong to a
possibly noncommutative algebra on which a hermitian conjugation
denoted by $*$ is defined.

\section{The Bosonic Inhomogeneous Symplectic Quantum Group \BISp}

If we consider the transformation matrix \eqref{M-matrix} being applied to the boson algebra given by:
\bea
c_i c_j - c_j c_i & = & 0 \\
c_i c^*_j - c^*_j c_i & = & \delta_{ij}
\eea
then we require that the transformed operators $c'_i$ and ${c^*}'_i$ are required to satisfy the same algebra in order for the transformation to be an algebra invariance. Thus we require that:
\bea
c'_i c'_j - c'_j c'_i & = & 0 \\
c'_i {c^*}'_j - {c^*}'_j c'_i & = & \delta_{ij}
\eea
Explicitly writing out the transformed operators, these relations become:
\begin{align}
\begin{split}
(\alpha_{ik} \otimes c_k + \beta_{ik} \otimes c^*_k + \gamma_i \otimes 1)
(\alpha_{jl} \otimes c_l + \beta_{jl} \otimes c^*_l + \gamma_j \otimes 1) \\
- (\alpha_{jl} \otimes c_l + \beta_{jl} \otimes c^*_l + \gamma_j \otimes 1)
(\alpha_{ik} \otimes c_k + \beta_{ik} \otimes c^*_k + \gamma_i \otimes 1)
&=  0
\end{split}
\\
\begin{split}(\alpha_{ik} \otimes c_k + \beta_{ik} \otimes c^*_k + \gamma_i \otimes 1)
(\alpha^*_{jl} \otimes c^*_l + \beta^*_{jl} \otimes c_l + \gamma^*_j \otimes 1) \\
- (\alpha^*_{jl} \otimes c^*_l + \beta^*_{jl} \otimes c_l + \gamma^*_j \otimes 1)
(\alpha_{ik} \otimes c_k + \beta_{ik} \otimes c^*_k + \gamma_i \otimes 1)
& =  \delta_{ij}
\end{split}
\end{align}
which gives us:
\beq
\begin{split}
& [\alpha_{ik}, \alpha_{jl}]c_l c_k + [\beta_{ik}, \beta_{jl}]c^*_l c^*_k \\
& + [\alpha_{ik}, \gamma_j] c_k + [\beta_{ik}, \gamma_j] c^*_k \\
& + [\gamma_i, \alpha_{jl}] c_l + [\gamma_i, \beta_{jl}] c^*_l \\
& + [\alpha_{ik}, \beta_{jl}] c_k c^*_l + [\beta_{ik}, \alpha_{jl}] c^*_k c_l \\
& + (\alpha_{jk} \beta_{ik} - \beta_{jk} \alpha_{ik} + [\gamma_i, \gamma_j]) =  0 \quad ,
\end{split}
\eeq
and
\beq
\begin{split}
& [\alpha_{ik}, \beta^*_{jl}]c_l c_k + [\beta_{ik}, \alpha^*_{jl}] c^*_l c^*_k \\
& + [\alpha_{ik}, \gamma^*_j] c_k + [\beta_{ik}, \gamma^*_j] c^*_k \\
& + [\gamma_i, \beta^*_{jl}] c_l + [\gamma_i, \alpha_{jl}] c^*_l \\
& + [\alpha_{ik}, \alpha^*_{jl}] c_k c^*_l + [\beta_{ik}, \beta^*_{jl}] c^*_k c_l \\
& + (\alpha^*_{jk} \alpha_{ik} - \beta^*_{jk} \beta_{ik} + [\gamma_i, \gamma^*_j]) = \delta_{ij} \quad .
\end{split}
\eeq

In the first of these relations, for the equality to be satisfied, it is sufficient for
the coefficients of all the terms on the left hand side to be equal to zero.
In the second one, however, we only have a term that is a multiple of the unit
element of the boson algebra on the right hand side, thus the coefficient of that term
should be equal on both sides and it is sufficient for the coefficients of the
other terms on the left hand side to be separately equal to zero.

Thus we have the following relations between the transformation elements:
\bea
\gamma_i \gamma^*_j - \gamma^*_j \gamma_i &=& \delta_{ij} - \alpha_{ik}\alpha^*_{jk} + \beta_{ik} \beta^*_{jk} \label{rel1bos} \\
\gamma_i \gamma_j - \gamma_j \gamma_i &=& \beta_{ik} \alpha_{jk} - \alpha_{ik} \beta_{jk} \label{rel2bos} \\
\alpha_{ij} \gamma_k - \gamma_k \alpha_{ij} & = & 0 \label{rel3bos} \\
\beta_{ij} \gamma_k - \gamma_k \beta_{ij} & = & 0 \label{rel4bos} \\
\alpha_{ij} \gamma^*_k - \gamma^*_k \alpha_{ij} & = & 0 \label{rel5bos} \\
\beta_{ij} \gamma^*_k - \gamma^*_k \beta_{ij} & = & 0 \label{rel6bos}
\eea
and any two elements from the set
$\alpha_{ij}, \beta_{ij}, \alpha^*_{ij}, \beta^*_{ij}$ commute.

The set of matrices $M$ obeying the above relations form the group
of inhomogeneous transformations of bosons. We name this quantum group
as the bosonic inhomogeneous symplectic quantum group \BISp
since it is an inhomogeneous extension of the symplectic group where
the inhomogeneous part exhibits bosonic behavior. This
symmetry group, however, is not a classical group like the symplectic group
but is in fact a quantum group with a Hopf algebra structure. As shown in
\cite{ab}, this Hopf algebra has an explicit $R$-matrix
representation and the coproduct, counit and coinverse are defined
as:
\bea
\Delta(M) & = & M \dot{\otimes} M \label{coproductbos} \\
\epsilon(M) & = & I \label{counitbos} \\
S(M) & = & M^{-1} \label{antipodebos} \quad . \eea In Equation
\eqref{coproductbos}, the symbol $\dot{\otimes}$ denotes the usual
matrix multiplication where when elements of the matrices are
multiplied, tensor multiplication is used.

The inverse of the matrix $M$ can be defined as: \beq M^{-1} =
\left(
\begin{array}{cc}
A^{-1} & -A^{-1} \Gamma \\
0 & 1
\end{array}
\right) \eeq where $A^{-1}$ is defined in the standard way since
matrix elements of $A$ are shown to be commutative.

\subsection{Subgroups}
After having shown that the inhomogeneous transformations of the boson
algebra forms the symmetry quantum group \BISp,
one important question is what sub(quantum)groups does this quantum
group have. For example, we know that the group $ISp(2d, \IR)$ is
an important special subgroup of \BISp and other
sub(quantum)groups could turn out to have similarly important
physical applications. While searching for sub(quantum)groups, we
would also like to find new (quantum)groups allowed by suitable
contractions \cite{inonu} of these quantum groups as well.

The sub(quantum)groups are obtained by imposing
additional relations on the matrix elements of $M$ which obey the
relations (\ref{rel1bos}) - (\ref{rel6bos}). The additional relations
that we will impose are:
\begin{subequations}
\bea
\delta_{ij} - \alpha_{ik}\alpha^*_{jk} + \beta_{ik} \beta^*_{jk} = \beta_{ik} \alpha_{jk} - \alpha_{ik} \beta_{jk} = 0 \label{suba} \\
\gamma_i = 0  \label{subb} \\
\beta_{ij} = 0  \label{subc} \\
\alpha_{ij} = 0  \label{subd}
\eea
\end{subequations}

We would like to study the implication of each relation one by one
in the following subsections.

\subsubsection{Inhomogeneous Subgroup}

The relation \eqref{suba}:
\[
\delta_{ij} - \alpha_{ik}\alpha^*_{jk} + \beta_{ik} \beta^*_{jk}
= \beta_{ik} \alpha_{jk} - \alpha_{ik} \beta_{jk} = 0
\]
by virtue of \eqref{rel1bos} and \eqref{rel2bos} implies that $\gamma_i
\gamma^*_j - \gamma^*_j \gamma_i = 0$ and $\gamma_i \gamma_j -
\gamma_j \gamma_i = 0$, i.e. that the inhomogeneous transformation
parameters are commutative variables.

For bosonic particles, the fact that the inhomogeneous elements of the
quantum group are commutative elements coupled with the fact that the
remaining relations between the transformation elements are already
commutative gives us a symmetry transformation of the boson algebra
where all the elements commute. However, we know that such a transformation
is nothing but the classical symmetry group of the bosonic particle algebra $ISp(2d,
\IR)$ in which all the parameters, including the inhomogeneous
elements, are commutative.

\subsubsection{Homogeneous Subgroup}

The relation \eqref{subb}:
\[
\gamma_i = 0
\]
practically gets rid of the inhomogeneous part of the
transformation and also implies the relation \eqref{suba} considered
in the previous subsection. Since the previous relation is implied
the resulting group will be a subgroup of $ISp(2d,
\IR)$ and since the group is not inhomogeneous anymore the resulting
subgroup is the classical symplectic group $Sp(2d, \IR)$.

\subsubsection{Bosonic Inhomogeneous Unitary Quantum Group}
The relation \eqref{subc}:
\[
\beta_{ij} = 0
\]
applied to the transformation gets rid of the off-diagonal members
of the homogeneous part of it and leaves us with the following
relation:
\bea
\gamma_i\gamma^*_j - \gamma^*_j\gamma_i &=& \delta_{ij} - \alpha_{ik}\alpha^*_{jk} \\
\gamma_i\gamma_j - \gamma_j\gamma_i &=& 0
\eea

This equation implies for the homogeneous
part of the transformation the relation:
\beq
\delta_{ij} = \alpha_{ik}\alpha^*_{jk}
\eeq
which tells us that the submatrices
$\alpha$ and $\alpha^*$ in equation \eqref{M-matrix} are
members of $U(d)$. The subgroup we have arrived at thus is an
inhomogeneous quantum group extension to the classical homogeneous
group $U(d)$. Since the inhomogeneous elements of the resulting group
obeys the same relations as \BISp, we will name this quantum group
 $BIU(d)$, the bosonic inhomogeneous quantum group.

\subsubsection{Boson Algebra}

The relation \eqref{subd}:
\[
\alpha_{ij} = 0
\]
applied alone onto the transformation gets rid of the diagonal
members of the homogeneous part and prevents such transformations
from forming a (quantum)group since the homogeneous parts of these
set of transformations can never include the identity
transformation.

However, if this relation is applied together with the previous
one, relation \eqref{subc}, the resulting relation gets rid of the
whole homogeneous part of the transformation leaving only the
inhomogeneous part and leaves us with two relations:
\bea
\gamma_i\gamma^*_j - \gamma^*_j\gamma_i &=& \delta_{ij} \\
\gamma_i\gamma_j - \gamma_j\gamma_i &=& 0
\eea
which
gives us back the boson algebra, $BA(d)$.

We should note, however, that after this condition is applied, the
resulting set of matrices $M$, which now form $BA(d)$, is
no longer a quantum or classical group since the antipode defined
in equation \eqref{antipodebos} no longer exits. For this reason,
the boson algebra can be considered to be a boundary for the
sub(quantum)groups of \BISp.

\subsubsection{Sub(quantum)group Diagram}
As a result of the above discussion, we get the sub(quantum)group
diagram:
\[
\begin{CD}
{BISp(2d, \IR)} @>\eqref{suba}>> {ISp(2d, \IR)} @>\eqref{subb}>> {Sp(2d, \IR)} \\
@V\eqref{subc}VV @V\eqref{subc}VV @V\eqref{subc}VV \\
{BIU(d)} @>\eqref{suba}>> {IU(d)} @>\eqref{subb}>> {U(d)} \\
@V\eqref{subd})VV \\
{BA(d)}
\end{CD}
\]
for the sub(quantum)groups of the \BISp we have introduced in this
section.

\subsection{Contractions}
In order to explore the new (quantum)groups that will come about
as the result of a contraction, we replace $\gamma_i$ by
$\gamma_i/\sqrt{\hbar}\;$ so that we may consider the case $\hbar
\rightarrow 0$. After this replacement, the equations \eqref{rel1bos}
and \eqref{rel2bos} become:
\bea
\gamma_i \gamma^*_j - \gamma^*_j \gamma_i &=& \hbar(\delta_{ij} - \alpha_{ik}\alpha^*_{jk} + \beta_{ik} \beta^*_{jk}) \\
\gamma_i \gamma_j - \gamma_j \gamma_i &=& \hbar(\beta_{ik} \alpha_{jk} - \alpha_{ik} \beta_{jk})
\eea
When we consider the
case $\hbar \rightarrow 0$, we get the relations:
\bea
\gamma_i \gamma^*_j - \gamma^*_j \gamma_i &=& 0 \\
\gamma_i \gamma_j - \gamma_j \gamma_i &=& 0
\eea
which imply
that the inhomogeneous part of the transformation form
ordinary complex numbers. What makes this case different
from the previous
case of subgroups is that the homogeneous part of this
transformation forms a matrix $A$ with non-zero determinant. We
can transform such a matrix $A$ with a similarity transformation
given by the unitary matrix:
\beq
U = \frac{1}{\sqrt{2}}
\left(
\begin{array}{cc}
1 & 1 \\
i & -i
\end{array}
\right)
\eeq
to put it in a real form. The transformation gives:
\beq
\begin{split}
A'
&= U A U^\dagger \\
&= \frac12
    \begin{pmatrix}
      1 & 1 \\
      i & -i
    \end{pmatrix}
    \begin{pmatrix}
      \alpha & \beta \\
      \beta^* & \alpha^*
    \end{pmatrix}
    \begin{pmatrix}
      1 & -i \\
      1 & i
    \end{pmatrix}
    \\
&=  \begin{pmatrix}
      Re(\alpha) + Re(\beta) & Im(\alpha) - Im(\beta) \\
      - Im(\alpha) - Im(\beta) & Re(\alpha) - Re(\beta)
    \end{pmatrix}
\end{split}
\eeq
which is a real matrix that is a member of the
general linear group $GL(2d, \IR)$. Thus we
have the group $IGL(2d, \IR)$, the inhomogeneous general linear group.

If we consider the contraction of the subgroups as well then we
should examine the $\hbar \rightarrow 0$ limit after the relations
\eqref{subc} and \eqref{subd} are applied.

After we apply relation \eqref{subc}, we get the subgroup $BIU(d)$
as discussed previously. After the contraction,
again, the inhomogeneous part of this group become
complex numbers. However, if we apply the previous similarity
transformation on the homogeneous part, we get:
\beq
\begin{split}
A'
&=  U A U^\dagger \\
&=  \frac12
      \begin{pmatrix}
        1 & 1 \\
        i & -i
      \end{pmatrix}
      \begin{pmatrix}
        \alpha & 0 \\
        0 & \alpha^*
      \end{pmatrix}
      \begin{pmatrix}
        1 & -i \\
        1 & i
      \end{pmatrix}
    \\
&=
    \begin{pmatrix}
      Re(\alpha) & Im(\alpha) \\
      - Im(\alpha) & Re(\alpha)
    \end{pmatrix}
    \\
&= Re(\alpha) \I1 + Im(\alpha) \II
\end{split}
\eeq
where $\I1$ stands for the identity matrix and
$\II$ stands for the matrix the square of which is minus the
identity matrix. This way we can see that the matrix $A'$ is a
actually member of $GL(d, \IC)$. This gives us $IGL(d, \IC)$
as the group we arrive at as the contraction of $BIU(d)$.

We have previously shown that we get the boson
algebra after applying both of the relations \eqref{subc} and
\eqref{subd}. We have also discussed that in this case only the
inhomogeneous part of the transformation survives. After
applying the contraction, the surviving inhomogeneous part of the
transformation turns into complex variables. Thus
in this case, the contraction of $BA(d)$ gives us $\IC^d$.

As a summary, for the contraction considered in this section
applied onto the subgroups obtained in the previous subsection
we get the following group diagram:
\[
\begin{CD}
{BISp(2d, \IR)} @>\hbar \rightarrow 0>> {IGL(2d, \IR)} \\
@V\eqref{subc}VV @V\eqref{subc}VV \\
{BIU(d)} @>\hbar \rightarrow 0>> {IGL(d, \IC)} \\
@V\eqref{subd}VV @V\eqref{subd}VV \\
{BA(d)} @>\hbar \rightarrow 0>> {\IC^d}
\end{CD}
\]

\section{The Fermionic Inhomogeneous Group \FIO}

Similarly to how it was done in the bosonic case one can also
consider the transformation matrix
\eqref{M-matrix} being applied to the fermion algebra given by:
\bea
c_i c_j + c_j c_i & = & 0 \\
c_i c^*_j + c^*_j c_i & = & \delta_{ij}
\eea
and then require that the transformed operators $c'_i$ and ${c^*}'_i$ satisfy
the same algebra in order for the transformation to be an algebra invariance.
Thus the requirement is that:
\bea
c'_i c'_j + c'_j c'_i & = & 0 \\
c'_i {c^*}'_j + {c^*}'_j c'_i & = & \delta_{ij}
\eea
Explicitly writing out the transformed operators, these relations become:
\begin{align}
\begin{split}
(\alpha_{ik} \otimes c_k + \beta_{ik} \otimes c^*_k + \gamma_i \otimes 1)
(\alpha_{jl} \otimes c_l + \beta_{jl} \otimes c^*_l + \gamma_j \otimes 1) \\
+(\alpha_{jl} \otimes c_l + \beta_{jl} \otimes c^*_l + \gamma_j \otimes 1)
(\alpha_{ik} \otimes c_k + \beta_{ik} \otimes c^*_k + \gamma_i \otimes 1)
& = 0
\end{split} \\
\begin{split}
(\alpha_{ik} \otimes c_k + \beta_{ik} \otimes c^*_k + \gamma_i \otimes 1)
(\alpha^*_{jl} \otimes c^*_l + \beta^*_{jl} \otimes c_l + \gamma^*_j \otimes 1) \\
+(\alpha^*_{jl} \otimes c^*_l + \beta^*_{jl} \otimes c_l + \gamma^*_j \otimes 1)
(\alpha_{ik} \otimes c_k + \beta_{ik} \otimes c^*_k + \gamma_i \otimes 1)
& = \delta_{ij}
\end{split}
\end{align}
which gives us:
\beq
\begin{split}
&[\alpha_{jl}, \alpha_{ik}]c_l c_k + [\beta_{jl}, \beta_{ik}]c^*_l c^*_k  \\
&+ \{\alpha_{ik}, \gamma_j\} c_k + \{\beta_{ik}, \gamma_j\} c^*_k  \\
&+ \{\gamma_i, \alpha_{jl}\} c_l + \{\gamma_i, \beta_{jl}\} c^*_l  \\
&+ [\alpha_{ik}, \beta_{jl}] c_k c^*_l + [\beta_{ik}, \alpha_{jl}] c^*_k c_l  \\
&+ (\alpha_{jk} \beta_{ik} + \beta_{jk} \alpha_{ik} + \{\gamma_i, \gamma_j\}) = 0 \quad ,
\end{split}
\eeq
and
\beq
\begin{split}
&[\beta^*_{jl}, \alpha_{ik}]c_l c_k + [\alpha^*_{jl}, \beta_{ik}] c^*_l c^*_k \\
&+ \{\alpha_{ik}, \gamma^*_j\} c_k + \{\beta_{ik}, \gamma^*_j\} c^*_k \\
&+ \{\gamma_i, \beta^*_{jl}\} c_l + \{\gamma_i, \alpha_{jl}\} c^*_l \\
&+ [\alpha_{ik}, \alpha^*_{jl}] c_k c^*_l + [\beta_{ik}, \beta^*_{jl}] c^*_k c_l \\
&+ (\alpha^*_{jk} \alpha_{ik} + \beta^*_{jk} \beta_{ik} + \{\gamma_i, \gamma^*_j\}) = \delta_{ij} \quad .
\end{split}
\eeq

In the first of these relations, for the equality to be satisfied, it is sufficient for
the coefficients of all the terms on the left hand side to be equal to zero.
In the second one, however, we only have a term that is a multiple of the unit
element of the boson algebra on the right hand side, thus the coefficient of that term
should be equal on both sides and it is sufficient for the coefficients of the
other terms on the left hand side to be separately equal to zero.

Thus we have the following relations between the transformation elements:
\bea
\gamma_i \gamma^*_j + \gamma^*_j \gamma_i &=& \delta_{ij} - \alpha_{ik}\alpha^*_{jk} - \beta_{ik} \beta^*_{jk} \label{rel1fer} \\
\gamma_i \gamma_j + \gamma_j \gamma_i &=& - \beta_{ik} \alpha_{jk} - \alpha_{ik} \beta_{jk} \label{rel2fer} \\
\alpha_{ij} \gamma_k + \gamma_k \alpha_{ij} & = & 0 \label{rel3fer} \\
\beta_{ij} \gamma_k + \gamma_k \beta_{ij} & = & 0 \label{rel4fer} \\
\alpha_{ij} \gamma^*_k + \gamma^*_k \alpha_{ij} & = & 0 \label{rel5fer} \\
\beta_{ij} \gamma^*_k + \gamma^*_k \beta_{ij} & = & 0 \label{rel6fer}
\eea
and any two elements from the set
$\alpha_{ij}, \beta_{ij}, \alpha^*_{ij}, \beta^*_{ij}$ commute.

The set of matrices $M$ obeying the above relations form the group
of inhomogeneous transformations of fermions. We call this quantum group
the fermionic inhomogeneous orthogonal quantum group \FIO
since it is an inhomogeneous extension of the orthogonal group where
the inhomogeneous part exhibits fermionic behavior. This
symmetry group, like its sister \BISp, is not a classical group
but is a quantum group with a Hopf algebra structure. Similar to
the case with \BISp, this Hopf algebra has an explicit $R$-matrix
representation and the coproduct, counit and coinverse are defined
as:
\bea
\Delta(M) & = & M \dot{\otimes} M \label{coproductfer} \\
\epsilon(M) & = & I \label{counitfer} \\
S(M) & = & M^{-1} \label{antipodefer} \quad . \eea

\subsection{Subgroups}
We have shown that there is a rich sub(quantum)group structure for
\BISp and it should naturally follow that there should be a similarly
rich sub(quantum)group structure for the fermionic counterpart \FIO.

In this subsection this sub(quantum)group structure will be explored using
relations similar to the ones considered for \BISp:
\begin{subequations}
\bea
\delta_{ij} - \alpha_{ik}\alpha^*_{jk} - \beta_{ik} \beta^*_{jk} = - \beta_{ik} \alpha_{jk} - \alpha_{ik} \beta_{jk} &=& 0 \label{subafer} \\
\gamma_i &=& 0  \label{subbfer} \\
\beta_{ij} &=& 0  \label{subcfer} \\
\alpha_{ij} &=& 0  \quad. \label{subdfer}
\eea
\end{subequations}
The implication of each of these relations will be explored in
the following subsections.

\subsubsection{Inhomogeneous Subsupergroup}
The relation \eqref{subafer}:
\[
\delta_{ij} - \alpha_{ik}\alpha^*_{jk} - \beta_{ik} \beta^*_{jk}
= - \beta_{ik} \alpha_{jk} - \alpha_{ik} \beta_{jk} = 0
\]
by virtue of \eqref{rel1fer} and \eqref{rel2fer} implies that $\gamma_i
\gamma^*_j + \gamma^*_j \gamma_i = 0$ and $\gamma_i \gamma_j +
\gamma_j \gamma_i = 0$, i.e. that the inhomogeneous transformation
parameters are anticommutative variables.

Thus we end up with an
inhomogeneous orthogonal algebra where the inhomogeneous
parameters are grassmannian variables giving us the Grassmannian
inhomogeneous orthogonal group, $GrIO(2d, \IR)$, as the resulting
subgroup of \FIO. This subgroup of \FIO can also be considered
as an inhomogeneous supergroup. Actually, more generally, the
transformation elements $\alpha_{ij}$, $\beta_{ij}$, $\alpha^*_{ij}$
and $\beta^*_{ij}$ anticommute with
$\gamma_i$, $\gamma^*_i$ and the \FIO matrices $M$ are multiplied
with each other using the standard tensor product. One can also show
that $\alpha_{ij}$, $\beta_{ij}$, $\alpha^*_{ij}$ and $\beta^*_{ij}$ can
be taken to commute with $\gamma_i$, $\gamma^*_i$ provided that
the matrices $M$ are multiplied with a braided \cite{majid} tensor product,
eg. \beq (A \otimes B)(C \otimes D) = - AC \otimes BD \eeq whenever $B$ and
$C$ are both fermionic. This approach is similar to the approach that was
taken with ACSA to obtain a Hopf algebra structure. As a result of this
redefinition the treatment of the transformation elements corresponds to
the usual superalgebra approach, i.e. that the elements $\alpha_{ij}$,
$\beta_{ij}$, $\alpha^*_{ij}$ and $\beta^*_{ij}$ are bosonic and the elements
$\gamma_i$, $\gamma^*_i$ are fermionic.

\subsubsection{Homogeneous Subgroup}
The relation \eqref{subbfer}:
\[
\gamma_i = 0
\]
practically gets rid of the inhomogeneous part of the
transformation and also implies the relation \eqref{subafer} considered
in the previous subsection.

This gives us the subgroup in the previous subsection
without the inhomogeneous part which is basically the
classical orthogonal group $O(2d, \IR)$.

\subsubsection{Fermionic Inhomogeneous Unitary Quantum Group}
The relation \eqref{subcfer}:
\[
\beta_{ij} = 0
\]
applied to the transformation gets rid of the off-diagonal members
of the homogeneous part of it and leaves us with the following
relation:
\bea
\gamma_i\gamma^*_j + \gamma^*_j\gamma_i &=& \delta_{ij} - \alpha_{ik}\alpha^*_{jk} \\
\gamma_i\gamma_j + \gamma_j\gamma_i &=& 0
\eea

For the homogeneous
part of the transformation, this equation implies $ \delta_{ij} =
\alpha_{ik}\alpha^*_{jk} $ which tells us that the submatrices
$\alpha$ and $\alpha^*$ in equation \eqref{M-matrix} are both
members of $U(d)$. The subgroup we have arrived at thus is an
inhomogeneous quantum group whose homogeneous part is $U(d)$.
For fermions we will name this group the fermionic
inhomogeneous quantum group, $FIU(d)$, since the inhomogeneous part of
the transformation exhibits fermionic behavior.

\subsubsection{Fermion Algebra}
The relation \eqref{subdfer}:
\[
\alpha_{ij} = 0
\]
applied alone onto the transformation gets rid of the diagonal
members of the homogeneous part and prevents such transformations
from forming a (quantum)group since the homogeneous parts of these
set of transformations can never include the identity
transformation.

However, if this relation is applied together with the previous
one, relation \eqref{subcfer}, the resulting relation gets rid of the
whole homogeneous part of the transformation leaving only the
inhomogeneous part and gives us a single relation:
\bea
\gamma_i\gamma^*_j + \gamma^*_j\gamma_i &=& \delta_{ij} \\
\gamma_i\gamma_j + \gamma_j\gamma_i &=& 0
\eea
which
gives us back the fermion algebra, $FA(d)$.

It should be noted that the fermion algebra in $d$ dimensions is isomorphic
to the Clifford algebra in $2d$ dimensions. If one considers operators $\psi_i$
defined as:
\bea
\psi_i =
\begin{cases}
   i (c_{i/2} - c^*_{i/2}) & \text{when $i$ is even} \\
   (c_{(i + 1)/2} + c^*_{(i + 1)/2}) & \text{when $i$ is odd} \\
\end{cases}
\eea
where $i = 1, 2,  \ldots, 2d$ and $c_j$ are elements of the fermion algebra,
then one can show that $\psi_i$ satisfy the Clifford algebra rule:
\beq
\{\psi_i, \psi_j\} = 2\delta_{ij}
\eeq
In order to show this, one only needs to prove the cases:
\beq
\{\psi_i, \psi_j\} =
\begin{cases}
  2\delta_{ij} & \text{$i$ and $j$ odd} \\
  2\delta_{ij} & \text{$i$ and $j$ even} \\
  0            & \text{$i$ odd, $j$ even} \\
\end{cases}
\eeq

For $i$ and $j$ both odd, $\{\psi_i, \psi_j\}$ becomes:
\beq
\begin{split}
\{\psi_i, \psi_j\}
&= (c_{(i + 1)/2} + c^*_{(i + 1)/2}) (c_{(j + 1)/2} + c^*_{(j + 1)/2}) \\
& \phantom{=} + (c_{(j + 1)/2} + c^*_{(j + 1)/2}) (c_{(i + 1)/2} + c^*_{(i + 1)/2}) \\
&= \{c_{(i + 1)/2}, c_{(j + 1)/2}\} + \{c^*_{(i + 1)/2}, c^*_{(j + 1)/2}\} \\
& \phantom{=} + \{c_{(i + 1)/2}, c^*_{(j + 1)/2}\} + \{c^*_{(i + 1)/2}, c_{(j + 1)/2}\} \\
&= 0 + 0 + \delta_{ij} + \delta_{ij} \\
&= 2 \delta_{ij}
\end{split}
\eeq
Similarly, when $i$ and $j$ are both even, one gets:
\beq
\begin{split}
\{\psi_i, \psi_j\}
&= i (c_{i/2} - c^*_{i/2}) i (c_{j/2} - c^*_{j/2}) \\
& \phantom{=} + i (c_{j/2} - c^*_{j/2}) i (c_{i/2} - c^*_{i/2}) \\
&= - \{c_{i/2}, c_{j/2}\} - \{c^*_{i/2}, c^*_{j/2}\} \\
& \phantom{=} + \{c_{i/2}, c^*_{j/2}\} + \{c^*_{i/2}, c_{j/2}\} \\
&= - 0 - 0 + \delta_{ij} + \delta_{ij} \\
&= 2 \delta_{ij}
\end{split}
\eeq
Finally, when $i$ is odd and $j$ is even, the anticommutator becomes:
\beq
\begin{split}
\{\psi_i, \psi_j\}
&= (c_{(i + 1)/2} + c^*_{(i + 1)/2}) i (c_{j/2} - c^*_{j/2}) \\
&  \phantom{=}+ i (c_{j/2} - c^*_{j/2}) (c_{(i + 1)/2} + c^*_{(i + 1)/2}) \\
&= i \{c_{(i + 1)/2}, c_{j/2}\} - i \{c^*_{(i + 1)/2}, c^*_{j/2}\} \\
&  \phantom{=}- i \{c_{(i + 1)/2}, c^*_{j/2}\} + i \{c^*_{(i + 1)/2}, c_{j/2}\} \\
&= 0 - 0 - \delta_{i+1,j} + \delta_{i+1,j} \\
&= 0
\end{split}
\eeq
thereby completing the proof that the operators $\psi_i$ form the elements of a Clifford algebra
of $2d$ dimensions. If one also considers that fact that the definition of $\psi_i$
is invertible and that $c_i$ can also be defined in terms of $\psi_i$, one can
conclude that the algebras $FA(d)$ and $Cliff(2d)$ are isomorphic.

\subsubsection{Sub(quantum)group Diagram}
As a result of the above discussion, we get the sub(quantum)group
diagram:
\[
\begin{CD}
{FIO(2d, \IR)}   @> \eqref{subafer} >> {GrIO(2d, \IR)} @> \eqref{subbfer} >> {O(2d, \IR)} \\
@V\eqref{subcfer}VV                    @V\eqref{subcfer}VV                   @V\eqref{subcfer}VV \\
{FIU(d)}         @> \eqref{subafer} >> {GrIU(d)}       @> \eqref{subbfer} >> {U(d)} \\
@V\eqref{subdfer}VV \\
{FA(d) \approx Cliff(2d)}
\end{CD}
\]
for the for the sub(quantum)groups of \FIO that have been introduced in
this section.

\subsection{Contractions}

It was observed for \BISp that using a suitable contraction one can
obtain new sub(quantum)groups. This should also be possible for \FIO and
the resulting structures will be examined in this section.

Similar to the bosonic treatment, we replace $\gamma_i$ by
$\gamma_i/\sqrt{\hbar}\;$ so that we may consider the case $\hbar
\rightarrow 0$. After this replacement, the equations \eqref{rel1fer}
and \eqref{rel2fer} become:
\bea
\gamma_i \gamma^*_j + \gamma^*_j \gamma_i &=& \hbar(\delta_{ij} - \alpha_{ik}\alpha^*_{jk} - \beta_{ik} \beta^*_{jk}) \\
\gamma_i \gamma_j + \gamma_j \gamma_i &=& \hbar(-\beta_{ik} \alpha_{jk} - \alpha_{ik} \beta_{jk})
\eea
and we consider the
case $\hbar \rightarrow 0$, we get the relations:
\bea
\gamma_i \gamma^*_j + \gamma^*_j \gamma_i &=& 0 \\
\gamma_i \gamma_j + \gamma_j \gamma_i &=& 0
\eea
which imply
that the inhomogeneous part of the transformation are Grassmannian
elements. What makes this case different
from the previous case of subgroups is that the homogeneous part of this
transformation forms a matrix $A$ with non-zero determinant. We
have previously shown that this matrix can be put in real form that
is a member of the general linear group $GL(2d, \IR)$. Since the
homogeneous part of the transformation is the general linear group
and the inhomogeneous part is Grassmannian, we
have the group $GrIGL(2d, \IR)$, the Grassmannian inhomogeneous general linear group
where the inhomogeneous part of the group are Grassmannian.

If we consider the contraction of the subgroups as well then we
should examine the $\hbar \rightarrow 0$ limit after the relations
\eqref{subcfer} and \eqref{subdfer} are applied.

After we apply relation \eqref{subcfer}, we get the subgroup $FIU(d)$
as discussed previously. After the contraction,
again, the inhomogeneous part of this group become
Grassmannian variables. However, as was shown during the contraction
of \BISp, if we apply the previous similarity
transformation on the homogeneous part of the resulting transformation
matrix after contraction, one can see that the
homogeneous part of the transformation is a member of the general
linear group $GL(d, \IC)$. Similarly this gives us $GrIGL(d, \IC)$.

We have previously shown that we get the fermion
algebra after applying both of the relations \eqref{subcfer} and
\eqref{subdfer}. We have also discussed that in this case only the
inhomogeneous part of the transformation survives. After
applying the contraction, the surviving inhomogeneous part of the
transformation turns into Grassmannian variables. Thus
in this case, the contraction of $FA(d)$ gives us $Gr(d, \IC)$,
the $d$ dimensional Grassmann algebra.

As a summary, for the contraction considered combined with the
remaining subgroup relations we get the table:
\[
\begin{CD}
{FIO(2d, \IR)}            @> \hbar \rightarrow 0   >> {GrIGL(2d, \IR)} \\
@V\eqref{subcfer}VV                                    @V\eqref{subcfer}VV \\
{FIU(d)}                  @> \hbar \rightarrow 0   >> {GrIGL(d, \IC)} \\
@V\eqref{subdfer}VV                                    @V\eqref{subdfer}VV \\
{FA(d) \approx Cliff(2d)} @> \hbar \rightarrow 0    >> {Gr(d, \IC)}
\end{CD}
\]

\section{The Fermionic Inhomogeneous
Orthogonal Quantum Group of Odd Dimension}

The bosonic transformation quantum group \BISp can only be defined
in even dimensions and it is not possible to extend this definition
to odd dimension. However, as will be shown in this section, it is
possible to define the fermionic inhomogeneous orthogonal quantum
group of odd dimension. In order to show this, one should first
consider a unitary transformation of the \FIO matrix:
\[
M \rightarrow UMU^{-1}
\]
using the unitary matrix:
\beq U = \left(
\begin{tabular}{cc|c}
$\frac{1}{\sqrt{2}}$ & $\frac{1}{\sqrt{2}}$ & $0$ \\
$\frac{i}{\sqrt{2}}$ & $\frac{-i}{\sqrt{2}}$ & $0$ \\
\hline $0$ & $0$ & $1$
\end{tabular}
\right)
\eeq
If one applies this unitary transformation, it can be seen that
$M$ can be put in the real form:
\beq
\left(
\begin{tabular}{cc|c}
$Re(\alpha + \beta)$ & $Im(\alpha - \beta)$ & $\sqrt{2}Re(\gamma)$ \\
$-Im(\alpha + \beta)$ & $Re(\alpha - \beta)$ & $-\sqrt{2}Im(\gamma)$ \\
\hline $0$ & $0$ & $1$
\end{tabular}
\right) = \left(
\begin{tabular}{c|c}
$A$ & $\Gamma$ \\
\hline $0$ & $1$
\end{tabular}
\right)
\eeq
where $A$ and $\Gamma$ matrices are defined as in
\eqref{M-matrix}, and $Re$ and $Im$ denote the hermitian and
anti-hermitian parts.

Using this form it is not too hard to show that for \FIO,
the transformation relations \eqref{rel1fer} and \eqref{rel2fer}
together transform into the single equation:
\beq
\{\Gamma_i, \Gamma_j\} = \delta_{ij} - A_{ik} A_{jk} \quad,\quad i, j = 1, 2, \ldots , 2d .
\label{relcombined}
\eeq
By extending the range of the indices in this relation to
odd-dimensions it is possible
to define $FIO(2d + 1, \IR)$, the fermionic inhomogeneous
orthogonal algebra of odd dimension.

In order to show the validity of \eqref{relcombined}, one needs
to consider the three cases:
\[
\{\Gamma_i, \Gamma_j\} =
\begin{cases}
  2\{Re(\gamma_i), Re(\gamma_j) \} & \text{for $1 \leq i,j \leq d$} \\
  2\{Im(\gamma_i), Im(\gamma_j) \} & \text{for $d + 1 \leq i,j \leq 2d$} \\
  - 2\{Re(\gamma_i), Im(\gamma_j) \} & \text{for $1 \leq i \leq d$ and $d + 1 \leq j \leq 2d$} \\
\end{cases}
\]
For the case when $1 \leq i,j \leq d$, the above form becomes:
\beq \label{Gamma-case1}
\begin{split}
\{\Gamma_i, \Gamma_j\} &= 2\{Re(\gamma_i), Re(\gamma_j) \} \\
&= \frac12 \left[\{\gamma_i,\gamma_j\} + \{\gamma_i,\gamma^*_j\} + \{\gamma^*_i,\gamma_j\} + \{\gamma^*_i,\gamma^*_j\}\right] \\
&= \frac12 \left[(-\beta_{ik}\alpha_{jk}-\alpha_{ik}\beta_{jk}) + (\delta_{ij} - \alpha_{ik}\alpha^*_{jk} - \beta_{ik}\beta^*_{jk}) \right.\\
&  \phantom{{=} \frac12 \left[\right.}\left. + (\delta_{ij} - \alpha^*_{ik}\alpha_{jk} - \beta^*_{ik}\beta_{jk}) + (-\beta^*_{ik}\alpha^*_{jk}-\alpha^*_{ik}\beta^*_{jk})\right] \\
&= \frac12 \left[2\delta_{ij} - (\alpha_{ik} + \beta^*_{ik})(\alpha^*_{jk} + \beta_{jk}) - (\beta_{ik} + \alpha^*_{ik})(\alpha_{jk} + \beta_{jk})\right] \\
&= \delta_{ij} - \frac12 \left[(\alpha_{ik} + \beta^*_{ik})(\alpha^*_{jk} + \beta_{jk}) + (\beta_{ik} + \alpha^*_{ik})(\alpha_{jk} + \beta_{jk})\right]
\end{split}
\eeq
however, for this case, the form of $A_{ik}A_{jk}$ is:
\beq
\begin{split}
\sum^{2d}_{k = 1} A_{ik} A_{jk} &= \sum^{d}_{k = 1} A_{ik} A_{jk} + \sum^{2d}_{k = d + 1} A_{ik} A_{jk} \\
&= \sum^{d}_{k = 1} \left[Re(\alpha_{ik} + \beta_{ik})Re(\alpha_{jk} + \beta_{jk}) + Im(\alpha_{ik} - \beta_{ik})Im(\alpha_{jk} - \beta_{jk})\right] \\
&= \frac14 \left[(\alpha_{ik} + \beta_{ik} + \alpha^*_{ik} + \beta^*_{ik})(\alpha_{jk} + \beta_{jk} + \alpha^*_{jk} + \beta^*_{jk}) \right. \\
&  \phantom{{=} \frac14 \left[\right.}\left. -(\alpha_{ik} - \beta_{ik} - \alpha^*_{ik} + \beta^*_{ik})(\alpha_{jk} - \beta_{jk} - \alpha^*_{jk} + \beta^*_{jk}) \right] \\
&= \frac14 \left[(\alpha_{ik} + \beta^*_{ik})(\alpha^*_{jk} + \beta_{jk}) + (\alpha_{ik} + \beta^*_{ik})(\alpha_{jk} + \beta^*_{jk}) \right. \\
&  \phantom{{=} \frac14 \left[\right.}\left.+(\alpha^*_{ik} + \beta_{ik})(\alpha_{jk} + \beta_{jk}) + (\alpha^*_{ik} + \beta_{ik}) (\alpha^*_{jk} + \beta^*_{jk}) \right. \\
&  \phantom{{=} \frac14 \left[\right.}\left.+(\alpha_{ik} + \beta^*_{ik})(\alpha^*_{jk} + \beta_{jk}) - (\alpha_{ik} + \beta^*_{ik}) (\alpha_{jk} + \beta^*_{jk}) \right. \\
&  \phantom{{=} \frac14 \left[\right.}\left.+(\alpha^*_{ik} + \beta_{ik})(\alpha_{jk} + \beta_{jk}) - (\alpha^*_{ik} + \beta_{ik}) (\alpha^*_{jk} + \beta^*_{jk}) \right] \\
&= \frac12 \left[(\alpha_{ik} + \beta^*_{ik})(\alpha^*_{jk} + \beta_{jk}) + (\alpha^*_{ik} + \beta_{ik})(\alpha_{jk} + \beta_{jk}) \right]
\end{split}
\eeq
and using this in equation \eqref{Gamma-case1} gives us:
\beq
\begin{split}
\{\Gamma_i, \Gamma_j\} &= \delta_{ij} - \frac12 \left[(\alpha_{ik} + \beta^*_{ik})(\alpha^*_{jk} + \beta_{jk}) + (\beta_{ik} + \alpha^*_{ik})(\alpha_{jk} + \beta_{jk})\right] \\
&= \delta_{ij} - A_{ik}A_{jk} \quad \quad \text{for $1 \leq i,j \leq d$}
\end{split}
\eeq
For the case when $d + 1 \leq i,j \leq 2d$, the above form becomes:
\beq
\begin{split} \label{Gamma-case2}
\{\Gamma_i, \Gamma_j\} &= 2\{Im(\gamma_i), Im(\gamma_j) \} \\
&= - \frac12 \left[\{\gamma_i,\gamma_j\} - \{\gamma_i,\gamma^*_j\} - \{\gamma^*_i,\gamma_j\} + \{\gamma^*_i,\gamma^*_j\}\right] \\
&= - \frac12 \left[(-\beta_{ik}\alpha_{jk}-\alpha_{ik}\beta_{jk}) - (\delta_{ij} - \alpha_{ik}\alpha^*_{jk} - \beta_{ik}\beta^*_{jk}) \right.\\
&    \phantom{{=} - \frac12 \left[\right.}\left. - (\delta_{ij} - \alpha^*_{ik}\alpha_{jk} - \beta^*_{ik}\beta_{jk}) + (-\beta^*_{ik}\alpha^*_{jk}-\alpha^*_{ik}\beta^*_{jk})\right] \\
&= - \frac12 \left[-2\delta_{ij} + (\alpha^*_{ik} - \beta_{ik})(\alpha_{jk} - \beta^*_{jk}) + (\beta^*_{ik} - \alpha_{ik})(\beta_{jk} - \alpha^*_{jk})\right] \\
&= \delta_{ij} - \frac12 \left[(\alpha^*_{ik} - \beta_{ik})(\alpha_{jk} - \beta^*_{jk}) + (\alpha_{ik} - \beta^*_{ik})(\alpha^*_{jk} - \beta_{jk})\right]
\end{split}
\eeq
however, for this case, the form of $A_{ik}A_{jk}$ is:
\beq
\begin{split}
\sum^{2d}_{k = 1} A_{ik} A_{jk} &= \sum^{d}_{k = 1} A_{ik} A_{jk} + \sum^{2d}_{k = d + 1} A_{ik} A_{jk} \\
&= \sum^{d}_{k = 1} \left[Im(\alpha_{ik} + \beta_{ik})Im(\alpha_{jk} + \beta_{jk}) + Re(\alpha_{ik} - \beta_{ik})Re(\alpha_{jk} - \beta_{jk})\right] \\
&= \frac14 \left[-(\alpha_{ik} + \beta_{ik} - \alpha^*_{ik} - \beta^*_{ik})(\alpha_{jk} + \beta_{jk} - \alpha^*_{jk} - \beta^*_{jk}) \right.\\
&  \phantom{{=} \frac14 \left[\right.}\left.+(\alpha_{ik} - \beta_{ik} + \alpha^*_{ik} - \beta^*_{ik})(\alpha_{jk} - \beta_{jk} + \alpha^*_{jk} - \beta^*_{jk}) \right] \\
&= \frac14 \left[(\alpha^*_{ik} - \beta_{ik})(\alpha_{jk} - \beta^*_{jk}) - (\alpha^*_{ik} - \beta_{ik})(\alpha^*_{jk} - \beta_{jk}) \right. \\
&  \phantom{{=} \frac14 \left[\right.}\left.+(\alpha_{ik} - \beta^*_{ik})(\alpha^*_{jk} - \beta_{jk}) - (\alpha_{ik} - \beta^*_{ik})(\alpha_{jk} - \beta^*_{jk}) \right. \\
&  \phantom{{=} \frac14 \left[\right.}\left.+(\alpha^*_{ik} - \beta_{ik})(\alpha_{jk} - \beta^*_{jk}) + (\alpha^*_{ik} - \beta_{ik})(\alpha^*_{jk} - \beta_{jk}) \right. \\
&  \phantom{{=} \frac14 \left[\right.}\left.+(\alpha_{ik} - \beta^*_{ik})(\alpha^*_{jk} - \beta_{jk}) + (\alpha_{ik} - \beta^*_{ik})(\alpha_{jk} - \beta^*_{jk}) \right] \\
&= \frac12 \left[(\alpha^*_{ik} - \beta_{ik})(\alpha_{jk} - \beta^*_{jk}) + (\alpha_{ik} - \beta^*_{ik})(\alpha^*_{jk} - \beta_{jk})\right]
\end{split}
\eeq
and using this in equation \eqref{Gamma-case2} gives us:
\beq
\begin{split}
\{\Gamma_i, \Gamma_j\} &= \delta_{ij} - \frac12 \left[(\alpha^*_{ik} - \beta_{ik})(\alpha_{jk} - \beta^*_{jk}) + (\alpha_{ik} - \beta^*_{ik})(\alpha^*_{jk} - \beta_{jk})\right] \\
&= \delta_{ij} - A_{ik}A_{jk} \quad \quad \text{for $d + 1 \leq i,j \leq 2d$}
\end{split}
\eeq
Finally, for the case when $1 \leq i \leq d$ and $d + 1 \leq j \leq 2d$, the above form becomes:
\beq
\begin{split} \label{Gamma-case3}
\{\Gamma_i, \Gamma_j\} &= - 2\{Re(\gamma_i), Im(\gamma_j) \} \\
&= - \frac1{2i} \left[\{\gamma_i,\gamma_j\} - \{\gamma_i,\gamma^*_j\} + \{\gamma^*_i,\gamma_j\} - \{\gamma^*_i,\gamma^*_j\}\right] \\
&= - \frac1{2i} \left[(-\beta_{ik}\alpha_{jk}-\alpha_{ik}\beta_{jk}) - (\delta_{ij} - \alpha_{ik}\alpha^*_{jk} - \beta_{ik}\beta^*_{jk}) \right.\\
&    \phantom{{=} - \frac1{2i} \left[\right.}\left. + (\delta_{ij} - \alpha^*_{ik}\alpha_{jk} - \beta^*_{ik}\beta_{jk}) - (-\beta^*_{ik}\alpha^*_{jk}-\alpha^*_{ik}\beta^*_{jk})\right] \\
&= - \frac1{2i} \left[(\alpha_{ik} + \beta^*_{ik})(\alpha^*_{jk} - \beta_{jk}) - (\alpha^*_{ik} + \beta_{ik})(\alpha_{jk} - \beta^*_{jk})\right]
\end{split}
\eeq
however, for this case, the form of $A_{ik}A_{jk}$ is:
\beq
\begin{split}
\sum^{2d}_{k = 1} A_{ik} A_{jk} &= \sum^{d}_{k = 1} A_{ik} A_{jk} + \sum^{2d}_{k = d + 1} A_{ik} A_{jk} \\
&= \sum^{d}_{k = 1} \left[-Re(\alpha_{ik} + \beta_{ik})Im(\alpha_{jk} + \beta_{jk}) + Im(\alpha_{ik} - \beta_{ik})Re(\alpha_{jk} - \beta_{jk})\right] \\
&= \frac1{4i} \left[-(\alpha_{ik} + \beta_{ik} + \alpha^*_{ik} + \beta^*_{ik})(\alpha_{jk} + \beta_{jk} - \alpha^*_{jk} - \beta^*_{jk}) \right.\\
&  \phantom{{=} \frac1{4i} \left[\right.}\left.+(\alpha_{ik} - \beta_{ik} - \alpha^*_{ik} + \beta^*_{ik})(\alpha_{jk} - \beta_{jk} + \alpha^*_{jk} - \beta^*_{jk}) \right] \\
&= \frac1{4i} \left[(\alpha_{ik} + \beta^*_{ik})(\alpha^*_{jk} - \beta_{jk}) - (\alpha_{ik} + \beta^*_{ik})(\alpha_{jk} - \beta^*_{jk}) \right. \\
&  \phantom{{=} \frac1{4i} \left[\right.}\left.-(\alpha^*_{ik} + \beta_{ik})(\alpha_{jk} - \beta^*_{jk}) + (\alpha^*_{ik} + \beta_{ik})(\alpha^*_{jk} - \beta_{jk}) \right. \\
&  \phantom{{=} \frac1{4i} \left[\right.}\left.+(\alpha_{ik} + \beta^*_{ik})(\alpha^*_{jk} - \beta_{jk}) + (\alpha_{ik} + \beta^*_{ik})(\alpha_{jk} - \beta^*_{jk}) \right. \\
&  \phantom{{=} \frac1{4i} \left[\right.}\left.-(\alpha^*_{ik} + \beta_{ik})(\alpha_{jk} - \beta^*_{jk}) - (\alpha^*_{ik} + \beta_{ik})(\alpha^*_{jk} - \beta_{jk}) \right] \\
&= \frac12    \left[(\alpha_{ik} + \beta^*_{ik})(\alpha^*_{jk} - \beta_{jk}) - (\alpha^*_{ik} + \beta_{ik})(\alpha_{jk} - \beta^*_{jk})\right]
\end{split}
\eeq
and using this in equation \eqref{Gamma-case3} gives us:
\beq
\begin{split}
\{\Gamma_i, \Gamma_j\} &= - \frac1{2i} \left[(\alpha_{ik} + \beta^*_{ik})(\alpha^*_{jk} - \beta_{jk}) - (\alpha^*_{ik} + \beta_{ik})(\alpha_{jk} - \beta^*_{jk})\right] \\
&= - A_{ik}A_{jk} \quad \quad \text{for $1 \leq i \leq d$ and $d + 1 \leq j \leq 2d$}
\end{split}
\eeq
which completes the derivation of equation \eqref{relcombined}.

Similar to the analysis that went into finding the
sub(quantum)groups of \FIO, we can also investigate the
sub(quantum)groups of $FIO(2d + 1, \IR)$. The sub(quantum)group
relations in this case, however, are more restricted owing to the
fact that the algebra is not described anymore by submatrices of
the $A$ matrix but is rather described by the whole matrix itself.
Thus, we cannot set $\alpha$ or $\beta$ to zero on their own, we
can only restrict the algebra by setting the whole of $A$ to zero.
Thus the resulting sub(quantum)algebra relations are:
\begin{subequations}
\bea
\delta_{ij} - A_{ik}A_{jk} = 0 \label{subaferodd} \\
\Gamma_i = 0 \label{subbferodd} \\
A_{ij} = 0 \label{subcferodd}
\eea
\end{subequations}
which, through a similar analysis to the even dimensional case,
gives us the following sub(quantum)group diagram:
\[
\begin{CD}
{FIO(2d + 1, \IR)} @>\eqref{subaferodd}>> {GrIO(2d + 1, \IR)} @>\eqref{subbferodd}>> {O(2d + 1, \IR)} \\
@V\eqref{subcferodd}VV\\
{Cliff(2d + 1)}
\end{CD}
\]
