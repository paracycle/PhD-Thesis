\section{Bosons and fermions}
The concept of bosonic and fermionic particles is one of the most important concepts
in modern quantum physics. The behavior of large scale matter, from chemical
properties of elements to superconductivity and superfluidity can mostly be
understood by referring to the fermionic or bosonic nature of the quantum
mechanical particles involved in such phenomena. It is for this reason that
understanding the symmetry properties of these phenomena and, motivated by their
importance, trying to find other behavior that mimic them is very meaningful.


Furthermore, while bosonic behavior has a classical counterpart, the
concept of a fermionic particle is one that can only exist in the quantum
domain. This fact makes the study of such behavior even more important.
However, what could be more interesting is the study of other such
constructs that cannot have a classical counterpart. These constructs would
thus belong solely in the quantum domain and could help us understand phenomena
that are strictly quantum mechanical in nature.


There is a strong relation between the spin properties of a particle and
the particle being a boson or a fermion. In fact, it is a proven fact of
quantum physics that integer spin particles are bosons and half-integer spin
particles are fermions. This is most often referred to as the {\it spin-statistics
theorem} in quantum mechanics and is a very interesting fact since it implies
a relationship between two concepts that seem to be totally unrelated. This
strong relation between the bosonic/fermionic nature of a particle and its spin
makes the angular momentum algebra also very central in quantum physics.


Before we start investigating such matters, it would be apt to give an
overview of the state of bosons and fermions and the angular momentum
algebra as it has been studied up to now.


When the harmonic oscillator is studied in a quantum mechanical manner,
one arrives at the relation:
\beq
a \adag - \adag a = 1 \label{bosonic comm rel}
\eeq
to describe the system. The Hamiltonian of this system is given
by $\frac{\hbar \omega}{2} (a \adag + \adag a)$. The spectrum of this
Hamiltonian, which in turn gives us the allowable energy levels of the
quantum harmonic oscillator, can be obtained easily by introducing the
hermitian operator $N = \adag a$ which satisfies the following relations
with $a$ and $\adag$:
\bea
 [ N , \adag ] &=& \adag \\ [0pt]
 [ N , a ]     &=& a
\eea
where $[\;,\; ]$ denotes the usual commutator. By observing the fact
that the Hamiltonian is nothing but $\hbar \omega (N + \frac12)$, one
can see that one can get the states that correspond to the energy levels
as eigenvectors $\ket{n}$, of the operator $N$. The action of $\adag$
and $a$ on such an eigenvector $\ket{n}$ is found to be:
\bea
\adag \ket{n} &=& \sqrt{n + 1} \ket{n + 1} \\
a \ket{n} &=& \sqrt{n} \ket{n - 1}
\eea
Due to the fact that the operator $N$ is a positive hermitian operator,
its eigenvalues, namely $n$, cannot be negative. For a given positive
value of $n$, however, one can construct states with eigenvalues $n-1$,
$n-2$, $n-3$, and so on, by repeatedly applying the operator $a$ on the
original state. This sequence of eigenvalues will contain negative values
eventually for any given finite $n$ unless it is an integer. In that case,
the sequence will end at the eigenvalue $0$ since a further application of
the operator $a$ on that state will give us the zero vector of the Hilbert
space which is not a physically observable state and is thus a state
out of our domain.


As a result of this study one finds that the values of $n$, the
eigenvalues of the operator $N$, begin from $0$ and increase by
$1$ every time $\adag$ is applied on the relevant state and that the energy
levels of the quantum harmonic oscillator are given by $\hbar \omega (n + \frac12)$.
The operators $\adag$ and $a$ turn out to be  operators that create and destroy,
respectively, one quanta of energy and for this reason they are usually called
creation and annihilation operators.


Even though this operator algebra seems to only describe the quantum harmonic
oscillator, when one studies quantum field theory, this algebra comes up as the
algebra of the Fourier coefficients of the field operator describing a bosonic
particle. Each normal mode of a quantum field behaves as if it is an independent
harmonic oscillator and for that reason we have a separate set of creation and
annihilation operators for each of these modes. In that setting, the operators
$\adag_p$ and $a_p$, which now carry a continuous momentum index, are interpreted
as the operators that create and destroy, respectively, one bosonic particle
of such a field with momentum $p$.


For fermionic particles the story is a little bit more different.
In 1925, Pauli first proposed his {\it exclusion principle} to explain the behavior
of electrons in an atom. According to this principle, no two electrons could
exist in the same quantum state and it was for this reason that electrons could not
all occupy the lowest energy state in the atomic orbitals but instead had to line up
the energy levels in a well ordered manner. The implication of this principle to the
electron gas was first considered by Fermi and Dirac and it is for this reason that
particles that obey these statistics are called {\it fermions}. In 1926 Dirac noted that
the exclusion principle could also apply to other particles by relating bosons and
fermions to the symmetry of the many-particle wavefunction. If the wavefunction
changes sign upon exchange of two particles then those particles would be fermions
and they would be bosons if the wavefunction did not change sign. This treatment
effectively implies the Pauli exclusion principle since if there were to be two
fermionic particles occupying the same quantum state, then upon their exchange
the wavefunction would change sign; on the other hand, we expect the wavefunction
to be identical to the original one before the exchange since nothing must have
changed about the quantum state of the system. For this reason the original
wavefunction can be nothing but zero if it is to be equal to its negative in
this manner. Thus, by contradiction, one can show that no two fermions can exist
in the same quantum state. It was only later, in 1928, that Jordan and Wigner
proposed that in order to treat fermions in quantum field theory, their field
operators had to anticommute so that the wavefunction could be antisymmetric.
They showed that a consistent second-quantization of fermions implied anticommutation
relations on the field operators. This is turn implies that the Fourier coefficients
of the field operators that belong to a normal mode also obey anticommutation
relations instead of the commutation relations that the bosonic creation and
annihilation operators obey.


In this work, we would like to give an alternative derivation of this algebra
by only starting from the Pauli exclusion principle and assuming that fermionic
particles also have creation and annihilation operators just like the bosonic
particles. If this is the case then Pauli exclusion principle tells us that we
cannot create a second fermion in the same quantum state, i.e. that $(\adag)^2$,
and in turn $a^2$, should be $0$. This relation, however, is not compatible with
the commutation relation (\ref{bosonic comm rel}) and thus should be supplemented
with another kind of relation. If we define the operator $K$ as the anticommutator
of $a$ and $\adag$:
\beq
K \equiv a \adag + \adag a
\eeq
then we find that $K$ is a central element of the algebra, since:
\bea
\adag K &= \adag (a \adag + \adag a ) =& \adag a \adag \\
K \adag &= (a \adag + \adag a ) \adag =& \adag a \adag
\eea
which implies that $K$ commutes with $\adag$ and similarly with $a$, thus making
it a central element of the algebra. The central operator $K$ can be written as
a multiple of the identity $k\I1$ and if we rescale the operators $a$ and $\adag$
by $1/\sqrt{k}$, we arrive at the fermion anticommutator algebra:
\bea
a^2 &=& 0 \\
a \adag + \adag a &=& 1
\eea


This derivation of the fermion algebra also shows clearly that the physically more
important relation is the fact that the square of the annihilation operator is zero,
since the other relation follows from this fact. In literature, it is often the
case that only the anticommutation relation is presented as describing fermionic
particles, completely omitting the other, more important, relation. This is
usually falsely motivated by the assumption that the anticommutation relation
uniquely describes a fermionic system just as the commutation relation alone
describes a bosonic system. However,
without the first relation, the anticommutation relation alone describes a
completely different system which still has two states but is not equivalent to the
fermionic system.


A study of this fermion algebra, similar to the boson algebra, shows that, again,
a hermitian positive-definite number operator $N = \adag a$ can be defined and has
eigenvalues $0$ and $1$ that correspond to the states $\ket{0}$ and $\ket{1}$,
respectively. In harmony with our original assumption, the operator $\adag$ takes
the state $\ket{0}$ to the state $\ket{1}$ thus fulfilling the interpretation of
it as a creation operator. Similarly, the operator $a$ acts as an annihilation
operator of the algebra.


\section{Quantum groups and Hopf algebras}


The discovery of quantum groups has historically been motivated by the study
of quantization of non-linear completely integrable systems \cite{sklyanin}. The study of
such systems has shown that some non-linear completely integrable systems that
possess group symmetries, when quantized, acquire a different kind of symmetry;
a symmetry under quantum groups. By definition quantum groups are non-commutative and
non-cocommutative Hopf algebras and thus the physical importance of quantum groups and
Hopf algebras, in general, is very great since the aforementioned discovery.


In order to give an overview of the definition of a Hopf algebra and the motivations
behind these definition, we will start from the definition of an associative algebra
and starting form that definition give definitions of coalgebra, bialgebra and Hopf
algebra.


\subsection{Associative algebras}
In abstract mathematics, an associative algebra $A$ over a field $F$ is
defined to be a vector space over
$F$ with an $F$ bilinear multiplication $m: A \otimes A \rightarrow A$ (where the image of
$(x, y) \in A \otimes A$ which is $m(x,y)$ is usually written as $xy$) such that the associativity
law:
\beq
(xy)z = x(yz) \quad \text{for all $x,y,z \in A$}
\eeq
is satisfied. This associativity condition can also be written without reference to any of
the elements of the algebra $A$ by first considering that the condition is equivalent to:
\beq
m\circ(m(x, y), z) = m\circ(x, m(y,z)) \quad \text{for all $x,y,z \in A$},
\eeq
where $\circ$ denotes functional composition,
and then realizing that the {\it for all} condition can be expressed as:
\beq
m\circ(m(A \otimes A) \otimes A) = m\circ(A \otimes m(A \otimes A)) \quad .
\eeq
If we further define the identity operator on $A$ as $id(x) = x$
for all $x \in A$, then we can write the above form as:
\beq
m\circ(m \otimes id) (A \otimes A \otimes A) = m\circ(id \otimes m) (A \otimes A \otimes A) \quad ,
\eeq
where it is obvious that we can drop the $A \otimes A \otimes A$ terms from both sides
of the equation without losing the expressive power of the relation. Thus we end up with:
\beq
m\circ(m \otimes id) = m\circ(id \otimes m)
\eeq
for the definition of associativity of the product on an algebra $A$. This form of {\it element free
notation}, where appropriate, will be used in this work from this point on.


An associative algebra is called unital if the algebra $A$ contains an identity element $1$ such
that $1x = x1 = x$ for all $x \in A$. Such a unital algebra is also a ring and contains all the
elements of the field $F$ by identifying an element $k$ of the field with the algebra element $k1$.
This identification can be expressed as the existence of a unit map $\eta: F \rightarrow A$ which
has the property:
\beq
m \circ (id \otimes \eta) = s = m \circ (\eta \otimes id)
\eeq
where $s$ is the scalar multiplication $s: F \otimes A \rightarrow A$ such that $s(k, x) = kx$.
Since $F \otimes A$ is isomorphic to the original algebra $A$, the above relation is sometimes
written with $id$ in place of $s$ with scalar multiplication being implicitly understood.


As a result, we can see that the definition of a unital associative algebra is a
vector space over a field $F$ with two operations, $m: A \otimes A \rightarrow A$
and $\eta: F \rightarrow A$ defined such that the operations satisfy:
\begin{align}
m\circ(m \otimes id) &= m\circ(id \otimes m) \\
m \circ (id \otimes \eta) = &\; id = m \circ (\eta \otimes id)
\end{align}
These relations can also be written as the condition that the following diagrams
commute:
\pagebreak
\begin{figure}[!h]
  \[
  \xymatrix@=100pt{
    A \otimes A \otimes A \ar[r]^-*+{m \circ id} \ar[d]^-*+{id \circ m}& A \otimes A \ar[d]^-*+{m}\\
    A \otimes A \ar[r]^-*+{m} & A \\
  }
  \]
  \caption{Associativity in an algebra $A$}
  \label{assoc-algebra}
\end{figure}
\begin{figure}[!h]
  \[
  \xymatrix@=100pt{
    F \otimes A \cong A \cong A \otimes F
       \ar[r]^-*+{id \circ \eta}
       \ar[d]^-*+{\eta \circ id}
       \ar[dr]^-*+{id} & A \otimes A \ar[d]^-*+{m}\\
    A \otimes A \ar[r]^-*+{m} & A \\
  }
  \]
  \caption{Existence of unit in the algebra $A$}
  \label{unit-algebra}
\end{figure}




\subsection{Coalgebras}


The primary motivation for coalgebras stem from the study of the effect of the multiplication and unity
operators defined on an algebra on the dual of that algebra. The dual $A^*$ of an algebra $A$
is defined to be the set of all linear maps from $A$ to $F$. By this definition, the dual
of an algebra is a vector space provided that the addition and scalar multiplication is
defined as:
\begin{align}
(\phi + \psi)(x) & = \phi(x) + \psi(x) \\
(k\phi)(x) & = k \phi(x)
\end{align}
for all $\phi, \psi \in A^*$, $x \in A$ and $k \in F$. The dual does not
naturally carry any of the algebra structure of the original algebra and, in general,
is not itself an algebra. For this
reason, it is very natural to inquire about the effect of multiplication in $A$ on
the dual $A^*$. For this we consider:
\beq
\phi(xy) = \phi(m(x \otimes y))
\eeq
for $\phi \in A^*$ and $x,y \in A$. This form, in general, is not equal to $\phi(x) \phi(y)$
but it should be possible
to write it as a tensor product in terms of other elements of $A^*$ valued at $x \otimes y$. The
possibility of this can be shown if $A$ is finite-dimensional. In general, the multiplication
$m: A \otimes A \rightarrow A$ yields a linear map on the dual $\Delta: A^* \rightarrow
(A \otimes A)^*$. However, if $A$ is finite-dimensional, $(A \otimes A)^*$ is naturally isomorphic
to $(A^* \otimes A^*)$ and for that reason the map on the dual can be written as $\Delta: A^*
\rightarrow A^* \otimes A^*$. This map is called to coproduct. In terms of the coproduct, the
above relation becomes:
\beq
\phi(xy) = \phi(m(x \otimes y)) = \Delta(\phi)(x \otimes y)
\eeq
Similarly, the action of the unit map $\eta: F \rightarrow A$ yields a linear map on the
dual $\epsilon: A^* \rightarrow F$, which is called the counit. The action of the counit is
as follows:
\beq
\phi(k1) = \phi(\eta(k)) = \epsilon(\phi)k
\eeq
for $\phi \in A^*$ and $k \in F$. Thus, we see that the multiplication and unit maps on $A$
naturally define the coproduct and counit maps on the dual $A^*$. Furthermore, the associativity
and existence of unit conditions on the algebra $A$ implies certain conditions on the maps defined
on the dual $A^*$. The structure we have thus arrived at is called a coalgebra and the dual
of an algebra $A$ becomes a coalgebra.


Formally, the definition of a coalgebra $C$ is a vector space over a field $F$ together with
two linear maps:
\begin{itemize}
    \item Coproduct: $\Delta: C \rightarrow C \otimes C$
    \item Counit: $\epsilon: C \rightarrow F$
\end{itemize}
such that the conditions:
\begin{align}
(id \otimes \Delta) \circ \Delta & = (\Delta \otimes id) \circ \Delta \\
(id \otimes \epsilon) \circ \Delta = &\; id = (\epsilon \otimes id) \circ \Delta
\end{align}
are satisfied. The first of these conditions is called the coassociativity condition and is
equivalent to the fact that Figure \ref{coassoc-coalgebra} is commutative.
\begin{figure}[!h]
  \[
  \xymatrix@=100pt{
    C \ar[r]^-*+{\Delta} \ar[d]^-*+{\Delta}& C \otimes C \ar[d]^-*+{id \otimes \Delta}\\
    C \otimes C \ar[r]^-*+{\Delta \otimes id} & C \otimes C \otimes C \\
  }
  \]
  \caption{Coassociativity in a coalgebra $C$}
  \label{coassoc-coalgebra}
\end{figure}
Similarly, the second condition is called the existence of counit and is equivalent to the
commutativity of Figure \ref{counit-coalgebra}.
\begin{figure}[!h]
  \[
  \xymatrix@=100pt{
    C
       \ar[r]^-*+{\Delta}
       \ar[d]^-*+{\Delta}
       \ar[dr]^-*+{id} & C \otimes C \ar[d]^-*+{id \otimes \epsilon}\\
    C \otimes C \ar[r]^-*+{\epsilon \otimes id} & F \otimes C \cong C \cong C \otimes F \\
  }
  \]
  \caption{Existence of counit in the coalgebra $C$}
  \label{counit-coalgebra}
\end{figure}


\subsection{Bialgebras}


Formally, a bialgebra $B$  over a field $F$ is both a unital associative algebra
and a coalgebra over $F$ such that the coproduct and counit maps are both algebra homomorphisms.
In this respect, the coalgebra structure should be compatible with the algebra structure
of the bialgebra. We will also show that the same statement can be expressed from the
opposite point of view, ie. that the algebra structure of the bialgebra should be
compatible with the coalgebra structure. For this reason, the product and the unit maps
should, equivalently, be algebra homomorphisms.


Before analyzing the implications of the compatibility condition, we should define
$m_{B \otimes B}$ which is the product defined on $B \otimes B$ using the product
defined on $B$. The map $m_{B \otimes B}: (B \otimes B) \otimes (B \otimes B) \rightarrow B \otimes B$
is a formalization of the product rule $(a\otimes b)(c \otimes d) = (ac) \otimes (bd)$ and for this
reason the action of this map is defined by:
\beq
m_{B \otimes B}((a\otimes b) \otimes (c \otimes d)) = m(a \otimes c) \otimes m(b \otimes d) \quad .
\eeq
One should notice that the definition of this product involves a permutation of the order of the terms $b$ and $c$.
Using this fact and defining the permutation operator $\tau: B \otimes B \rightarrow B \otimes B$ by:
\beq
\tau(a \otimes b) = b \otimes a \quad ,
\eeq
we can rewrite the action of the product map on $B \otimes B$ as:
\beq
\begin{split}
m_{B \otimes B}((a \otimes b) \otimes (c \otimes d))
 & = m(a \otimes c) \otimes m(b \otimes d) \\
 & = (m \otimes m) (a \otimes c \otimes b \otimes d) \\
 & = (m \otimes m) \circ (id \otimes \tau \otimes id) (a \otimes b \otimes c \otimes d) \quad .
\end{split}
\eeq
As a result, we find that $m_{B \otimes B}$ is defined in terms of the product map on $B$ as:
\beq
m_{B \otimes B} = (m \otimes m) \circ (id \otimes \tau \otimes id) \quad .
\eeq


The statement that the coproduct map is a algebra homomorphism implies that:
\begin{align}
\Delta(ab) = \Delta(m(a \otimes b)) & = m_{B \otimes B}(\Delta(a) \otimes \Delta(b)) \\
\Delta(1) & = 1 \otimes 1
\end{align}
These two equations say that the action of the coproduct map respects both the product and the
unit of the algebra structure in the bialgebra. Similarly, the condition that the counit
is an algebra homomorphism implies:
\begin{align}
\epsilon(ab) = \epsilon(m(a \otimes b)) &= m(\epsilon(a) \otimes \epsilon(b)) = \epsilon(a) \epsilon(b) \\
\epsilon(1) &= 1
\end{align}


The content of these relations that define a bialgebra
can again be expressed by the commutative diagrams in Figures
\ref{coproduct-comp-product-bialgebra} and \ref{coproduct-comp-unit-bialgebra} for the homomorphism
conditions on the coproduct and Figures \ref{counit-comp-product-bialgebra} and
\ref{counit-comp-unit-bialgebra} for the homomorphism
conditions on the counit.
\begin{figure}[!h]
  \[
  \xymatrix@=100pt{
    B \otimes B \ar[r]^-*+{\Delta \otimes \Delta} \ar[d]^-*+{m}& (B \otimes B) \otimes (B \otimes B) \ar[d]^-*+{m_{B \otimes B}}\\
    B \ar[r]^-*+{\Delta} & B \otimes B\\
  }
  \]
  \caption{Compatibility of the coproduct with the product on the bialgebra $B$}
  \label{coproduct-comp-product-bialgebra}
\end{figure}


\begin{figure}[!h]
  \[
  \xymatrix@=60pt{
    F \cong F \otimes F \ar[dr]^-*+{\eta} \ar[rr]^-*+{\eta \otimes \eta} &                       & B \otimes B\\
                                                                         & B \ar[ur]^-*+{\Delta} &            \\
  }
  \]
  \caption{Compatibility of the coproduct with the unit on the bialgebra $B$}
  \label{coproduct-comp-unit-bialgebra}
\end{figure}


\begin{figure}[!h]
  \[
  \xymatrix@=60pt{
    B \otimes B \ar[dr]^-*+{m} \ar[rr]^-*+{\epsilon \otimes \epsilon} &                         & F \otimes F \cong F \\
                                                                      & B \ar[ur]^-*+{\epsilon} &                     \\
  }
  \]
  \caption{Compatibility of the counit with the product on the bialgebra $B$}
  \label{counit-comp-product-bialgebra}
\end{figure}


\begin{figure}[!h]
  \[
  \xymatrix@=60pt{
    F \ar[dr]^-*+{\eta} \ar[rr]^-*+{id} &                         & F  \\
                                        & B \ar[ur]^-*+{\epsilon} & \\
  }
  \]
  \caption{Compatibility of the counit with the unit on the bialgebra $B$}
  \label{counit-comp-unit-bialgebra}
\end{figure}
\pagebreak
One can see from these commutative diagrams, that the diagrams are completely symmetric with
respect to the coalgebra and algebra maps. In other words, one can see that these diagrams
can also be read as the coalgebra homomorphism conditions of the product and the unit maps
of the algebra structure of the bialgebra $B$. The only diagram that does not explicitly exhibit
this symmetry is Figure \ref{coproduct-comp-product-bialgebra}. This diagram, however, can be
written in an explicitly symmetric way by using the definition of $m_{B\otimes B}$ to produce the
commutative diagram shown in Figure \ref{coproduct-comp-product-bialgebra-symm}. This way
the content of all the diagrams can be read both as the compatibility of the coalgebra maps
on the algebra structure and the compatibility of the algebra maps on the coalgebra structure of
the bialgebra $B$.


\begin{figure}[!h]
  \[
  \xymatrix@=70pt{
                                                                   & B  \ar[dr]^-*+{\Delta} & \\
    B \otimes B \ar[ur]^-*+{m} \ar[d]^-*+{\Delta \otimes \Delta}            &  & B \otimes B \\
    B \otimes B \otimes B \otimes B \ar[rr]^-*+{id \otimes \tau \otimes id} &  & B \otimes B \otimes B \otimes B \ar[u]^-*+{m \otimes m}\\
  }
  \]
  \caption{Compatibility of the coproduct with the product on the bialgebra $B$}
  \label{coproduct-comp-product-bialgebra-symm}
\end{figure}




\subsection{Hopf algebras}


A Hopf algebra $H$ is basically a bialgebra, ie. both a unital associative algebra and a coalgebra, with an additional
structure called the coinverse (or the antipode) which is a linear map $S: H \rightarrow H$
such that the diagram in Figure \ref{antipode-hopf} is commutative.


\begin{figure}[!h]
  \[
  \xymatrix@=50pt{
    H \otimes H \ar[rr]^-*+{S \otimes id}                        &                    & H \otimes H \ar[d]^-*+{m}\\
    H \ar[u]^-*+{\Delta} \ar[d]^-*+{\Delta} \ar[r]^-*+{\epsilon} & F \ar[r]^-*+{\eta} & H \\
    H \otimes H \ar[rr]^-*+{id \otimes S}                        &                    & H \otimes H \ar[u]^-*+{m}\\
  }
  \]
  \caption{Definition of coinverse on the Hopf algebra $H$}
  \label{antipode-hopf}
\end{figure}

\pagebreak
In order to write the concept of a coinverse more
explicitly, we will introduce Sweedler's \cite{sweedler} notation which can be considered to
be the analogue of Einstein summation convention for coproducts. Given an element $c$ of a coalgebra,
there exists elements $c_{(1)}^i$ and $c_{(2)}^i$ in the coalgebra such that:
\beq
\Delta(c) = \sum_i c_{(1)}^i \otimes c_{(2)}^i \quad .
\eeq
Using Sweedler's notation, this can be abbreviated to:
\beq
\Delta(c) = \sum_c c_{(1)} \otimes c_{(2)}
\eeq
and in the sumless version of Sweedler's notation, it further becomes:
\beq
\Delta(c) = c_{(1)} \otimes c_{(2)}
\eeq
Thus the coinverse map $S$ can also be expressed as:
\beq
S(c_{(1)})c_{(2)} = m(S(c_{(1)}) \otimes c_{(2)}) = \epsilon(c) 1 = m(c_{(1)} \otimes S(c_{(2)})) = c_{(1)}S(c_{(2)})
\eeq

The notion commutativity in a Hopf algebra is defined by the commutativity of the product
map of the algebra structure. An algebra is commutative if and only if the product map
satisfies the relation:
\beq
m = m \circ \tau
\eeq
so that the order of multiplying terms in the product does not matter. In terms of
elements of the algebra this relation becomes:
\beq
m(a \otimes b) = m(b \otimes a)
\eeq
for all elements $a,b$ of the algebra. Similarly, the
notion of cocommutativity in a Hopf algebra is defined by the cocommutativity of the
coproduct map of the coalgebra structure. A coalgebra is cocommutative if and only if
the coproduct map satisfies the relation:
\beq
\Delta = \tau \circ \Delta
\eeq
so that the order of terms in the outcome of the coproduct does not matter. In
terms of elements of the coalgebra and using sumless Sweedler's notation, this
relation implies:
\beq
\Delta(c) = c_{(1)} \otimes c_{(2)} = c_{(2)} \otimes c_{(1)}
\eeq
for all elements $c$ of the coalgebra.

There are various examples of Hopf algebras. Out of these the most important examples
are the group algebras and universal enveloping algebras of Lie algebras. Given a
group $G$, the group algebra $FG$ is a unital associative algebra over the field $F$.
It becomes a Hopf algebra, if we define the coproduct, counit and coinverse maps by:
\begin{align}
\Delta(g) & = g \otimes g \\
\epsilon(g) & = 1 \\
S(g) & = g^{-1}
\end{align}
for all $g \in G$. In this instance the resulting Hopf algebra is always 
cocommutative  (since $g \otimes g = g \otimes g$) and is
commutative depending on the original group $G$ being abelian or not. If the
underlying group $G$ is abelian, the resulting Hopf algebra is both cocommutative
and commutative. Otherwise, it is cocommutative but noncommutative.

Similarly, given a Lie algebra $g$ over a field $F$, its universal enveloping algebra
$U(g)$ is a unital associative algebra. This algebra $U(g)$ becomes a Hopf algebra
if we define the coproduct, counit and the coinverse maps as:
\begin{align}
\Delta(x) & = 1 \otimes x + x \otimes 1 \\
\epsilon(x) & = 0 \\
S(x) & = -x
\end{align}
for all $x \in U(g)$. Notice that the coproduct rule is not only compatible with the
product on the universal enveloping algebra but it is also compatible with the
antisymmetric product defined on the Lie algebra itself. This Hopf algebra is
cocommutative but noncommutative.

Quantum groups are, loosely, defined as Hopf algebras that are neither commutative nor
cocommutative. As such, they are important in non-commutative geometry. The reason for
this stems from the observation that in order to study geometry on a manifold $M$,
it is possible to work with the algebra of functions $A = C(M)$ on $M$ which is a Hopf
algebra. Thus, one can continue studying Hopf algebras, including noncommutative
and noncocommutative ones, and do geometry with them even though the underlying manifold
does not exist anymore in a conventional sense. They are called quantum groups because of
a similar reasoning stating that a standard algebraic group is well described by the
Hopf algebra of regular functions on the algebraic group and that a deformed version
of the Hopf algebra should, in some sense, describe a deformed, quantized version of the
algebraic group. In essence, identifying these quantized algebraic groups with their
Hopf algebras one can study them in full generality and make a theory of these quantum
groups.

\section{Quantum matrix groups}

A quantum matrix group is defined by a set of $n$ x $n$ matrices $M$:
\beq
M =
\left(
\begin{array}{cccc}
a_{11} & a_{12} & \ldots & a_{1n}  \\
a_{21} & \ddots &        &  \vdots \\
\vdots &        & \ddots &  \vdots \\
a_{n1} & \ldots & \ldots & a_{nn}
\end{array}
\right)
\eeq
such that every element of the matrix belong to a Hopf algebra $H$.
The matrix group defined in this way naturally becomes a Hopf algebra with the
coproduct, counit and coinverse of the matrix algebra being defined as:
\bea
\triangle(M) &=& M \dot{\otimes} M \label{qmg-coproduct}\\
\epsilon(M) &=& \I1_n \label{qmg-counit} \\
S(M) &=& M^{-1} \label{qmg-coinverse}
\eea
where $\dot{\otimes}$ stands for the operation
where when the matrix multiplication is performed the matrix
elements are multiplied using the tensor product is instead of the
normal product and $\I1_n$ stands for the $n$ x $n$ unit matrix.
The relations above imply the definitions of the
coproduct, counit and coinverse of the matrix elements:
\bea
\triangle(\alpha_{ij}) &=& \sum_k \alpha_{ik} \otimes \alpha_{kj} \\
\epsilon(\alpha_{ij}) &=& \delta_{ij} \\
\sum_j S(\alpha_{ij}) \alpha_{jk} &=& \delta_{ij} = \sum_j \alpha_{ij} S(\alpha_{jk})
\eea

One of the most important examples of quantum matrix groups is the quantum
group $GL_q(n)$. This quantum group is a quantum subgroup of the bialgebra
$M_q(n)$. An element $T$ of $M_q(n)$ has matrix entries $t_{ij}$ that satisfy:
\begin{align}
t_{ik} t_{il} & = q t_{il} t_{ik} & &\text{for $k < l$} \\
t_{ik} t_{jk} & = q t_{jk} t_{ik} & &\text{for $i < j$} \\
t_{il} t_{jk} & = t_{jk} t_{il} & &\text{for $i  < j$, $k < l$} \\
t_{ik} t_{jl} - t_{jl} t_{ik} & = (q - q^{-1}) t_{il} t_{jk} & &\text{for $i  < j$, $k < l$}
\end{align}
for some $q \in \IC$. One can immediately see that $M_q(n)$ is a
bialgebra with the definitions of coproduct and counit given in
equations \eqref{qmg-coproduct} and \eqref{qmg-counit}. In order to define
the coinverse, one needs to define the inverse of such a matrix and for
this one should define the quantum analogues of the
determinant and the adjoint. For an element $T$ of $M_q(n)$ one defines
the {\it quantum determinant} \cite{krob-leclerc} as:
\beq \label{qdet}
det_q(T) = \sum_{\sigma \in S_n} (-q)^{i(\sigma)} t_{1\sigma(1)} \cdots t_{n\sigma(n)}
\eeq
where $S_n$ is the symmetric group on ${1, \cdots, n}$ and $i(\sigma)$ is the number
of adjacent transpositions in the permutation $\sigma$. Similar to normal matrices,
one can also define the
{\it quantum adjoint} matrix $adj(T) = (a_{ij})$ such that:
\beq \label{qadjoint}
a_{ij} = (-q)^{j - i} det_q(T^{ij})
\eeq
where $T^{ij}$ stands for the $(n-1)$ x $(n-1)$ matrix obtained from $T$ by
deleting the $i$th row and the $j$th column. Just as in classical matrices, one
has:
\beq
T \cdot adj(T)^T = adj(T)^T \cdot T = det_q(T) \I1_n
\eeq
which yields the inverse, and in turn the coinverse, of such a quantum matrix as:
\beq \label{qinverse}
S(T) = T^{-1} = det_q^{-1}(T) adj(T)^T
\eeq
This coinverse is obviously only defined when $det_q(T)$ is invertible for all $T$ in the
quantum matrix group. The quantum subgroup of $M_q(n)$ that satisfies this condition
is called $GL_q(n)$, the quantum general linear group of dimension $n$ and this structure
is a Hopf algebra since the coinverse is now defined. In analogy with
the classical matrix groups, one can further restrict the determinant of such matrices
to be equal to $1$ and obtain the quantum subgroup $SL_q(n)$, the special linear quantum
group of dimension $n$.

\section{Quantum group invariance of an algebra}
Given an algebra $\mathcal{A}$ with generators $a_1$, $a_2$, $\ldots$, $a_n$, one can construct the column matrix:
\beq
A =
\left(
\begin{array}{c}
a_1 \\
a_2 \\
\vdots \\
a_n
\end{array}
\right)
\eeq
\section{Summary}

This work is divided into four chapters. In the first chapter, an introduction was given to
the basic concepts of boson and fermions and basic mathematical structures related to the
succeeding chapters were introduced.

In the second chapter, the anticommuting spin algebra (ACSA) will be introduced. In that section
it will be shown that the invariance group of ACSA is $SO_{q = -1}(3)$ and that the representations
of ACSA show great similarity to the representations of its sister spin algebra. Finally, the exact
relationship between ACSA and the spin algebra will be examined and a braided Hopf algebra structure
for ACSA will be introduced.

The third chapter deals entirely with the inhomogeneous invariance (quantum) groups of the
boson and fermion algebras. The bosonic inhomogeneous symplectic quantum group and the
fermionic inhomogeneous orthogonal group will be introduced to describe these invariance
conditions. In the subsections of this chapter, the sub(quantum)groups and contractions of
these new quantum groups will be studied. Finally, the fermionic inhomogeneous orthogonal
group will defined in odd number of dimensions and its sub(quantum)groups will be studied.

The fourth and final chapter is reserved for concluding remarks about the body of work that
has been introduced in this study.
