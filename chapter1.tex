The concept of bosonic and fermionic particles is one of the most important concepts
in modern quantum physics. The behavior of large scale matter, from chemical properties
of elements to superconductivity and superfluidity can mostly be understood by
referring to the fermionic or bosonic nature of the quantum mechanical particles
involved in such processes. It is for this reason that understanding the symmetry
properties of these phenomena and, motivated by their importance, trying to find
other behavior that mimic them is very meaningful.

Furthermore, while bosonic behavior has a classical counterpart, the concept of a fermionic
particle is one that can only exist in the quantum domain. This fact makes the study
of such behavior even more important. However, what could be more interesting is the
study of other such constructs that cannot have a classical counterpart. These constructs
would thus belong solely in the quantum domain and could help us understand phenomena that
are strictly quantum mechanical in nature.

There is a strong relation between the spin properties of a particle and the particle being
a boson or a fermion. In fact, it is a proven fact of quantum physics that integer spin
particles are bosons and half-integer spin particles are fermions. This is most often referred
to as the "spin-statistics theorem" in quantum mechanics and is a very interesting fact since
it implies a relationship between two concepts that seems to be totally unrelated. This
strong relation between the bosonic/fermionic nature of a particle and its spin make the
angular momentum algebra also very central in quantum physics.

Before we start investigating such matters, it would be apt to give an overview of the
state of bosons and fermions and the angular momentum algebra as it has been studied up to
now.

When the harmonic oscillator is studied in a quantum mechanical manner, one arrives
at the relation:
\beq
a \adag - \adag a = 1
\eeq
for the system the Hamiltonian of which is given by $\hbar \omega (a \adag + \adag a)$. The spectrum
of this Hamiltonian, which in turn gives us the allowable energy levels of the quantum
harmonic oscillator, can be obtained easily by introducing the hermitian operator
$N = \adag a$ which has the following relations with $a$ and $\adag$:
\bea
 [ N , \adag ] &=& \adag \\ [0pt]
 [ N , a ]     &=& a
\eea
where $[\;,\; ]$ denotes the usual commutator. By observing the fact that the
Hamiltonian is nothing but $N + \frac12$, one can see that one can get the energy levels
states as eigenvectors $\ket{n}$, of the operator $N$. The action of $\adag$ and $a$ on such an
eigenvector $\ket{n}$ can be found to be:
\bea
\adag \ket{n} &=& \sqrt{n + 1} \ket{n + 1} \\
a \ket{n} &=& \sqrt{n} \ket{n - 1}
\eea
which in turn implies that the values of $n$, the eigenvalues of the operator $N$,
begin from $0$ and increase by $1$ every time $\adag$ is applied on the relevant
state. As a result of this study one finds that the energy levels of the quantum
harmonic oscillator is given by $\hbar \omega (n + \frac12)$ and that the operators
$\adag$ and $a$ are operators that create and destroy, respectively,
one quanta of energy. For this reason they are usually called creation and annihilation operators.

Even though this operator algebra seems to only describe the quantum harmonic oscillator,
when one studies quantum field theory, this algebra comes up as the algebra of
of the Fourier coefficients of the field operator of a bosonic particle. In that setting,
the operators $\adag_p$ and $a_p$, which now carry a continuous momentum index, are
interpreted as the operators that create and destroy, respectively, one bosonic
particle of such a field with momentum $p$.

For fermionic particle fields as similar treatment again yields creation and annihilation
operators as the Fourier coefficients of the field operator; however, the algebra obeyed
by these operators is not the same as the algebra of the quantum harmonic oscillator. Instead,
the defining relations of this algebra are:
\bea
a \adag + \adag a &=& 1 \\
a a &=& 0
\eea
which, through a similar treatment, yields the eigenvalues $0$ and $1$ for the operator
$N$.
