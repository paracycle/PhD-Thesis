\section{Bosons and fermions}
The concept of bosonic and fermionic particles is one of the most important concepts
in modern quantum physics. The behavior of large scale matter, from chemical
properties of elements to superconductivity and superfluidity can mostly be
understood by referring to the fermionic or bosonic nature of the quantum
mechanical particles involved in such phenomena. It is for this reason that
understanding the symmetry properties of these phenomena and, motivated by their
importance, trying to find other behavior that mimic them is very meaningful.

Furthermore, while bosonic behavior has a classical counterpart, the
concept of a fermionic particle is one that can only exist in the quantum
domain. This fact makes the study of such behavior even more important.
However, what could be more interesting is the study of other such
constructs that cannot have a classical counterpart. These constructs would
thus belong solely in the quantum domain and could help us understand phenomena
that are strictly quantum mechanical in nature.

There is a strong relation between the spin properties of a particle and
the particle being a boson or a fermion. In fact, it is a proven fact of
quantum physics that integer spin particles are bosons and half-integer spin
particles are fermions. This is most often referred to as the "spin-statistics
theorem" in quantum mechanics and is a very interesting fact since it implies
a relationship between two concepts that seem to be totally unrelated. This
strong relation between the bosonic/fermionic nature of a particle and its spin
makes the angular momentum algebra also very central in quantum physics.

Before we start investigating such matters, it would be apt to give an
overview of the state of bosons and fermions and the angular momentum
algebra as it has been studied up to now.

When the harmonic oscillator is studied in a quantum mechanical manner,
one arrives at the relation:
\beq
a \adag - \adag a = 1 \label{bosonic comm rel}
\eeq
to describe the system. The Hamiltonian of this system is given
by $\frac{\hbar \omega}{2} (a \adag + \adag a)$. The spectrum of this
Hamiltonian, which in turn gives us the allowable energy levels of the
quantum harmonic oscillator, can be obtained easily by introducing the
hermitian operator $N = \adag a$ which satisfies the following relations
with $a$ and $\adag$:
\bea
 [ N , \adag ] &=& \adag \\ [0pt]
 [ N , a ]     &=& a
\eea
where $[\;,\; ]$ denotes the usual commutator. By observing the fact
that the Hamiltonian is nothing but $\hbar \omega (N + \frac12)$, one
can see that one can get the states that correspond to the energy levels
as eigenvectors $\ket{n}$, of the operator $N$. The action of $\adag$
and $a$ on such an eigenvector $\ket{n}$ is found to be:
\bea
\adag \ket{n} &=& \sqrt{n + 1} \ket{n + 1} \\
a \ket{n} &=& \sqrt{n} \ket{n - 1}
\eea
Due to the fact that the operator $N$ is a positive hermitian operator,
its eigenvalues, namely $n$, cannot be negative. For a given positive
value of $n$, however, one can construct states with eigenvalues $n-1$,
$n-2$, $n-3$, and so on, by repeatedly applying the operator $a$ on the
original state. This sequence of eigenvalues will contain negative values
eventually for any given finite $n$ unless it is an integer. In that case,
the sequence will end at the eigenvalue $0$ since a further application of
the operator $a$ on that state will give us the zero vector of the Hilbert
space which is not a physically observable state and is thus a state
out of our domain.

As a result of this study one finds that the values of $n$, the
eigenvalues of the operator $N$, begin from $0$ and increase by
$1$ every time $\adag$ is applied on the relevant state and that the energy
levels of the quantum harmonic oscillator are given by $\hbar \omega (n + \frac12)$.
The operators $\adag$ and $a$ turn out to be  operators that create and destroy,
respectively, one quanta of energy and for this reason they are usually called
creation and annihilation operators.

Even though this operator algebra seems to only describe the quantum harmonic
oscillator, when one studies quantum field theory, this algebra comes up as the
algebra of the Fourier coefficients of the field operator describing a bosonic
particle. Each normal mode of a quantum field behaves as if it is an independent
harmonic oscillator and for that reason we have a separate set of creation and
annihilation operators for each of these modes. In that setting, the operators
$\adag_p$ and $a_p$, which now carry a continuous momentum index, are interpreted
as the operators that create and destroy, respectively, one bosonic particle
of such a field with momentum $p$.

For fermionic particles the story is a little bit more different.
In 1925, Pauli first proposed his "exclusion principle" to explain the behavior
of electrons in an atom. According to this principle, no two electrons could
exist in the same quantum state and it was for this reason that electrons could not
all occupy the lowest energy state in the atomic orbitals but instead had to line up
the energy levels in a well ordered manner. The implication of this principle to the
electron gas was first considered by Fermi and Dirac and it is for this reason that
particles that obey these statistics are called "fermions". In 1926 Dirac noted that
the "exclusion principle" could apply to other particles by relating bosons and
fermions to the symmetry of the many-particle wavefunction. If the wavefunction
changes sign upon exchange of two particles then those particles would be fermions
and they would be bosons if the wavefunction did not change sign. This treatment
effectively implies the Pauli exclusion principle since if there were to be two
fermionic particles occupying the same quantum state, then upon their exchange
the wavefunction would change sign; on the other hand, we expect the wavefunction
to be identical to the original one before the exchange since nothing must have
changed about the quantum state of the system. For this reason the original
wavefunction can be nothing but zero if it is to be equal to its negative in
this manner. Thus, by contradiction, one can show that no two fermions can exist
in the same quantum state. It was only later, in 1928, that Jordan and Wigner
proposed that in order to treat fermions in quantum field theory, their field
operators had to anticommute so that the wavefunction could be antisymmetric.
They showed that a consistent second-quantization of fermions implied anticommutation
relations on the field operators. This is turn implies that the Fourier coefficients
of the field operators that belong to a normal mode also obey anticommutation
relations instead of the commutation relations that the bosonic creation and
annihilation operators obey.

In this work, we would like to give an alternative derivation of this algebra
by only starting from the Pauli exclusion principle and assuming that fermionic
particles also have creation and annihilation operators just like the bosonic
particles. If this is the case then Pauli exclusion principle tells us that we
cannot create a second fermion in the same quantum state, i.e. that $(\adag)^2$,
and in turn $a^2$, should be $0$. This relation, however, is not compatible with
the commutation relation (\ref{bosonic comm rel}) and thus should be supplemented
with another kind of relation. If we define the operator $K$ as the anticommutator
of $a$ and $\adag$:
\beq
K \equiv a \adag + \adag a
\eeq
then we find that $K$ is a central element of the algebra, since:
\bea
\adag K &= \adag (a \adag + \adag a ) =& \adag a \adag \\
K \adag &= (a \adag + \adag a ) \adag =& \adag a \adag
\eea
which implies that $K$ commutes with $\adag$ and similarly with $a$, thus making
it a central element of the algebra. The central operator $K$ can be written as
a multiple of the identity $k\I1$ and if we rescale the operators $a$ and $\adag$
by $1/\sqrt{k}$, we arrive at the fermion anticommutator algebra:
\bea
a^2 &=& 0 \\
a \adag + \adag a &=& 1
\eea

This derivation of the fermion algebra also shows clearly that the physically more
important relation is the fact that the square of the annihilation operator is zero,
since the other relation follows from this fact. In literature, it is often the
case that only the anticommutation relation is presented as describing fermionic
particles, completely omitting the other, more important, relation. This is
usually falsely motivated by the assumption that the anticommutation relation
uniquely describes a fermionic system just as the commutation relation alone
describes a bosonic system. As we show in Appendix A of this work, however,
without the first relation, the anticommutation relation alone describes a
completely different system which still has two states but is not equivalent to the
fermionic system.

A study of this fermion algebra, similar to the boson algebra, shows that, again,
a hermitian positive-definite number operator $N = \adag a$ can be defined and has
eigenvalues $0$ and $1$ that correspond to the states $\ket{0}$ and $\ket{1}$,
respectively. In harmony with our original assumption, the operator $\adag$ takes
the state $\ket{0}$ to the state $\ket{1}$ thus fulfilling the interpretation of
it as a creation operator. Similarly, the operator $a$ acts as an annihilation
operator of the algebra.

\section{Quantum groups and Hopf algebras}

The discovery of quantum groups has historically been motivated by the study
of quantization of non-linear completely integrable systems \cite{sklyanin}. The study of
such systems has shown that some non-linear completely integrable systems that
possess group symmetries, when quantized, acquire a different kind of symmetry;
a symmetry under quantum groups. By definition quantum groups are non-commutative and
non-cocommutative Hopf algebras and thus the physical importance of quantum groups and
Hopf algebras, in general, is very great since the aforementioned discovery.

In order to give an overview of the definition of a Hopf algebra and the motivations
behind these definition, we will start from the definition of an associative algebra
and starting form that definition give definitions of coalgebra, bialgebra and Hopf
algebra.

\subsection{Associative Algebras}
In abstract mathematics, an associative algebra $A$ over a field $F$ is
defined to be a vector space over
$F$ with an $F$ bilinear multiplication $m: A \otimes A \rightarrow A$ (where the image of
$(x, y) \in A \otimes A$ which is $m(x,y)$ is usually written as $xy$) such that the associativity
law:
\beq
(xy)z = x(yz) \quad \text{for all $x,y,z \in A$}
\eeq
is satisfied. This associativity condition can also be written without reference to any of
the elements of the algebra $A$ by first considering that the condition is equivalent to:
\beq
m\circ(m(x, y), z) = m\circ(x, m(y,z)) \quad \text{for all $x,y,z \in A$},
\eeq
where $\circ$ denotes functional composition,
and then realizing that the "for all" condition can be expressed as:
\beq
m\circ(m(A \otimes A) \otimes A) = m\circ(A \otimes m(A \otimes A)) \quad .
\eeq
If we further define the identity operator on $A$ as $id(x) = x$
for all $x \in A$, then we can write the above form as:
\beq
m\circ(m \otimes id) (A \otimes A \otimes A) = m\circ(id \otimes m) (A \otimes A \otimes A) \quad ,
\eeq
where it is obvious that we can drop the $A \otimes A \otimes A$ terms from both sides
of the equation without losing the expressive power of the relation. Thus we end up with:
\beq
m\circ(m \otimes id) = m\circ(id \otimes m)
\eeq
for the definition of associativity of the product on an algebra $A$. This form of "element free
notation", where appropriate, will be used in this work from this point on.

An associative algebra is called unital if the algebra $A$ contains an identity element $1$ such
that $1x = x1 = x$ for all $x \in A$. Such a unital algebra is also a ring and contains all the
elements of the field $F$ by identifying an element $a$ of the field with the algebra element $a1$.
This identification can be expressed as the existence of a unit map $\eta: F \rightarrow A$ which
has the property:
\beq
m \circ (id \otimes \eta) = s = m \circ (\eta \otimes id)
\eeq
where $s$ is the scalar multiplication $s: F \otimes A \rightarrow A$ such that $s(k, x) = kx$.
Since $F \otimes A$ is isomorphic to the original algebra $A$, the above relation is sometimes
written with $id$ in place of $s$ with scalar multiplication being implicitly understood.

As a result, we can see that the definition of a unital associative algebra is a
vector space over a field $F$ with two operations, $m: A \otimes A \rightarrow A$
and $\eta: F \rightarrow A$ defined such that the operations satisfy:
\begin{align}
m\circ(m \otimes id) &= m\circ(id \otimes m) \\
m \circ (id \otimes \eta) = &\; id = m \circ (\eta \otimes id)
\end{align}
These relations can also be written as the condition that the following diagrams
commute:
\pagebreak
\begin{figure}[!h]
  \[
  \xymatrix@=100pt{
    A \otimes A \otimes A \ar[r]^-*+{m \circ id} \ar[d]^-*+{id \circ m}& A \otimes A \ar[d]^-*+{m}\\
    A \otimes A \ar[r]^-*+{m} & A \\
  }
  \]
  \caption{Associativity in an algebra $A$}
  \label{assoc-algebra}
\end{figure}
\begin{figure}[!h]
  \[
  \xymatrix@=100pt{
    F \otimes A \cong A \cong A \otimes F
       \ar[r]^-*+{\eta \circ id}
       \ar[d]^-*+{id \circ \eta}
       \ar[dr]^-*+{id} & A \times A \ar[d]^-*+{m}\\
    A \times A \ar[r]^-*+{m} & A \\
  }
  \]
  \caption{Existence of unit in the algebra $A$}
  \label{unit-algebra}
\end{figure}


\subsection{Coalgebras}

\section{Quantum matrix groups}

A quantum matrix group is defined by a set of $n$ x $n$ matrices $M$:
\beq
M =
\left(
\begin{array}{cccc}
a_{11} & a_{12} & \ldots & a_{1n}  \\
a_{21} & \ddots &        &  \vdots \\
\vdots &        & \ddots &  \vdots \\
a_{n1} & \ldots & \ldots & a_{nn}
\end{array}
\right)
\eeq
such that every element of the matrix belong to a Hopf Algebra $\mathcal{H}$.
The matrix group defined in this way naturally becomes a Hopf algebra with the
coproduct, counit and coinverse of the matrix algebra being defined as:
\bea
\triangle(M) &=& M \dot{\otimes} M \\
\epsilon(M) &=& \I1 \\
S(M) &=& M^{-1}
\eea
where $\dot{\otimes}$ stands for the operation
where when the matrix multiplication is performed the matrix
elements are multiplied using the tensor product is instead of the
normal product. The relations above imply the definitions of the
coproduct, counit and coinverse of the matrix elements:
\bea
\triangle(\alpha_{ij}) &=& \sum_k \alpha_{ik} \otimes \alpha_{kj} \\
\epsilon(\alpha_{ij}) &=& \delta_{ij} \\
\sum_j S(\alpha_{ij}) \alpha_{jk} &=& \delta_{ij} = \sum_j \alpha_{ij} S(\alpha_{jk})
\eea

\section{Quantum group invariance of an algebra}
Given an algebra $\mathcal{A}$ with generators $a_1$, $a_2$, $\ldots$, $a_n$, one can construct the column matrix:
\beq
A =
\left(
\begin{array}{c}
a_1 \\
a_2 \\
\vdots \\
a_n
\end{array}
\right)
\eeq
\section{Summary}
