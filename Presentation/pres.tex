\documentclass[pdf,colorBG,slideColor,fyma]{prosper}

\hypersetup{pdfpagemode=FullScreen}

\usepackage{amscd}
\usepackage{amsmath}
\usepackage{amssymb}

% For multi-line comments
\usepackage{verbatim}

% For commutative diagrams
\usepackage{xypic}
\xyoption{all}

\newcommand{\beq}{\[}
\newcommand{\eeq}{\]}
\newcommand{\bea}{\begin{align*}}
\newcommand{\eea}{\end{align*}}

\newcommand{\adag}{a^{\dagger}}
\newcommand{\bdag}{b^{\dagger}}
\newcommand{\ket}[1]{\mid #1\;\rangle}
\newcommand{\bra}[1]{\langle\; #1\mid}
\newcommand{\braket}[2]{\langle\; #1\mid #2\;\rangle}
\newcommand{\delfn}[1]{\delta(#1)}
\newcommand{\kro}[1]{\delta_{#1}}
\newcommand{\anti}[2]{\{#1, #2\}}


\def\IC{\mathbb{C}}
\def\IR{\mathbb{R}}
\def\I1{\mathbb{I}}
\def\II{\mathbb{J}}
\def\FIO{$FIO(2d, \IR)\;$}
\def\BISp{$BISp(2d, \IR)\;$}

\title{QUANTUM GROUP STRUCTURES ASSOCIATED WITH INVARIANCES OF SOME PHYSICAL ALGEBRAS}
\author{Ufuk Kayserilio\~ glu}

\begin{document}
\maketitle
%%%%%%%%%%%%%%%%%%
%%    SLIDE     %%
%%%%%%%%%%%%%%%%%%

\begin{slide}{Motivations}
\begin{itemize}
    \item{Bosons and fermions are important}
    \item{Angular momentum algebra is important}
    \item{Quantum groups (Hopf algebras) will be important}
\end{itemize}
\end{slide}

%%%%%%%%%%%%%%%%%%
%%    SLIDE     %%
%%%%%%%%%%%%%%%%%%

\begin{slide}{Bosons}
The boson algebra arises from the quantization of the harmonic oscillator:
\begin{align*}
a \adag - \adag a & = 1 \\
a a - a a & = 0
\end{align*}
Defining $N$ as $\adag a$ gives:
\begin{align*}
N \ket{n} &= n \ket{n} \\
\adag \ket{n} &= \sqrt{n + 1} \ket{n + 1} \\%
a \ket{n} &= \sqrt{n} \ket{n - 1}
\end{align*}
for $n = 0, 1, 2, \cdots$.
\end{slide}

%%%%%%%%%%%%%%%%%%
%%    SLIDE     %%
%%%%%%%%%%%%%%%%%%

\begin{slide}{Bosonic Energy Levels}
\xymatrix@R=20pt@C=40pt{
\ar@{}[rr]^-*+{\vdots} & & \\
\ar@{}[rr]^-*+{\vdots} & & \\
\ar@/^/[u]_*+{\adag} \ar@{-}[rr]^-*+{\ket{2}} & &\ar@/^/[d]^*+{a} \\
\ar@/^/[u]_*+{\adag} \ar@{-}[rr]^-*+{\ket{1}} & &\ar@/^/[d]^*+{a} \\
\ar@/^/[u]_*+{\adag} \ar@{-}[rr]^-*+{\ket{0}} & \ar@{->}[d]^*+{a} & \\
& & \\
\ar@{}[rr]^-*+{\text{0 - Null State}} & &
}
\end{slide}

%%%%%%%%%%%%%%%%%%
%%    SLIDE     %%
%%%%%%%%%%%%%%%%%%

\begin{slide}{Fermions}
The fermion algebra is invented the other way around.

The algebra is defined by:
\begin{align*}
a \adag + \adag a & = 1 \\
a a + a a & = 0
\end{align*}
Defining $N$ as $\adag a$ gives:
\begin{align*}
N \ket{n} &= n \ket{n} \\
\adag \ket{n} &= \sqrt{n + 1} \ket{n + 1} \\%
a \ket{n} &= \sqrt{n} \ket{n - 1}
\end{align*}
for $n = 0$ and $1$.
\end{slide}

%%%%%%%%%%%%%%%%%%
%%    SLIDE     %%
%%%%%%%%%%%%%%%%%%

\begin{slide}{Fermionic Energy Levels}
\xymatrix@R=20pt@C=40pt{
\ar@{}[rr]^-*+{\text{0 - Null State}}& &\\
& \ar@{->}[u]_*+{\adag} & \\
\ar@{-}[rr]^-*+{\ket{1}} & & \ar@/^/[d]^*+{a} \\
\ar@/^/[u]_*+{\adag} \ar@{-}[rr]^-*+{\ket{0}} & \ar@{->}[d]^*+{a} &  \\
& & \\
\ar@{}[rr]^-*+{\text{0 - Null State}} & &
}
\end{slide}

%%%%%%%%%%%%%%%%%%
%%    SLIDE     %%
%%%%%%%%%%%%%%%%%%
\overlays{6}{
\begin{slide}{Associative Algebras}
\fromSlide{1}
{
    An algebra $A$ is a vector space over $F$ with multiplication $m$.
    \[
    m(x, y) \equiv m(x \otimes y) \equiv xy
    \]
    We require that, $m$ is associative:
}
\onlySlide*{1}
{
    \[
    (xy)z = x(yz) \quad \text{for all $x, y, z \in A$}
    \]
}
\onlySlide*{2}
{
    \[
    m(m(x,y), z) = m(x, m(y, z)) \quad \text{for all $x, y, z \in A$}
    \]
}
\onlySlide*{3}
{
    \[
    m(m(A \otimes A) \otimes A) = m(A \otimes m(A \otimes A))
    \]
}
\onlySlide*{4}
{
    \[
    m \circ (m \otimes id) (A \otimes A \otimes A) = m \circ (id \otimes m)(A \otimes A \otimes A)
    \]
}
\fromSlide*{5}
{
    \[
    m \circ (m \otimes id) = m \circ (id \otimes m)
    \]
}
\fromSlide{6}
{
\[
  \xymatrix@=60pt{
    A \otimes A \otimes A \ar[r]^-*+{m \circ id} \ar[d]^-*+{id \circ m}& A \otimes A \ar[d]^-*+{m}\\
    A \otimes A \ar[r]^-*+{m} & A \\
  }
\]
}
\end{slide}
}

%%%%%%%%%%%%%%%%%%
%%    SLIDE     %%
%%%%%%%%%%%%%%%%%%
\overlays{2} {
\begin{slide}{Associative Algebras}
The algebra $A$ is called \underline{unital} if there exists an identity
\[
x1 = 1x = x \quad \text{for all $x \in A$}
\]
This is the same thing as defining the map $\eta: F \rightarrow A$
$ \eta(k) = k1 \quad \text{for all $k \in F$} $
which satisfies:
\[ m \circ (id \otimes \eta) = id = m \circ (\eta \otimes id) \]
\onlySlide{2}
{
    \[
    \xymatrix@R=50pt@C=80pt{
    F \otimes A \cong A \cong A \otimes F
        \ar[r]^-*+{id \circ \eta}
        \ar[d]^-*+{\eta \circ id}
        \ar[dr]^-*+{id} & A \otimes A \ar[d]^-*+{m}\\
    A \otimes A \ar[r]^-*+{m} & A \\
    }
    \]
}
\end{slide}
}
%%%%%%%%%%%%%%%%%%
%%    SLIDE     %%
%%%%%%%%%%%%%%%%%%

\begin{slide}{Coalgebras}
\underline{Motivation}: The action of the product on the dual of the algebra
\break
\break
The dual of the algebra $A$ is also a vector space over $F$. The addition
and scalar multiplication are carried over. However, what about the effect of $m$
and $\eta$?
\begin{align*}
\phi(xy) &= \phi(m(x \otimes y)) = \Delta(\phi)(x \otimes y) \\
\phi(k1) &= \phi(\eta(k)) = \epsilon(\phi)k
\end{align*}
Two new maps on $A^*$ are defined:
\begin{itemize}
  \item{The coproduct $\Delta: A^* \rightarrow A^* \otimes A^*$}
  \item{The counit $\epsilon: A^* \rightarrow F$}
\end{itemize}
The associativity and unity conditions are carried over as coassociativity
and counit conditions.
\end{slide}

%%%%%%%%%%%%%%%%%%
%%    SLIDE     %%
%%%%%%%%%%%%%%%%%%

\overlays{2} {
\begin{slide}{Coalgebras}
A coalgebra $C$ is a vector field over $F$ with a coproduct $\Delta$ and a
counit $\epsilon$ such that:
\begin{align*}
(id \otimes \Delta) \circ \Delta & = (\Delta \otimes id) \circ \Delta \\
(id \otimes \epsilon) \circ \Delta = &\; id = (\epsilon \otimes id) \circ \Delta
\end{align*}
\fromSlide{2}
{
    \begin{minipage}{85pt} % A minipage that covers half the page
    \[
    \xymatrix@=50pt{
        C \ar[r]^-*+{\Delta} \ar[d]^-*+{\Delta}& C \otimes C \ar[d]_-*+{id \otimes \Delta}\\
        C \otimes C \ar[r]^-*+{\Delta \otimes id} & C \otimes C \otimes C \\
    }
    \]
    \end{minipage}
    \hspace{2cm} % To get a little bit of space between the figures
    \vline{}
    \begin{minipage}{85pt}
    \[
    \xymatrix@=50pt{
        C
        \ar[r]^-*+{\Delta}
        \ar[d]^-*+{\Delta}
        \ar[dr]^-*+{id} & C \otimes C \ar[d]^-*+{id \otimes \epsilon}\\
        C \otimes C \ar[r]^-*+{\epsilon \otimes id} & F \otimes C \cong C \cong C \otimes F \\
    }
    \]
    \end{minipage}
}
\end{slide}
}

%%%%%%%%%%%%%%%%%%
%%    SLIDE     %%
%%%%%%%%%%%%%%%%%%

\begin{slide}{Bialgebras}
A bialgebra $B$ is both an algebra and a coalgebra over the field $F$.
The coalgebra and the algebra structure should be compatible.
\begin{align*}
\Delta(ab) = \Delta(m(a \otimes b)) & = m_{B \otimes B}(\Delta(a) \otimes \Delta(b)) \\
\Delta(1) & = 1 \otimes 1
\end{align*}
and
\begin{align*}
\epsilon(ab) = \epsilon(m(a \otimes b)) &= m(\epsilon(a) \otimes \epsilon(b)) = \epsilon(a) \epsilon(b) \\
\epsilon(1) &= 1
\end{align*}
where $m_{B \otimes B} = (m \otimes m) \circ (id \otimes \tau \otimes id)$ so that:
\[
m_{B \otimes B}((a \otimes b) \otimes (c \otimes d)) = m(a \otimes c) \otimes m(b \otimes d) = (ac) \otimes (bd)
\]
\end{slide}
%%%%%%%%%%%%%%%%%%
%%    SLIDE     %%
%%%%%%%%%%%%%%%%%%

\overlays{5} {
\begin{slide}{Bialgebra Diagrams}
\onlySlide*{1}
{
    \[
    \xymatrix@=40pt{
                                                                    & B  \ar[dr]^-*+{\Delta} & \\
        B \otimes B \ar[ur]^-*+{m} \ar[d]^-*+{\Delta \otimes \Delta}            &  & B \otimes B \\
        B \otimes B \otimes B \otimes B \ar[rr]^-*+{id \otimes \tau \otimes id} &  & B \otimes B \otimes B \otimes B \ar[u]^-*+{m \otimes m}\\
    }
    \]
}
\onlySlide*{2}
{
  \[
  \xymatrix@=60pt{
    F \cong F \otimes F \ar[dr]^-*+{\eta} \ar[rr]^-*+{\eta \otimes \eta} &                       & B \otimes B\\
                                                                         & B \ar[ur]^-*+{\Delta} &            \\
  }
  \]
}
\onlySlide*{3}
{
  \[
  \xymatrix@=60pt{
    B \otimes B \ar[dr]^-*+{m} \ar[rr]^-*+{\epsilon \otimes \epsilon} &                         & F \otimes F \cong F \\
                                                                      & B \ar[ur]^-*+{\epsilon} &                     \\
  }
  \]
}
\onlySlide*{4}
{
  \[
  \xymatrix@=60pt{
    F \ar[dr]^-*+{\eta} \ar[rr]^-*+{id} &                         & F  \\
                                        & B \ar[ur]^-*+{\epsilon} & \\
  }
  \]
}
\fromSlide{5}
{
All diagrams are symmetric between $m, \eta$ and $\Delta, \epsilon$.
\break
\break
Replace $m$ and $\eta$ with $\Delta$ and $\epsilon$ and reverse the arrows, one
gets the same diagrams.
\break
\break
\begin{minipage}{130pt}
The coalgebra is \break compatible with the \break algebra structure
\end{minipage}
\hspace{10pt}
$\Leftrightarrow$
\hspace{10pt}
\begin{minipage}{130pt}
The algebra is \break compatible with the \break coalgebra structure
\end{minipage}
}
\end{slide}
}
%%%%%%%%%%%%%%%%%%
%%    SLIDE     %%
%%%%%%%%%%%%%%%%%%

\begin{slide}{Hopf Algebras}
A Hopf algebra $H$ is a bialgebra with an additional map $S: H \rightarrow H$
such that $S$ satisfies:
\[
\xymatrix@=30pt{
H \otimes H \ar[rr]^-*+{S \otimes id}                        &                    & H \otimes H \ar[d]^-*+{m}\\
H \ar[u]^-*+{\Delta} \ar[d]^-*+{\Delta} \ar[r]^-*+{\epsilon} & F \ar[r]^-*+{\eta} & H \\
H \otimes H \ar[rr]^-*+{id \otimes S}                        &                    & H \otimes H \ar[u]^-*+{m}\\
}
\]

In terms of elements, this means that if $\Delta(c) = \sum_c c_{(1)} \otimes c_{(2)}$ then
\[
\sum_c S(c_{(1)})c_{(2)} = \epsilon(c) 1 = \sum_c c_{(1)}S(c_{(2)})
\]


\end{slide}
%%%%%%%%%%%%%%%%%%
%%    SLIDE     %%
%%%%%%%%%%%%%%%%%%

\begin{slide}{Commutativity}
An algebra $A$ is commutative if the input to the product is symmetric.
\[
xy = yx
\]
which means:
\[
m(x \otimes y) = m(y \otimes x)
\]
Without reference to elements:
\[
m  = m \circ \tau
\]
\end{slide}
%%%%%%%%%%%%%%%%%%
%%    SLIDE     %%
%%%%%%%%%%%%%%%%%%

\begin{slide}{Cocommutativity}
A coalgebra $C$ is cocommutative if the output of coproduct is symmetric.
\break
\break
If $\Delta(c) = \sum_c c_{(1)} \otimes c_{(2)}$, then for a cocommutative coproduct:
\[
\sum_c c_{(1)} \otimes c_{(2)} = \Delta(c) = \sum_c c_{(2)} \otimes c_{(1)}
\]

Without reference to elements:
\[
\Delta  = \tau \circ \Delta
\]
\end{slide}
%%%%%%%%%%%%%%%%%%
%%    SLIDE     %%
%%%%%%%%%%%%%%%%%%

\begin{slide}{Hopf Algebra examples}
\begin{itemize}
  \item{
    \textbf{\underline{Group Algebra}:} For a group $G$, the group algebra $FG$ over $F$
    is a Hopf algebra with:
        \[
        \left.
        \begin{aligned}
        \Delta(g) & = g \otimes g \\
        \epsilon(g) & = 1 \\
        S(g) & = g^{-1}
        \end{aligned}\;
        \right\} \quad
        \shortstack{
            \text{Cocommutative} \\
            \text{Commutativity depends on $G$.}
        }
        \]
    }
  \item{
    \textbf{\underline{Lie Algebra}:} The universal enveloping algebra $U(g)$ of a Lie
      algebra $g$ is a Hopf algebra with:
        \[
        \left.
        \begin{aligned}
        \Delta(x) & = 1 \otimes x + x \otimes 1 \\
        \epsilon(x) & = 0 \\
        S(x) & = -x
        \end{aligned}\;
        \right\} \quad
        \shortstack{
            \text{Cocommutative} \\
            \text{Noncommutative}
        }
        \]

    }
\end{itemize}
\end{slide}
%%%%%%%%%%%%%%%%%%
%%    SLIDE     %%
%%%%%%%%%%%%%%%%%%

\begin{slide}{Quantum Groups}
The most important Hopf algebras are noncommutative and
noncocommutative ones. These are called quantum groups.

Motivations:
\begin{itemize}
  \item{A manifold $M$ can be studied by looking at $C(M)$ which is a cocommutative Hopf algebra.
   If we study noncocommutative Hopf algebras then we can study noncommutative manifold. Thus, we
   can study noncommutative geometry by looking at quantum groups.
  }
  \item{A group can be described fully by its Hopf algebra. A "deformed" version of the Hopf algebra
   will enable one to study "deformed" ("quantized") groups. The study of "quantized" Hopf algebras is
   a study of "quantum" groups.
  }
\end{itemize}
\end{slide}
%%%%%%%%%%%%%%%%%%
%%    SLIDE     %%
%%%%%%%%%%%%%%%%%%

\begin{slide}{Quasitriangular Hopf algebra}
A quasitriangular Hopf algebra is almost cocommutative. The coproduct satisfies:
\[
\tau \circ \Delta = R \Delta R^{-1}
\]
for some element $R$ in $H \otimes H$ such that
\begin{align*}
(\Delta \otimes id)(R) & = R^{13} R^{13} \\
(id \otimes \Delta)(R) & = R^{13} R^{12}
\end{align*}
where if $R = a_{i} \otimes\; b_{i}$
\begin{align*}
R^{12} & = a_{i} \otimes\; b_{i} \otimes\; 1 \\
R^{13} & = a_{i} \otimes\; 1 \otimes\; b_{i} \\
R^{23} & = 1 \otimes\; a_{i} \otimes\; b_{i}
\end{align*}

\end{slide}
%%%%%%%%%%%%%%%%%%
%%    SLIDE     %%
%%%%%%%%%%%%%%%%%%

\begin{slide}{$R$-Matrix}
For a quasitriangular Hopf algebra, the condition on $R$ leads to the
quantum Yang-Baxter eqaution:
\[
R^{12} R^{13} R^{23} = R^{23} R^{13} R^{12}
\]
Every matrix representation of the Hopf algebra gives a matrix solution to the
QYBE and quasitriangular Hopf algebras can be categorized by finding solutions to QYBE.
\break
\break
Most interesting quantum groups are quasitriangular.
\end{slide}

%%%%%%%%%%%%%%%%%%
%%    SLIDE     %%
%%%%%%%%%%%%%%%%%%

\begin{slide}{Quantum matrix groups}
Set of $n$ x $n$ matrices $M$ such that each entry $m_{ij}$ belongs to a Hopf algebra with:
\begin{align*}
\triangle(M) &= M \dot{\otimes} M \\
\epsilon(M) &= \I1_n \\
S(M) &= M^{-1}
\end{align*}
This is shorthand for:
\begin{align*}
\triangle(m_{ij}) &= \sum_k m_{ik} \otimes m_{kj} \\
\epsilon(m_{ij}) &= \delta_{ij} \\
\sum_j S(m_{ij}) m_{jk} &= \delta_{ij} = \sum_j m_{ij} S(m_{jk})
\end{align*}
\end{slide}

%%%%%%%%%%%%%%%%%%
%%    SLIDE     %%
%%%%%%%%%%%%%%%%%%

\begin{slide}{$M_q(n)$}
An element $T$ of $M_q(n)$ has matrix entries $t_{ij}$ that satisfy:
\begin{align*}
t_{ik} t_{il} & = q t_{il} t_{ik} & &\text{for $k < l$} \\
t_{ik} t_{jk} & = q t_{jk} t_{ik} & &\text{for $i < j$} \\
t_{il} t_{jk} & = t_{jk} t_{il} & &\text{for $i  < j$, $k < l$} \\
t_{ik} t_{jl} - t_{jl} t_{ik} & = (q - q^{-1}) t_{il} t_{jk} & &\text{for $i  < j$, $k < l$}
\end{align*}
for some $q \in \IC$.
\break
\break
The coproduct and counit are defined as in a normal quantum matrix group.
\break
\break
The coinverse might not be defined if $T$ is not invertible.

\end{slide}

%%%%%%%%%%%%%%%%%%
%%    SLIDE     %%
%%%%%%%%%%%%%%%%%%

\begin{slide}{$GL_q(n)$ and $SL_q(n)$}
The $q$-determinant of $T$:
\[
det_q(T) = \sum_{\sigma \in S_n} (-q)^{i(\sigma)} t_{1\sigma(1)} \cdots t_{n\sigma(n)}
\]
and $T^{-1}$ is given by:
\[
T^{-1} = det_q^{-1}(T) adj(T)^T
\]
$\therefore\quad$ $S(T) = T^{-1}$ is defined $\Leftrightarrow$ $det_q(T)$ is invertible.
\break
\break
This condition gives us $GL_q(n)$.
\break
\break
Further require $det_q(T) = 1$, we get $SL_q(n)$.
\end{slide}

%%%%%%%%%%%%%%%%%%
%%    SLIDE     %%
%%%%%%%%%%%%%%%%%%

\begin{slide}{Quantum group invariance of an algebra}
Interesting facts:
\begin{itemize}
  \item{The tensor product of two vector space representations of a Hopf algebra is also
    a vector space representation}
  \item{The product on the Hopf algebra can be extended to the set of vector space
    representations of a Hopf algebra. Thus, Hopf algebra have algebra representations}
\end{itemize}

Thus, given an algebra $A$ which is a representation of $H$, if the invariant elements of $A$ form the
whole of the algebra then $A$ is said to be invariant under the action of the Hopf algebra $H$.
\end{slide}

%%%%%%%%%%%%%%%%%%
%%    SLIDE     %%
%%%%%%%%%%%%%%%%%%

\begin{slide}{The Anticommuting Spin Algebra}
Definition:
\begin{align*}
\anti{J_1}{J_2} & = J_3 \\
\anti{J_2}{J_3} & = J_1 \\
\anti{J_3}{J_1} & = J_2
\end{align*}

A (non-exceptional) Jordan algebra where the Jordan product is
defined by:
\[
A \circ B \equiv \frac12 (AB + BA)
\]
A formal Jordan algebra, in addition to a commutative Jordan
product, also satisfies
\[
A^2\circ(B\circ A) = (A^2\circ B)\circ A
\]
\end{slide}

%%%%%%%%%%%%%%%%%%
%%    SLIDE     %%
%%%%%%%%%%%%%%%%%%

\begin{slide}{The invariance quantum group of ACSA}
Transform the generators $J_i$ to $J'_i$ by:
\[
J'_i = \sum_j\alpha_{ij} J_j
\]
where $\alpha_{ij}$ do not necessarily commute and define:
\[
u_{ijk} =
  \begin{cases}
    1, & \text{for $i \neq j \neq k \neq i$,} \\
    0, & \text{otherwise.}
  \end{cases}
\]
gives us relations among $\alpha_{ij}$ as:
\begin{align*}
\alpha_{in} \alpha_{jn} + \alpha_{jn} \alpha_{in} & = 0  && \text{for $i \neq j$}\\
\alpha_{in} \alpha_{jm} - \alpha_{jm} \alpha_{in}& = 0 && \text{for $i \neq j$ and $n \neq m$}\\
\sum_{n,\;m} \alpha_{in} \alpha_{jm} u_{nml} & = \alpha_{kl} && \text{for $i \neq j \neq k \neq i$}
\end{align*}

\end{slide}

%%%%%%%%%%%%%%%%%%
%%    SLIDE     %%
%%%%%%%%%%%%%%%%%%

\begin{slide}{$SO_{q = -1}(3)$}
We will show that the previous relations are exactly $SO_{q = -1}(3)$ relations.
\break
\break
One can obtain $SO_q(3)$ by imposing on $SL_q(3)$ two conditions:
\begin{itemize}
  \item{Reality condition: $A_{ij} = A^*_{ij}$}
  \item{Unitarity condition: $A^\dagger = A^{-1}$}
\end{itemize}
For $SO_q(3)$ one finds that $q = \pm 1$ only.
\break
\break
The matrix elements of a matrix $A$ in $SO_{q = -1}(3)$ satisfy:
\[
\left.
\begin{aligned}
A_{in} A_{jn} & = - A_{jn} A_{in} \\
A_{in} A_{jm} + A_{im} A_{jn} & = A_{jm} A_{in} + A_{jn} A_{im} \\
A_{im} A_{jn} & = A_{jn} A_{im} \\
A_{in} A_{jm} &= A_{jm} A_{in}
\end{aligned}\;
\right\}\text{for $i \neq j$ and $n \neq m$}
\]

\end{slide}

%%%%%%%%%%%%%%%%%%
%%    SLIDE     %%
%%%%%%%%%%%%%%%%%%

\begin{slide}{The invariance group of ACSA = $SO_{q = -1}(3)$}
First two relations follow from the first and the last equations from $SO_{q = -1}(3)$.
\break
\break
The last invariance relation gives:
\[
\alpha_{kl} = \alpha_{in} \alpha_{jm} + \alpha_{im} \alpha_{jn} = \alpha_{jm} \alpha_{in} +  \alpha_{jn} \alpha_{im}
\]
for $i, j, k$ all different and $n, m, l$ all different.

The definition of the inverse quantum matrix
\[
A^{-1} = det_{q = -1}^{-1}(A) adj(A)^T
\]
implies for $A \in SO_{q = -1}(3)$:
\[
A^T = adj(A)^T \qquad \Rightarrow \qquad A = adj(A)
\]
which results in the relation given above.
\end{slide}

%%%%%%%%%%%%%%%%%%
%%    SLIDE     %%
%%%%%%%%%%%%%%%%%%

\begin{slide}{Representations of ACSA}
Define the operators:
\begin{align*}
J_+ & = J_1 + J_2 \\
J_- & = J_1 - J_2 \\
J^2 & = J_1^2 + J_2^2 + J_3^2
\end{align*}
They obey the following relations:
\begin{align*}
\anti{J_+}{J_3} & = J_3 \\
\anti{J_-}{J_3} & = -J_3 \\
J_+^2 & = J^2 - J_3^2 + J_3 \\
J_-^2 & = J^2 - J_3^2 - J_3
\end{align*}
\end{slide}

%%%%%%%%%%%%%%%%%%
%%    SLIDE     %%
%%%%%%%%%%%%%%%%%%

\begin{slide}{Representations of ACSA}
$J^2$ is central in the algebra and $J_+$, $J_-$ are hermitian.
\break
\break
Label the states with the eigenvalues of $J^2$ and $J_3$:
\begin{align*}
J^2 \ket{\lambda, \mu} & = \lambda \ket{\lambda, \mu} \\
J_3 \ket{\lambda, \mu} & = \mu \ket{\lambda, \mu}
\end{align*}
then we get the action of $J_+$ and $J_-$ as:
\begin{align*}
J_+ \ket{\lambda, \mu} & = \sqrt{\lambda - \mu^2 + \mu} \ket{\lambda,- \mu + 1} \\
J_- \ket{\lambda, \mu} & = \sqrt{\lambda - \mu^2 - \mu} \ket{\lambda,- \mu - 1}
\end{align*}
where $\lambda = j(j+1)$ for some $j$ and $j \geq \mu \geq -j$ in order to have positive
square norms.
\end{slide}

%%%%%%%%%%%%%%%%%%
%%    SLIDE     %%
%%%%%%%%%%%%%%%%%%

\overlays{3} {
\begin{slide}{State diagram of ACSA for $j=0$}
\onlySlide*{1}
{
  \[
  \xymatrix@R=20pt@C=100pt{
    \ar@{}[] \ar@{-}[r]^-*+{0} & \ar@{}[]
  }
  \]
}
\onlySlide*{2}
{
  \[
  \xymatrix@R=20pt@C=100pt{
    \ar@(ul,dl)[]_*+{J_+} \ar@{-}[r]^-*+{0} & \ar@{}[]\hspace{42pt}
  }
  \]
}
\onlySlide*{3}
{
  \[
  \xymatrix@R=20pt@C=100pt{
    \ar@(ul,dl)[]_*+{J_+} \ar@{-}[r]^-*+{0} & \ar@(ur,dr)[]^*+{J_-}
  }
  \]
}
\end{slide}
}
%%%%%%%%%%%%%%%%%%
%%    SLIDE     %%
%%%%%%%%%%%%%%%%%%
\overlays{3} {
\begin{slide}{State diagram of ACSA for $j=1$}
\onlySlide*{1}
{
  \[
  \xymatrix@R=20pt@C=100pt{
    \ar@{}[d] \ar@{-}[r]^-*+{1} & \\
    \ar@{-}[r]^-*+{0} & \ar@{}[d] \\
    \ar@{-}[r]^-*+{-1} &  \\
  }
  \]
}
\onlySlide*{2}
{
  \[
  \xymatrix@R=20pt@C=100pt{
    \ar@/_/[d]_*+{J_+} \ar@{-}[r]^-*+{1} & \hspace{30pt}\\
    \ar@{-}[r]^-*+{0} & \ar@{}[d]\hspace{30pt} \\
    \ar@{-}[r]^-*+{-1} &  \hspace{30pt}\\
  }
  \]
}
\onlySlide*{3}
{
  \[
  \xymatrix@R=20pt@C=100pt{
    \ar@/_/[d]_*+{J_+} \ar@{-}[r]^-*+{1} & \\
    \ar@{-}[r]^-*+{0} & \ar@/^/[d]^*+{J_-} \\
    \ar@{-}[r]^-*+{-1} &  \\
  }
  \]
}
\end{slide}
}
%%%%%%%%%%%%%%%%%%
%%    SLIDE     %%
%%%%%%%%%%%%%%%%%%

\overlays{5} {
\begin{slide}{State diagram of ACSA for $j=2$}
\onlySlide*{1}
{
  \[
  \xymatrix@R=20pt@C=100pt{
    \ar@{}[ddd] \ar@{-}[r]^-*+{2} & \\
    \ar@{}[ddd] \ar@{-}[r]^-*+{1} & \\
    \ar@{-}[r]^-*+{0} & \ar@{}[u] \\
    \ar@{-}[r]^-*+{-1} & \ar@{}[u] \\
    \ar@{-}[r]^-*+{-2} &
  }
  \]
}
\onlySlide*{2}
{
  \[
  \xymatrix@R=20pt@C=100pt{
    \ar@(dl,ul)[ddd]_<<<<<*+{J_+} \ar@{-}[r]^-*+{2} & \hspace{34pt}\\
    \ar@{}[ddd] \ar@{-}[r]^-*+{1} & \hspace{34pt}\\
    \ar@{-}[r]^-*+{0} & \ar@{}[u] \hspace{34pt}\\
    \ar@{-}[r]^-*+{-1} & \ar@{}[u] \hspace{34pt}\\
    \ar@{-}[r]^-*+{-2} & \hspace{34pt}
  }
  \]
}
\onlySlide*{3}
{
  \[
  \xymatrix@R=20pt@C=100pt{
    \ar@(dl,ul)[ddd]_<<<<<*+{J_+} \ar@{-}[r]^-*+{2} & \\
    \ar@{}[ddd] \ar@{-}[r]^-*+{1} & \\
    \ar@{-}[r]^-*+{0} & \ar@{}[u] \\
    \ar@{-}[r]^-*+{-1} & \ar@/_/[u]_-*+{J_-} \\
    \ar@{-}[r]^-*+{-2} &
  }
  \]
}
\onlySlide*{4}
{
  \[
  \xymatrix@R=20pt@C=100pt{
    \ar@(dl,ul)[ddd]_<<<<<*+{J_+} \ar@{-}[r]^-*+{2} & \\
    \ar@{}[ddd] \ar@{-}[r]^-*+{1} & \\
    \ar@{-}[r]^-*+{0} & \ar@/_/[u]_-*+{J_+} \\
    \ar@{-}[r]^-*+{-1} & \ar@/_/[u]_-*+{J_-} \\
    \ar@{-}[r]^-*+{-2} &
  }
  \]
}
\onlySlide*{5}
{
  \[
  \xymatrix@R=20pt@C=100pt{
    \ar@(dl,ul)[ddd]_<<<<<*+{J_+} \ar@{-}[r]^-*+{2} & \\
    \ar@(dl,ul)[ddd]_>>>>>*+{J_-} \ar@{-}[r]^-*+{1} & \\
    \ar@{-}[r]^-*+{0} & \ar@/_/[u]_-*+{J_+} \\
    \ar@{-}[r]^-*+{-1} & \ar@/_/[u]_-*+{J_-} \\
    \ar@{-}[r]^-*+{-2} &
  }
  \]
}
\end{slide}
}
%%%%%%%%%%%%%%%%%%
%%    SLIDE     %%
%%%%%%%%%%%%%%%%%%

\overlays{3} {
\begin{slide}{State diagram of ACSA for $j=\frac{1}{2}$}
\onlySlide*{1}
{
  \[
  \xymatrix@R=20pt@C=50pt{
    \ar@{-}[rrr]^-*+{\frac12} & & &\\
    \ar@{-}[rrr]^-*+{-\frac12} & & & \\
  }
  \]
}
\onlySlide*{2}
{
  \[
  \xymatrix@R=20pt@C=50pt{
    \ar@{}[] \ar@{-}[rr]^-*+{\frac12} & & \\
                                                  & \ar@{-}[rr]^-*+{-\frac12} & & \ar@{}[] \\
  }
  \]
}
\onlySlide*{3}
{
  \[
  \xymatrix@R=20pt@C=50pt{
    \ar@(ul,dl)[]_*+{J_+} \ar@{-}[rr]^-*+{\frac12} & & \\
                                                  & \ar@{-}[rr]^-*+{-\frac12} & & \ar@(ur,dr)[]^*+{J_-} \\
  }
  \]
}
\end{slide}
}
%%%%%%%%%%%%%%%%%%
%%    SLIDE     %%
%%%%%%%%%%%%%%%%%%

\overlays{3} {
\begin{slide}{State diagram of ACSA for $j=\frac{3}{2}$}
\onlySlide*{1}
{
  \[
  \xymatrix@R=20pt@C=50pt{
    \ar@{-}[rrr]^-*+{\frac32} & &  & \\
    \ar@{-}[rrr]^-*+{\frac12} & &  & \\
    \ar@{-}[rrr]^-*+{-\frac12} & &  & \\
    \ar@{-}[rrr]^-*+{-\frac32} &  &  & \\
  }
  \]
}
\onlySlide*{2}
{
  \[
  \xymatrix@R=20pt@C=50pt{
    \ar@{-}[rr]^-*+{\frac32} &                           &  & \\
                                                 & \ar@{-}[rr]^-*+{\frac12}  &  & \\
    \ar@{-}[rr]^-*+{-\frac12}                    &                           &  & \\
                                                 & \ar@{-}[rr]^-*+{-\frac32} &  & \\
  }
  \]
}
\onlySlide*{3}
{
  \[
  \xymatrix@R=20pt@C=50pt{
    \ar@/_/[dd]_*+{J_+} \ar@{-}[rr]^-*+{\frac32} &                           &  & \\
                                                 & \ar@{-}[rr]^-*+{\frac12}  &  & \\
    \ar@{-}[rr]^-*+{-\frac12}                    &                           &  & \\
                                                 & \ar@{-}[rr]^-*+{-\frac32} &  & \ar@/_/[uu]_*+{J_-}\\
  }
  \]
}
\end{slide}
}
%%%%%%%%%%%%%%%%%%
%%    SLIDE     %%
%%%%%%%%%%%%%%%%%%

\begin{slide}{Hopf Algebra Structure with braiding}
$SU(2)$ and $ACSA$ are very closely related:
\break
\break
Observe that if $I_i$ is a generator of $SU(2)$ then:
\[
\tilde{J}_i = - I_i \otimes \sigma_i
\]
satisfies the ACSA relations. Similarly:
\[
\tilde{I}_i = J_i \otimes \sigma_i
\]
satisfies $SU(2)$ relations.
\break
\break
Try to find a coproduct using the coproduct of $SU(2)$:
\[
\Delta(I_i) = 1 \otimes I_i + I_i \otimes 1
\]

\end{slide}

%%%%%%%%%%%%%%%%%%
%%    SLIDE     %%
%%%%%%%%%%%%%%%%%%

\begin{slide}{Hopf Algebra Structure with braiding}
Redefine the action of $\tau$:
\beq
\tau(A \otimes B) = B \otimes A
\eeq
by introducing grading; thus:
\beq
\tau(A \otimes B) = (-1)^{deg\;A\; deg\;B\;} B \otimes A
\eeq
The Hopf algebra relations remain invariant.
\break
\break
Define the degree of $J_1, J_2, J_3$ as $1$ and degree of $1$ as $0$. Then, the coproduct:
\beq
\Delta(J_i) = 1 \otimes J_i + J_i \otimes 1
\eeq
satisfies the Hopf algebra axioms.
\end{slide}

%%%%%%%%%%%%%%%%%%
%%    SLIDE     %%
%%%%%%%%%%%%%%%%%%

\begin{slide}{The Bosonic Inhom. Symplectic Quantum Group}
Consider the multiparticle boson algebra:
\begin{align*}
c_i c_j - c_j c_i & = 0 \\
c_i c^*_j - c^*_j c_i & = \delta_{ij}
\end{align*}
being transformed by:
\beq
\left(
\begin{array}{c}
c' \\
{c^{*}}' \\
1
\end{array}
\right) = \left(
\begin{array}{ccc}
\alpha & \beta & \gamma \\
\beta^* & \alpha^* & \gamma^* \\
0 & 0 & 1
\end{array}
\right) \dot{\otimes} \left(
\begin{array}{c}
c \\
c^* \\
1
\end{array}
\right)
\eeq
\break
\break
The transformation matrix is inhomogeneous and entries are non-commuting.
\end{slide}

%%%%%%%%%%%%%%%%%%
%%    SLIDE     %%
%%%%%%%%%%%%%%%%%%

\begin{slide}{The Bosonic Inhom. Symplectic Quantum Group}
In order for the boson algebra to be invariant, the matrix elements
should satisfy:
\[
\left.
\begin{aligned}
\gamma_i \gamma^*_j - \gamma^*_j \gamma_i &= \delta_{ij} - \alpha_{ik}\alpha^*_{jk} + \beta_{ik} \beta^*_{jk} \\
\gamma_i \gamma_j - \gamma_j \gamma_i &= \beta_{ik} \alpha_{jk} - \alpha_{ik} \beta_{jk} \\
\alpha_{ij} \gamma_k - \gamma_k \alpha_{ij} & = 0 \\
\beta_{ij} \gamma_k - \gamma_k \beta_{ij} & = 0 \\
\alpha_{ij} \gamma^*_k - \gamma^*_k \alpha_{ij} & = 0 \\
\beta_{ij} \gamma^*_k - \gamma^*_k \beta_{ij} & = 0
\end{aligned}
\right|
\text{$\alpha_{ij}, \beta_{ij}, \alpha^*_{ij}, \beta^*_{ij}$ commute}
\]
\break
\break
We call this the Bosonic Inhomogeneous Symplectic Quantum Group, \BISp
\end{slide}

%%%%%%%%%%%%%%%%%%
%%    SLIDE     %%
%%%%%%%%%%%%%%%%%%

\begin{slide}{\BISp - Hopf Algebra structure}
In terms of the matrix $M$:
\beq M = \left(
\begin{array}{cc|c}
\alpha & \beta & \gamma \\
\beta^* & \alpha^* & \gamma^* \\
\hline 0 & 0 & 1
\end{array}
\right)
 =
\left(
\begin{array}{c|c}
A & \Gamma \\
\hline 0 & 1
\end{array}
\right)
\eeq
the Hopf algebra structure is given by:
\begin{align*}
\Delta(M) & = M \dot{\otimes} M \\
\epsilon(M) & = I  \\
S(M) & = M^{-1}
\end{align*}
and as such it is a quantum matrix group. It is also
a quasitriangular Hopf algebra with an $R$-matrix formulation.
\end{slide}

%%%%%%%%%%%%%%%%%%
%%    SLIDE     %%
%%%%%%%%%%%%%%%%%%

\begin{slide}{\BISp - Subgroups}
Impose the following conditions to get the subgroups below:
\begin{subequations}
\begin{align}
\delta_{ij} - \alpha_{ik}\alpha^*_{jk} + \beta_{ik} \beta^*_{jk} = \beta_{ik} \alpha_{jk} - \alpha_{ik} \beta_{jk} & = 0 \label{suba}\\
\gamma_i &= 0  \label{subb}\\
\beta_{ij} &= 0  \label{subc}\\
\alpha_{ij} &= 0 \label{subd}
\end{align}
\end{subequations}
    \[
    \begin{CD}
    {BISp(2d, \IR)} @>\eqref{suba}>> {ISp(2d, \IR)} @>\eqref{subb}>> {Sp(2d, \IR)} \\
    @V\eqref{subc}VV @V\eqref{subc}VV @V\eqref{subc}VV \\
    {BIU(d)} @>\eqref{suba}>> {IU(d)} @>\eqref{subb}>> {U(d)} \\
    @V\eqref{subd})VV \\
    {BA(d)}
    \end{CD}
    \]
\end{slide}

%%%%%%%%%%%%%%%%%%
%%    SLIDE     %%
%%%%%%%%%%%%%%%%%%

\begin{slide}{\BISp - Contractions}
Rescale $\gamma_i$ to $\gamma_i/\sqrt{\hbar}\;$ and take the limit
$\hbar \rightarrow 0$ to study contractions:
    \[
    \begin{CD}
    {BISp(2d, \IR)} @>\hbar \rightarrow 0>> {IGL(2d, \IR)} \\
    @V\eqref{subc}VV @V\eqref{subc}VV \\
    {BIU(d)} @>\hbar \rightarrow 0>> {IGL(d, \IC)} \\
    @V\eqref{subd}VV @V\eqref{subd}VV \\
    {BA(d)} @>\hbar \rightarrow 0>> {\IC^d}
    \end{CD}
    \]
\end{slide}

%%%%%%%%%%%%%%%%%%
%%    SLIDE     %%
%%%%%%%%%%%%%%%%%%

\begin{slide}{The Fermionic Inhomogeneous Group \FIO}
Study the invariance of the multiparticle fermion
algebra:
\begin{align*}
c_i c_j + c_j c_i & = 0 \\
c_i c^*_j + c^*_j c_i & = \delta_{ij}
\end{align*}
under the same inhomogeneous transformation to get:
\[
\left.
\begin{aligned}
\gamma_i \gamma^*_j + \gamma^*_j \gamma_i &= \delta_{ij} - \alpha_{ik}\alpha^*_{jk} - \beta_{ik} \beta^*_{jk} \\
\gamma_i \gamma_j + \gamma_j \gamma_i &= - \beta_{ik} \alpha_{jk} - \alpha_{ik} \beta_{jk} \\
\alpha_{ij} \gamma_k + \gamma_k \alpha_{ij} & = 0 \\
\beta_{ij} \gamma_k + \gamma_k \beta_{ij} & = 0 \\
\alpha_{ij} \gamma^*_k + \gamma^*_k \alpha_{ij} & = 0 \\
\beta_{ij} \gamma^*_k + \gamma^*_k \beta_{ij} & = 0
\end{aligned}
\right|
\text{$\alpha_{ij}, \beta_{ij}, \alpha^*_{ij}, \beta^*_{ij}$ commute}
\]
\end{slide}

%%%%%%%%%%%%%%%%%%
%%    SLIDE     %%
%%%%%%%%%%%%%%%%%%
\begin{slide}{\FIO - Subgroups}
Impose the following conditions to get the subgroups below:
\begin{subequations}
\begin{align}
\delta_{ij} - \alpha_{ik}\alpha^*_{jk} - \beta_{ik} \beta^*_{jk} = - \beta_{ik} \alpha_{jk} - \alpha_{ik} \beta_{jk} &= 0 \label{subafer} \\
\gamma_i &= 0  \label{subbfer} \\
\beta_{ij} &= 0  \label{subcfer} \\
\alpha_{ij} &= 0  \quad. \label{subdfer}
\end{align}
\end{subequations}
    \[
    \begin{CD}
    {FIO(2d, \IR)}   @> \eqref{subafer} >> {GrIO(2d, \IR)} @> \eqref{subbfer} >> {O(2d, \IR)} \\
    @V\eqref{subcfer}VV                    @V\eqref{subcfer}VV                   @V\eqref{subcfer}VV \\
    {FIU(d)}         @> \eqref{subafer} >> {GrIU(d)}       @> \eqref{subbfer} >> {U(d)} \\
    @V\eqref{subdfer}VV \\
    {FA(d) \approx Cliff(2d)}
    \end{CD}
    \]
\end{slide}

%%%%%%%%%%%%%%%%%%
%%    SLIDE     %%
%%%%%%%%%%%%%%%%%%

\begin{slide}{\FIO - Contractions}
Again rescale $\gamma_i$ to $\gamma_i/\sqrt{\hbar}\;$ and take the limit
$\hbar \rightarrow 0$ to study contractions:
    \[
    \begin{CD}
    {FIO(2d, \IR)}            @> \hbar \rightarrow 0   >> {GrIGL(2d, \IR)} \\
    @V\eqref{subcfer}VV                                    @V\eqref{subcfer}VV \\
    {FIU(d)}                  @> \hbar \rightarrow 0   >> {GrIGL(d, \IC)} \\
    @V\eqref{subdfer}VV                                    @V\eqref{subdfer}VV \\
    {FA(d) \approx Cliff(2d)} @> \hbar \rightarrow 0    >> {Gr(d, \IC)}
    \end{CD}
    \]
\end{slide}

%%%%%%%%%%%%%%%%%%
%%    SLIDE     %%
%%%%%%%%%%%%%%%%%%

\begin{slide}{$FIO(2d + 1, \IR)$}
Consider the similarity transformation on $M$ as:
\[
M \rightarrow UMU^{-1}
\]
with
\beq U = \left(
\begin{tabular}{cc|c}
$\frac{1}{\sqrt{2}}$ & $\frac{1}{\sqrt{2}}$ & $0$ \\
$\frac{i}{\sqrt{2}}$ & $\frac{-i}{\sqrt{2}}$ & $0$ \\
\hline $0$ & $0$ & $1$
\end{tabular}
\right)
\eeq
One gets the real form of $M$ after the transformation as:
\beq
\left(
\begin{tabular}{cc|c}
$Re(\alpha + \beta)$ & $Im(\alpha - \beta)$ & $\sqrt{2}Re(\gamma)$ \\
$-Im(\alpha + \beta)$ & $Re(\alpha - \beta)$ & $-\sqrt{2}Im(\gamma)$ \\
\hline $0$ & $0$ & $1$
\end{tabular}
\right) = \left(
\begin{tabular}{c|c}
$A$ & $\Gamma$ \\
\hline $0$ & $1$
\end{tabular}
\right)
\eeq
\end{slide}

%%%%%%%%%%%%%%%%%%
%%    SLIDE     %%
%%%%%%%%%%%%%%%%%%

\begin{slide}{$FIO(2d + 1, \IR)$}
In the real form, for \FIO, the non-trivial relations:
\begin{align*}
\gamma_i \gamma^*_j + \gamma^*_j \gamma_i &= \delta_{ij} - \alpha_{ik}\alpha^*_{jk} - \beta_{ik} \beta^*_{jk} \\
\gamma_i \gamma_j + \gamma_j \gamma_i &= - \beta_{ik} \alpha_{jk} - \alpha_{ik} \beta_{jk} \\
\end{align*}
can be cast into a single equation:
\beq
\{\Gamma_i, \Gamma_j\} = \delta_{ij} - A_{ik} A_{jk} \quad,\quad i, j = 1, 2, \ldots , 2d .
\eeq
Using this form as the defining relation, there is no more restrictions
on the dimension of the fermionic inhomogeneous algebra and one can extend it to
$FIO(2d + 1, \IR)$.
\end{slide}

%%%%%%%%%%%%%%%%%%
%%    SLIDE     %%
%%%%%%%%%%%%%%%%%%

\begin{slide}{$FIO(2d + 1, \IR)$ - Subgroups}
If one considers imposing the relations:
\begin{subequations}
\begin{align}
\delta_{ij} - A_{ik}A_{jk} &= 0 \label{subaferodd} \\
\Gamma_i &= 0 \label{subbferodd} \\
A_{ij} &= 0 \label{subcferodd}
\end{align}
\end{subequations}
one gets the subgroup diagram:
    \[
    \begin{CD}
    {FIO(2d + 1, \IR)} @>\eqref{subaferodd}>> {GrIO(2d + 1, \IR)} @>\eqref{subbferodd}>> {O(2d + 1, \IR)} \\
    @V\eqref{subcferodd}VV\\
    {Cliff(2d + 1)}
    \end{CD}
    \]
\end{slide}

%%%%%%%%%%%%%%%%%%
%%    SLIDE     %%
%%%%%%%%%%%%%%%%%%

\begin{slide}{Conclusions}
\begin{itemize}
  \item{Quantized classical systems or non-classical systems have deformed invariance
    groups.}
  \item{Inhomogeneous quantum groups are novel and interesting. In 3 dimensions, the inhomogeneous
    invariance quantum group of the bosonic oscillator is also a "quantum" symmetry of the phase space.}
  \item{Both the fermionic and the bosonic inhomogeneous quantum groups can be extended to infinite
    dimension by considering their action on fermions-bosons with continuous indices.}
  \item{\BISp and \FIO can be considered to be deformations of bosons and fermions respectively.}
  \item{ACSA and its close resemblance to $SU(2)$ is very interesting}
\end{itemize}
\end{slide}

\end{document}
