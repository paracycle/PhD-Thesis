%% TODO: Write this

%% Conclusions from ACSA
\begin{comment}
\section{Conclusions}

The Anticommutator Spin Algebra, which is a special Jordan
algebra, has many implications. The first of these is the fact
that this algebra is a consistent fermionic algebra which is not a
superalgebra.
% The
%bosonic and fermionic generators $B, F$ in a superalgebra obey:
%\bea
%[B,B] & = & B \nonumber \\ \nonumber
%[F,B] & = & F \\ \nonumber
%\{F,F\} & = & B \nonumber
%\eea
%whereas in ACSA this relation is of the form:
%\[
%\{F,F\} = F
%\]
%which shows how different it is from a superalgebra.
For possible physical applications the right-hand side of the
defining relations (\ref{eqn:defrel1}-\ref{eqn:defrel3}) must also
be supplied with an $\hbar$. In a superalgebra approach where the
$J_i$ are regarded as odd operators, the $\hbar$ on the right-hand
side should also be regarded as an operator anticommuting with the
$J_i$. These models \cite{leites,batalin} result from the
quantization of the odd Poisson bracket. In our approach however,
the concept of grading and therefore an underlying Poisson bracket
formalism does not exist. In particular, there is no Jacobi
identity. Nevertheless, the associative algebra we consider is
consistent with quantum mechanics where physical observables
correspond to hermitian operators and their eigenvalues to
possible results of physical measurement of these observables. It
is for this reason that ACSA suggests a new kind of statistics
which, we believe, will be useful in physics.

The second implication is the important role of quantum groups in
mathematical physics. As we have shown in this paper, the
invariance group of ACSA turns out to be a quantum group. Given
the fact that ACSA is very similar to normal spin algebra and that
the invariance group of spin algebra plays an important role in
physics, the invariance quantum group of ACSA, $SO_{q=-1}(3)$,
becomes a prime example of how central quantum groups have become
in mathematical physics. It is also interesting to note that more
algebras like ACSA can be constructed where the commutators of the
original Lie algebra are turned into anticommutators and that such
algebras might also have invariance quantum groups that is the
same as the invariance group of the original Lie algebra with
$q=-1$. This possibility is open to investigation in a more
general framework.
\end{comment}

% From BISP and FIO paper
\begin{comment}
\section{Discussion}
As we have shown, the boson and fermion algebras can be obtained
as a limit of the inhomogeneous quantum groups \BISp and \FIO. We
can understand why these boson and fermion algebras are not
quantum groups from this construction, since in this limit the
quantum group becomes singular and the antipode does not exist.
Thus we can consider these quantum groups as deformations with a
Hopf algebra structure of their respective particle algebras. This
construction is similar to $q$-deforming the bosonic oscillator to
obtain Pusz-Woronowicz \cite{puszwor} oscillators and then
constructing the $q$-deformed quantum unitary groups as their left
modules. Similarly, in that construction, the $q$-deformed
oscillator can be reobtained as a limit of these $q$-deformed
quantum unitary groups. However, unlike that construction the
quantum groups presented in this paper are inhomogeneous quantum
groups.

Finally, we would like to remark that the widely used field
theoretical generalization achieved by extending the discrete
indices $i, j, k$ to continuous variables together with a
replacement of the Kr\"onecker deltas to Dirac delta functions is
also applicable to the quantum groups we have presented. In this
respect, these quantum groups are also different from the
Pusz-Woronowicz oscillators which cannot be extended to continuous
indices.

We believe that the establishment of these and similar quantum
groups in field theory will be helpful in generalizing methods of
quantization. These approaches will yield a more consistent
approach to interacting field theory and will be the subject of
further investigations.

\end{comment}

