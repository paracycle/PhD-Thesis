%
% This is a template for the fbe_tez.sty, v4.1 (and maybe later).
%
%
\documentclass[12pt]{report}

% Main Thesis Layout style
\usepackage{fbe_tez}

% The following are needed for chapter3
\usepackage{amscd}
\usepackage{amsmath}
\usepackage{amssymb}

% For multi-line comments
\usepackage{verbatim}

% For commutative diagrams
\usepackage{xypic}
\xyoption{all}

\newcommand{\beq}{\begin{equation}}
\newcommand{\eeq}{\end{equation}}
\newcommand{\bea}{\begin{eqnarray}}
\newcommand{\eea}{\end{eqnarray}}
\newcommand{\bean}{\begin{eqnarray*}}
\newcommand{\eean}{\end{eqnarray*}}

\newcommand{\adag}{a^{\dagger}}
\newcommand{\bdag}{b^{\dagger}}
\newcommand{\ket}[1]{\mid #1\;\rangle}
\newcommand{\bra}[1]{\langle\; #1\mid}
\newcommand{\braket}[2]{\langle\; #1\mid #2\;\rangle}
\newcommand{\delfn}[1]{\delta(#1)}
\newcommand{\kro}[1]{\delta_{#1}}
\newcommand{\anti}[2]{\{#1, #2\}}


\def\IC{\mathbb{C}}
\def\IR{\mathbb{R}}
\def\I1{\mathbb{I}}
\def\II{\mathbb{J}}
\def\FIO{$FIO(2d, \IR)\;$}
\def\BISp{$BISp(2d, \IR)\;$}

%
% LaTeX 2.09 users: Replace the above two lines with the following line
%\documentstyle[12pt,fbe_tez]{report}
%
% Declarations
%
%\title{QUANTUM GROUP INVARIANCE OF SOME PHYSICAL ALGEBRAS}
\title{QUANTUM GROUP STRUCTURES ASSOCIATED WITH INVARIANCES OF SOME PHYSICAL ALGEBRAS}
\turkcebaslik{BAZI F\.{I}Z\.{I}KSEL CEB\.{I}RLER\.{I}N
  DE\~G\.{I}\c{S}MEZL\.{I}\~G\.{I} \.{I}LE \.{I}LG\.{I}L\.{I} KUANTUM GRUP YAPILARI}
\degree{
B.S. in Physics, Bo\~ gazi\c ci University, 1997 \\
B.S. in Mathematics, Bo\~ gazi\c ci University, 1997
}
\author{Ufuk Kayserilio\~ glu}
% The \program declaration is needed only for M.S. theses
%\program{FBE Program for which the Thesis is Submitted}
%\subyear{20xx}
%
% Declare relevant examiners
%
\supervisor{Prof. Dr. Metin Ar\i k}
%\cosuperi{Title and Name of Cosupervisor I}
%\cosuperii{Title and Name of Cosupervisor II}
\examineri{Prof. Dr. \"Omer Faruk Day{\i}}
\examinerii{Prof. Dr. Fahr\"{u}nisa Neyzi}
\examineriii{Prof. Dr. Cihan Sa\c{c}l{\i}o\~{g}lu}
\examineriv{Do\c{c}. Dr. Teoman Turgut}
%\examinerv{Title and Name of Examiner}
\dateofapproval{Day.Month.Year}

\begin{document}
\pagenumbering{roman}
%
% Choose the relevant one
%\makemstitle        % For M.S. theses
\makephdtitle      % For Ph.D. theses
%\makeproposaltitle % For Proposals

\makeapprovalpage

\begin{acknowledgements}
\begin{comment}
For my loving wife Emi for all her support; my wise and understanding mentor Metin Bey for
teaching me a lot of what I know.
\end{comment}
\end{acknowledgements}

\begin{abstract}
Type your abstract here.
\end{abstract}
%
% The usage of the "foreword" and "preface" environments are similar
% the "abstract" and "acknowledgements". See FBE manual for the
% correct order of these pages in the thesis.
%
\begin{ozet}
T\"urk\c ce tez \"ozetini buraya yaz\i n\i z.
\end{ozet}

\tableofcontents
\listoffigures
\listoftables

\begin{symabbreviations}
% The title will be typeset as "LIST OF SYMBOLS/ABBREVIATIONS".
% If you prefer it as "LIST OF SYMBOLS" or "LIST OF ABBREVIATIONS"
% use the environments "symbols" or "abbreviations", respectively.
%
% Use a separate \sym command for each symbols definition.
% First Latin symbols in alphabetical order
\sym{$a_{ij}$}{Description of $a_{ij}$}
% Then Greek symbols in alphabetical order
\sym{$\alpha$}{Description of $\alpha$}
\sym{}{}       % Separating line between symbols and abbreviations
\sym{DA}{Description of abbreviation}
\end{symabbreviations}

\chapter{INTRODUCTION}
\pagenumbering{arabic}

\section{Bosons and fermions}
The concept of bosonic and fermionic particles is one of the most important concepts
in modern quantum physics. The behavior of large scale matter, from chemical
properties of elements to superconductivity and superfluidity can mostly be
understood by referring to the fermionic or bosonic nature of the quantum
mechanical particles involved in such phenomena. It is for this reason that
understanding the symmetry properties of these phenomena and, motivated by their
importance, trying to find other behavior that mimic them is very meaningful.

Furthermore, while bosonic behavior has a classical counterpart, the
concept of a fermionic particle is one that can only exist in the quantum
domain. This fact makes the study of such behavior even more important.
However, what could be more interesting is the study of other such
constructs that cannot have a classical counterpart. These constructs would
thus belong solely in the quantum domain and could help us understand phenomena
that are strictly quantum mechanical in nature.

There is a strong relation between the spin properties of a particle and
the particle being a boson or a fermion. In fact, it is a proven fact of
quantum physics that integer spin particles are bosons and half-integer spin
particles are fermions. This is most often referred to as the {\it spin-statistics
theorem} in quantum mechanics and is a very interesting fact since it implies
a relationship between two concepts that seem to be totally unrelated. This
strong relation between the bosonic/fermionic nature of a particle and its spin
makes the angular momentum algebra also very central in quantum physics.

Before we start investigating such matters, it would be apt to give an
overview of the state of bosons and fermions as it has been studied up to now.

When the harmonic oscillator is studied in a quantum mechanical manner \cite{dirac-book},
one arrives at the relation:
\beq
a \adag - \adag a = 1 \label{bosonic comm rel}
\eeq
to describe the system. The Hamiltonian of this system is given
by $\frac{\hbar \omega}{2} (a \adag + \adag a)$. The spectrum of this
Hamiltonian, which in turn gives us the allowable energy levels of the
quantum harmonic oscillator, can be obtained easily by introducing the
hermitian operator $N = \adag a$ which satisfies the following relations
with $a$ and $\adag$:
\bea
 [ N , \adag ] &=& \adag \\ [0pt]
 [ N , a ]     &=& a
\eea
where $[\;,\; ]$ denotes the usual commutator. By observing the fact
that the Hamiltonian is nothing but $\hbar \omega (N + \frac12)$, one
can see that one can get the states that correspond to the energy levels
as eigenvectors $\ket{n}$, of the operator $N$. The action of $\adag$
and $a$ on such an eigenvector $\ket{n}$ is found to be:
\bea
\adag \ket{n} &=& \sqrt{n + 1} \ket{n + 1} \\
a \ket{n} &=& \sqrt{n} \ket{n - 1}
\eea
Due to the fact that the operator $N$ is a positive hermitian operator,
its eigenvalues, namely $n$, cannot be negative. For a given positive
value of $n$, however, one can construct states with eigenvalues $n-1$,
$n-2$, $n-3$, and so on, by repeatedly applying the operator $a$ on the
original state. This sequence of eigenvalues will contain negative values
eventually for any given finite $n$ unless it is an integer. In that case,
the sequence will end at the eigenvalue $0$ since a further application of
the operator $a$ on that state will give us the zero vector of the Hilbert
space which is not a physically observable state and is thus a state
out of our domain.


As a result of this study one finds that the values of $n$, the
eigenvalues of the operator $N$, begin from $0$ and increase by
$1$ every time $\adag$ is applied on the relevant state and that the energy
levels of the quantum harmonic oscillator are given by $\hbar \omega (n + \frac12)$.
The operators $\adag$ and $a$ turn out to be  operators that create and destroy,
respectively, one quanta of energy and for this reason they are usually called
creation and annihilation operators.


Even though this operator algebra seems to only describe the quantum harmonic
oscillator, when one studies quantum field theory, this algebra comes up as the
algebra of the Fourier coefficients of the field operator describing a bosonic
particle. Each normal mode of a quantum field behaves as if it is an independent
harmonic oscillator and for that reason we have a separate set of creation and
annihilation operators for each of these modes. In that setting, the operators
$\adag_p$ and $a_p$, which now carry a continuous momentum index, are interpreted
as the operators that create and destroy, respectively, one bosonic particle
of such a field with momentum $p$.


For fermionic particles the story is a little bit more different.
In 1925, Pauli first proposed his {\it exclusion principle} \cite{pauli} to explain the behavior
of electrons in an atom. According to this principle, no two electrons could
exist in the same quantum state and it was for this reason that electrons could not
all occupy the lowest energy state in the atomic orbitals but instead had to line up
the energy levels in a well ordered manner. The implication of this principle to the
electron gas was first considered by Fermi and Dirac and it is for this reason that
particles that obey these statistics are called {\it fermions}. In 1926 Dirac noted
\cite{dirac} that the exclusion principle could also apply to other particles
by relating bosons and fermions to the symmetry of the many-particle
wavefunction. If the wavefunction changes sign upon exchange of two
particles then those particles would be fermions and they would be bosons
if the wavefunction did not change sign. This treatment effectively
implies the Pauli exclusion principle since if there were to be two fermionic
particles occupying the same quantum state, then upon their exchange
the wavefunction would change sign; on the other hand, we expect the wavefunction
to be identical to the original one before the exchange since nothing must have
changed about the quantum state of the system. For this reason the original
wavefunction can be nothing but zero if it is to be equal to its negative in
this manner. Thus, by contradiction, one can show that no two fermions can exist
in the same quantum state. It was only later, in 1928, that Jordan and Wigner
proposed \cite{jordan-wigner} that in order to treat fermions in
quantum field theory, their field operators had to anticommute so
that the wavefunction could be antisymmetric. They showed that a
consistent second-quantization of fermions implied anticommutation
relations on the field operators. This is turn implies that the Fourier coefficients
of the field operators that belong to a normal mode also obey anticommutation
relations instead of the commutation relations that the bosonic creation and
annihilation operators obey.


In this work, we would like to give an alternative derivation of this algebra
by only starting from the Pauli exclusion principle and assuming that fermionic
particles also have creation and annihilation operators just like the bosonic
particles. If this is the case then Pauli exclusion principle tells us that we
cannot create a second fermion in the same quantum state, i.e. that $(\adag)^2$,
and in turn $a^2$, should be $0$. This relation, however, is not compatible with
the commutation relation (\ref{bosonic comm rel}) and thus should be supplemented
with another kind of relation. If we define the operator $K$ as the anticommutator
of $a$ and $\adag$:
\beq
K \equiv a \adag + \adag a
\eeq
then we find that $K$ is a central element of the algebra, since:
\bea
\adag K &= \adag (a \adag + \adag a ) =& \adag a \adag \\
K \adag &= (a \adag + \adag a ) \adag =& \adag a \adag
\eea
which implies that $K$ commutes with $\adag$ and similarly with $a$, thus making
it a central element of the algebra. The central operator $K$ can be written as
a multiple of the identity $k\I1$ and if we rescale the operators $a$ and $\adag$
by $1/\sqrt{k}$, we arrive at the fermion anticommutator algebra:
\bea
a^2 &=& 0 \\
a \adag + \adag a &=& 1
\eea


This derivation of the fermion algebra also shows clearly that the physically more
important relation is the fact that the square of the annihilation operator is zero,
since the other relation follows from this fact. In literature, it is often the
case that only the anticommutation relation is presented as describing fermionic
particles, completely omitting the other, more important, relation. This is
usually falsely motivated by the assumption that the anticommutation relation
uniquely describes a fermionic system just as the commutation relation alone
describes a bosonic system. However,
without the first relation, the anticommutation relation alone describes a
completely different system which still has two states but is not equivalent to the
fermionic system.


A study of this fermion algebra, similar to the boson algebra, shows that, again,
a hermitian positive-definite number operator $N = \adag a$ can be defined and has
eigenvalues $0$ and $1$ that correspond to the states $\ket{0}$ and $\ket{1}$,
respectively. In harmony with our original assumption, the operator $\adag$ takes
the state $\ket{0}$ to the state $\ket{1}$ thus fulfilling the interpretation of
it as a creation operator. Similarly, the operator $a$ acts as an annihilation
operator of the algebra.


\section{Quantum groups and Hopf algebras}


The discovery of quantum groups has historically been motivated by the study
of quantization of non-linear completely integrable systems \cite{sklyanin}. The study of
such systems has shown that some non-linear completely integrable systems that
possess group symmetries, when quantized, acquire a different kind of symmetry;
a symmetry under quantum groups. By definition quantum groups are non-commutative and
non-cocommutative Hopf algebras and thus the physical importance of quantum groups and
Hopf algebras, in general, is very great since the aforementioned discovery.


In order to give an overview of the definition of a Hopf algebra and the motivations
behind these definition, we will start from the definition of an associative algebra
and starting form that definition give definitions of coalgebra, bialgebra and Hopf
algebra.


\subsection{Associative algebras}
In abstract mathematics, an associative algebra $A$ over a field $F$ is
defined to be a vector space over
$F$ with an $F$ bilinear multiplication $m: A \otimes A \rightarrow A$ (where the image of
$(x, y) \in A \otimes A$ which is $m(x,y)$ is usually written as $xy$) such that the associativity
law:
\beq
(xy)z = x(yz) \quad \text{for all $x,y,z \in A$}
\eeq
is satisfied. This associativity condition can also be written without reference to any of
the elements of the algebra $A$ by first considering that the condition is equivalent to:
\beq
m\circ(m(x, y), z) = m\circ(x, m(y,z)) \quad \text{for all $x,y,z \in A$},
\eeq
where $\circ$ denotes functional composition,
and then realizing that the {\it for all} condition can be expressed as:
\beq
m\circ(m(A \otimes A) \otimes A) = m\circ(A \otimes m(A \otimes A)) \quad .
\eeq
If we further define the identity operator on $A$ as $id(x) = x$
for all $x \in A$, then we can write the above form as:
\beq
m\circ(m \otimes id) (A \otimes A \otimes A) = m\circ(id \otimes m) (A \otimes A \otimes A) \quad ,
\eeq
where it is obvious that we can drop the $A \otimes A \otimes A$ terms from both sides
of the equation without losing the expressive power of the relation. Thus we end up with:
\beq
m\circ(m \otimes id) = m\circ(id \otimes m)
\eeq
for the definition of associativity of the product on an algebra $A$. This form of {\it element free
notation}, where appropriate, will be used in this work from this point on.


An associative algebra is called unital if the algebra $A$ contains an identity element $1$ such
that $1x = x1 = x$ for all $x \in A$. Such a unital algebra is also a ring and contains all the
elements of the field $F$ by identifying an element $k$ of the field with the algebra element $k1$.
This identification can be expressed as the existence of a unit map $\eta: F \rightarrow A$ which
has the property:
\beq
m \circ (id \otimes \eta) = s = m \circ (\eta \otimes id)
\eeq
where $s$ is the scalar multiplication $s: F \otimes A \rightarrow A$ such that $s(k, x) = kx$.
Since $F \otimes A$ is isomorphic to the original algebra $A$, the above relation is sometimes
written with $id$ in place of $s$ with scalar multiplication being implicitly understood.


As a result, we can see that the definition of a unital associative algebra is a
vector space over a field $F$ with two operations, $m: A \otimes A \rightarrow A$
and $\eta: F \rightarrow A$ defined such that the operations satisfy:
\begin{align}
m\circ(m \otimes id) &= m\circ(id \otimes m) \\
m \circ (id \otimes \eta) = &\; id = m \circ (\eta \otimes id)
\end{align}
These relations can also be written as the condition that the following diagrams
commute:
\pagebreak
\begin{figure}[!h]
  \[
  \xymatrix@=100pt{
    A \otimes A \otimes A \ar[r]^-*+{m \circ id} \ar[d]^-*+{id \circ m}& A \otimes A \ar[d]^-*+{m}\\
    A \otimes A \ar[r]^-*+{m} & A \\
  }
  \]
  \caption{Associativity in an algebra $A$}
  \label{assoc-algebra}
\end{figure}
\begin{figure}[!h]
  \[
  \xymatrix@=100pt{
    F \otimes A \cong A \cong A \otimes F
       \ar[r]^-*+{id \circ \eta}
       \ar[d]^-*+{\eta \circ id}
       \ar[dr]^-*+{id} & A \otimes A \ar[d]^-*+{m}\\
    A \otimes A \ar[r]^-*+{m} & A \\
  }
  \]
  \caption{Existence of unit in the algebra $A$}
  \label{unit-algebra}
\end{figure}




\subsection{Coalgebras}


The primary motivation for coalgebras stem from the study of the effect of the multiplication and unity
operators defined on an algebra on the dual of that algebra. The dual $A^*$ of an algebra $A$
is defined to be the set of all linear maps from $A$ to $F$. By this definition, the dual
of an algebra is a vector space provided that the addition and scalar multiplication is
defined as:
\begin{align}
(\phi + \psi)(x) & = \phi(x) + \psi(x) \\
(k\phi)(x) & = k \phi(x)
\end{align}
for all $\phi, \psi \in A^*$, $x \in A$ and $k \in F$. The dual does not
naturally carry any of the algebra structure of the original algebra and, in general,
is not itself an algebra. For this
reason, it is very natural to inquire about the effect of multiplication in $A$ on
the dual $A^*$. For this we consider:
\beq
\phi(xy) = \phi(m(x \otimes y))
\eeq
for $\phi \in A^*$ and $x,y \in A$. This form, in general, is not equal to $\phi(x) \phi(y)$
but it should be possible
to write it as a tensor product in terms of other elements of $A^*$ valued at $x \otimes y$. The
possibility of this can be shown if $A$ is finite-dimensional. In general, the multiplication
$m: A \otimes A \rightarrow A$ yields a linear map on the dual $\Delta: A^* \rightarrow
(A \otimes A)^*$. However, if $A$ is finite-dimensional, $(A \otimes A)^*$ is naturally isomorphic
to $(A^* \otimes A^*)$ and for that reason the map on the dual can be written as $\Delta: A^*
\rightarrow A^* \otimes A^*$. This map is called to coproduct. In terms of the coproduct, the
above relation becomes:
\beq
\phi(xy) = \phi(m(x \otimes y)) = \Delta(\phi)(x \otimes y)
\eeq
Similarly, the action of the unit map $\eta: F \rightarrow A$ yields a linear map on the
dual $\epsilon: A^* \rightarrow F$, which is called the counit. The action of the counit is
as follows:
\beq
\phi(k1) = \phi(\eta(k)) = \epsilon(\phi)k
\eeq
for $\phi \in A^*$ and $k \in F$. Thus, we see that the multiplication and unit maps on $A$
naturally define the coproduct and counit maps on the dual $A^*$. Furthermore, the associativity
and existence of unit conditions on the algebra $A$ implies certain conditions on the maps defined
on the dual $A^*$. The structure we have thus arrived at is called a coalgebra and the dual
of an algebra $A$ becomes a coalgebra.


Formally, the definition of a coalgebra $C$ is a vector space over a field $F$ together with
two linear maps:
\begin{itemize}
    \item Coproduct: $\Delta: C \rightarrow C \otimes C$
    \item Counit: $\epsilon: C \rightarrow F$
\end{itemize}
such that the conditions:
\begin{align}
(id \otimes \Delta) \circ \Delta & = (\Delta \otimes id) \circ \Delta \\
(id \otimes \epsilon) \circ \Delta = &\; id = (\epsilon \otimes id) \circ \Delta
\end{align}
are satisfied. The first of these conditions is called the coassociativity condition and is
equivalent to the fact that Figure \ref{coassoc-coalgebra} is commutative.
\begin{figure}[!h]
  \[
  \xymatrix@=100pt{
    C \ar[r]^-*+{\Delta} \ar[d]^-*+{\Delta}& C \otimes C \ar[d]^-*+{id \otimes \Delta}\\
    C \otimes C \ar[r]^-*+{\Delta \otimes id} & C \otimes C \otimes C \\
  }
  \]
  \caption{Coassociativity in a coalgebra $C$}
  \label{coassoc-coalgebra}
\end{figure}
Similarly, the second condition is called the existence of counit and is equivalent to the
commutativity of Figure \ref{counit-coalgebra}.
\begin{figure}[!h]
  \[
  \xymatrix@=100pt{
    C
       \ar[r]^-*+{\Delta}
       \ar[d]^-*+{\Delta}
       \ar[dr]^-*+{id} & C \otimes C \ar[d]^-*+{id \otimes \epsilon}\\
    C \otimes C \ar[r]^-*+{\epsilon \otimes id} & F \otimes C \cong C \cong C \otimes F \\
  }
  \]
  \caption{Existence of counit in the coalgebra $C$}
  \label{counit-coalgebra}
\end{figure}


\subsection{Bialgebras}


Formally, a bialgebra $B$  over a field $F$ is both a unital associative algebra
and a coalgebra over $F$ such that the coproduct and counit maps are both algebra homomorphisms.
In this respect, the coalgebra structure should be compatible with the algebra structure
of the bialgebra. We will also show that the same statement can be expressed from the
opposite point of view, ie. that the algebra structure of the bialgebra should be
compatible with the coalgebra structure. For this reason, the product and the unit maps
should, equivalently, be algebra homomorphisms.


Before analyzing the implications of the compatibility condition, we should define
$m_{B \otimes B}$ which is the product defined on $B \otimes B$ using the product
defined on $B$. The map $m_{B \otimes B}: (B \otimes B) \otimes (B \otimes B) \rightarrow B \otimes B$
is a formalization of the product rule $(a\otimes b)(c \otimes d) = (ac) \otimes (bd)$ and for this
reason the action of this map is defined by:
\beq
m_{B \otimes B}((a\otimes b) \otimes (c \otimes d)) = m(a \otimes c) \otimes m(b \otimes d) \quad .
\eeq
One should notice that the definition of this product involves a permutation of the order of the terms $b$ and $c$.
Using this fact and defining the permutation operator $\tau: B \otimes B \rightarrow B \otimes B$ by:
\beq
\tau(a \otimes b) = b \otimes a \quad ,
\eeq
we can rewrite the action of the product map on $B \otimes B$ as:
\beq
\begin{split}
m_{B \otimes B}((a \otimes b) \otimes (c \otimes d))
 & = m(a \otimes c) \otimes m(b \otimes d) \\
 & = (m \otimes m) (a \otimes c \otimes b \otimes d) \\
 & = (m \otimes m) \circ (id \otimes \tau \otimes id) (a \otimes b \otimes c \otimes d) \quad .
\end{split}
\eeq
As a result, we find that $m_{B \otimes B}$ is defined in terms of the product map on $B$ as:
\beq
m_{B \otimes B} = (m \otimes m) \circ (id \otimes \tau \otimes id) \quad .
\eeq


The statement that the coproduct map is a algebra homomorphism implies that:
\begin{align}
\Delta(ab) = \Delta(m(a \otimes b)) & = m_{B \otimes B}(\Delta(a) \otimes \Delta(b)) \\
\Delta(1) & = 1 \otimes 1
\end{align}
These two equations say that the action of the coproduct map respects both the product and the
unit of the algebra structure in the bialgebra. Similarly, the condition that the counit
is an algebra homomorphism implies:
\begin{align}
\epsilon(ab) = \epsilon(m(a \otimes b)) &= m(\epsilon(a) \otimes \epsilon(b)) = \epsilon(a) \epsilon(b) \\
\epsilon(1) &= 1
\end{align}


The content of these relations that define a bialgebra
can again be expressed by the commutative diagrams in Figures
\ref{coproduct-comp-product-bialgebra} and \ref{coproduct-comp-unit-bialgebra} for the homomorphism
conditions on the coproduct and Figures \ref{counit-comp-product-bialgebra} and
\ref{counit-comp-unit-bialgebra} for the homomorphism
conditions on the counit.
\begin{figure}[!h]
  \[
  \xymatrix@=100pt{
    B \otimes B \ar[r]^-*+{\Delta \otimes \Delta} \ar[d]^-*+{m}& (B \otimes B) \otimes (B \otimes B) \ar[d]^-*+{m_{B \otimes B}}\\
    B \ar[r]^-*+{\Delta} & B \otimes B\\
  }
  \]
  \caption{Compatibility of the coproduct with the product on the bialgebra $B$}
  \label{coproduct-comp-product-bialgebra}
\end{figure}


\begin{figure}[!h]
  \[
  \xymatrix@=60pt{
    F \cong F \otimes F \ar[dr]^-*+{\eta} \ar[rr]^-*+{\eta \otimes \eta} &                       & B \otimes B\\
                                                                         & B \ar[ur]^-*+{\Delta} &            \\
  }
  \]
  \caption{Compatibility of the coproduct with the unit on the bialgebra $B$}
  \label{coproduct-comp-unit-bialgebra}
\end{figure}


\begin{figure}[!h]
  \[
  \xymatrix@=60pt{
    B \otimes B \ar[dr]^-*+{m} \ar[rr]^-*+{\epsilon \otimes \epsilon} &                         & F \otimes F \cong F \\
                                                                      & B \ar[ur]^-*+{\epsilon} &                     \\
  }
  \]
  \caption{Compatibility of the counit with the product on the bialgebra $B$}
  \label{counit-comp-product-bialgebra}
\end{figure}


\begin{figure}[!h]
  \[
  \xymatrix@=60pt{
    F \ar[dr]^-*+{\eta} \ar[rr]^-*+{id} &                         & F  \\
                                        & B \ar[ur]^-*+{\epsilon} & \\
  }
  \]
  \caption{Compatibility of the counit with the unit on the bialgebra $B$}
  \label{counit-comp-unit-bialgebra}
\end{figure}
\pagebreak
One can see from these commutative diagrams, that the diagrams are completely symmetric with
respect to the coalgebra and algebra maps. In other words, one can see that these diagrams
can also be read as the coalgebra homomorphism conditions of the product and the unit maps
of the algebra structure of the bialgebra $B$. The only diagram that does not explicitly exhibit
this symmetry is Figure \ref{coproduct-comp-product-bialgebra}. This diagram, however, can be
written in an explicitly symmetric way by using the definition of $m_{B\otimes B}$ to produce the
commutative diagram shown in Figure \ref{coproduct-comp-product-bialgebra-symm}. This way
the content of all the diagrams can be read both as the compatibility of the coalgebra maps
on the algebra structure and the compatibility of the algebra maps on the coalgebra structure of
the bialgebra $B$.


\begin{figure}[!h]
  \[
  \xymatrix@=70pt{
                                                                   & B  \ar[dr]^-*+{\Delta} & \\
    B \otimes B \ar[ur]^-*+{m} \ar[d]^-*+{\Delta \otimes \Delta}            &  & B \otimes B \\
    B \otimes B \otimes B \otimes B \ar[rr]^-*+{id \otimes \tau \otimes id} &  & B \otimes B \otimes B \otimes B \ar[u]^-*+{m \otimes m}\\
  }
  \]
  \caption{Compatibility of the coproduct with the product on the bialgebra $B$}
  \label{coproduct-comp-product-bialgebra-symm}
\end{figure}




\subsection{Hopf algebras}


A Hopf algebra $H$ is basically a bialgebra, ie. both a unital associative algebra and a coalgebra, with an additional
structure called the coinverse (or the antipode) which is a linear map $S: H \rightarrow H$
such that the diagram in Figure \ref{antipode-hopf} is commutative.


\begin{figure}[!h]
  \[
  \xymatrix@=50pt{
    H \otimes H \ar[rr]^-*+{S \otimes id}                        &                    & H \otimes H \ar[d]^-*+{m}\\
    H \ar[u]^-*+{\Delta} \ar[d]^-*+{\Delta} \ar[r]^-*+{\epsilon} & F \ar[r]^-*+{\eta} & H \\
    H \otimes H \ar[rr]^-*+{id \otimes S}                        &                    & H \otimes H \ar[u]^-*+{m}\\
  }
  \]
  \caption{Definition of coinverse on the Hopf algebra $H$}
  \label{antipode-hopf}
\end{figure}

\pagebreak
In order to write the concept of a coinverse more
explicitly, we will introduce Sweedler's \cite{sweedler} notation which can be considered to
be the analogue of Einstein summation convention for coproducts. Given an element $c$ of a coalgebra,
there exists elements $c_{(1)}^i$ and $c_{(2)}^i$ in the coalgebra such that:
\beq
\Delta(c) = \sum_i c_{(1)}^i \otimes c_{(2)}^i \quad .
\eeq
Using Sweedler's notation, this can be abbreviated to:
\beq
\Delta(c) = \sum_c c_{(1)} \otimes c_{(2)}
\eeq
and in the sumless version of Sweedler's notation, it further becomes:
\beq
\Delta(c) = c_{(1)} \otimes c_{(2)}
\eeq
Thus the coinverse map $S$ can also be expressed as:
\beq
S(c_{(1)})c_{(2)} = m(S(c_{(1)}) \otimes c_{(2)}) = \epsilon(c) 1 = m(c_{(1)} \otimes S(c_{(2)})) = c_{(1)}S(c_{(2)})
\eeq

The notion commutativity in a Hopf algebra is defined by the commutativity of the product
map of the algebra structure. An algebra is commutative if and only if the product map
satisfies the relation:
\beq
m = m \circ \tau
\eeq
so that the order of multiplying terms in the product does not matter. In terms of
elements of the algebra this relation becomes:
\beq
m(a \otimes b) = m(b \otimes a)
\eeq
for all elements $a,b$ of the algebra. Similarly, the
notion of cocommutativity in a Hopf algebra is defined by the cocommutativity of the
coproduct map of the coalgebra structure. A coalgebra is cocommutative if and only if
the coproduct map satisfies the relation:
\beq
\Delta = \tau \circ \Delta
\eeq
so that the order of terms in the outcome of the coproduct does not matter. In
terms of elements of the coalgebra and using sumless Sweedler's notation, this
relation implies:
\beq
\Delta(c) = c_{(1)} \otimes c_{(2)} = c_{(2)} \otimes c_{(1)}
\eeq
for all elements $c$ of the coalgebra.

There are various examples of Hopf algebras. Out of these the most important examples
are the group algebras and universal enveloping algebras of Lie algebras. Given a
group $G$, the group algebra $FG$ is a unital associative algebra over the field $F$.
It becomes a Hopf algebra, if we define the coproduct, counit and coinverse maps by:
\begin{align}
\Delta(g) & = g \otimes g \\
\epsilon(g) & = 1 \\
S(g) & = g^{-1}
\end{align}
for all $g \in G$. In this instance the resulting Hopf algebra is always
cocommutative  (since $g \otimes g = g \otimes g$) and is
commutative depending on the original group $G$ being abelian or not. If the
underlying group $G$ is abelian, the resulting Hopf algebra is both cocommutative
and commutative. Otherwise, it is cocommutative but noncommutative.

Similarly, given a Lie algebra $g$ over a field $F$, its universal enveloping algebra
$U(g)$ is a unital associative algebra. This algebra $U(g)$ becomes a Hopf algebra
if we define the coproduct, counit and the coinverse maps as:
\begin{align}
\Delta(x) & = 1 \otimes x + x \otimes 1 \\
\epsilon(x) & = 0 \\
S(x) & = -x
\end{align}
for all $x \in U(g)$. Notice that the coproduct rule is not only compatible with the
product on the universal enveloping algebra but it is also compatible with the
antisymmetric product defined on the Lie algebra itself. This Hopf algebra is
cocommutative but noncommutative.

Quantum groups are, loosely, defined as Hopf algebras that are neither commutative nor
cocommutative. As such, they are important in non-commutative geometry. The reason for
this stems from the observation that in order to study geometry on a manifold $M$,
it is possible to work with the algebra of functions $A = C(M)$ on $M$ which is a Hopf
algebra. Thus, one can continue studying Hopf algebras, including noncommutative
and noncocommutative ones, and do geometry with them even though the underlying manifold
does not exist anymore in a conventional sense. They are called quantum groups because of
a similar reasoning stating that a standard algebraic group is well described by the
Hopf algebra of regular functions on the algebraic group and that a deformed version
of the Hopf algebra should, in some sense, describe a deformed, quantized version of the
algebraic group. In essence, identifying these quantized algebraic groups with their
Hopf algebras one can study them in full generality and make a theory of these quantum
groups.

Since it is essentially the noncocommutativity of a Hopf algebra that makes it interesting,
it is natural for there to be a mathematical property to quantify the amount of
its noncocommutativity. This is analogous to the definition of the commutator to describe
the amount of noncommutativity of an associative algebra. Thus, a quasitriangular Hopf
algebra is defined as a Hopf algebra $H$, where there is an invertible element $R$
in $H \otimes H$, such that it satisfies:
\begin{align}
\tau \circ \Delta & = R \Delta R^{-1} \\
(\Delta \otimes id)(R) & = R^{13} R^{13} \\
(id \otimes \Delta)(R) & = R^{13} R^{12}
\end{align}
where if $R = a_{i} \otimes\; b_{i}$ then $R^{12}$, $R^{13}$ and $R^{23}$ are defined as:
\begin{align}
R^{12} & = a_{i} \otimes\; b_{i} \otimes\; 1 \\
R^{13} & = a_{i} \otimes\; 1 \otimes\; b_{i} \\
\intertext{and}
R^{23} & = 1 \otimes\; a_{i} \otimes\; b_{i} \quad .
\end{align}
As can be seen, in a quasitriangular Hopf algebra the coproduct is almost cocommutative
up to a conjugation by the invertible element $R$. Moreover, if one works with
the equations given above, one can arrive at a matrix equation for $R$
given by the quantum Yang-Baxter equation:
\beq
R^{12} R^{13} R^{23} = R^{23} R^{13} R^{12}
\eeq
which plays a fundamental role in the theory of completely integrable systems \cite{qybe}.
If one starts from this matrix equation for $R$, one can start categorizing the solutions
to the matrix equation and thus categorize quasitriangular Hopf algebras. Equivalently,
every matrix representation of a quasitriangular Hopf algebra, implies a matrix representation
of $R$ and as such gives one a solution to the quantum Yang-Baxter equation. Thus, from
a single Hopf algebra, it is possible to extract many solutions to this equation by using
different matrix representations. This is the reason why the element $R$ of $H \otimes H$ is
sometimes called the universal $R$-matrix. It is mostly for these reasons that commonly
studied physical Hopf algebras and quantum groups are generally quasitriangular.

\section{Quantum matrix groups}

A quantum matrix group is defined by a set of $n$ x $n$ matrices $M$:
\beq
M =
\left(
\begin{array}{cccc}
a_{11} & a_{12} & \ldots & a_{1n}  \\
a_{21} & \ddots &        &  \vdots \\
\vdots &        & \ddots &  \vdots \\
a_{n1} & \ldots & \ldots & a_{nn}
\end{array}
\right)
\eeq
such that every element of the matrix belong to a Hopf algebra $H$.
The matrix group defined in this way naturally becomes a Hopf algebra with the
coproduct, counit and coinverse of the matrix algebra being defined as:
\bea
\triangle(M) &=& M \dot{\otimes} M \label{qmg-coproduct}\\
\epsilon(M) &=& \I1_n \label{qmg-counit} \\
S(M) &=& M^{-1} \label{qmg-coinverse}
\eea
where $\dot{\otimes}$ stands for the operation
where when the matrix multiplication is performed the matrix
elements are multiplied using the tensor product is instead of the
normal product and $\I1_n$ stands for the $n$ x $n$ unit matrix.
The relations above imply the definitions of the
coproduct, counit and coinverse of the matrix elements:
\bea
\triangle(a_{ij}) &=& \sum_k a_{ik} \otimes a_{kj} \\
\epsilon(a_{ij}) &=& \delta_{ij} \\
\sum_j S(a_{ij}) a_{jk} &=& \delta_{ij} = \sum_j a_{ij} S(a_{jk})
\eea

One of the most important examples of quantum matrix groups is the quantum
group $GL_q(n)$. This quantum group is a quantum subgroup of the bialgebra
$M_q(n)$. An element $T$ of $M_q(n)$ has matrix entries $t_{ij}$ that satisfy:
\begin{align}
t_{ik} t_{il} & = q t_{il} t_{ik} & &\text{for $k < l$} \\
t_{ik} t_{jk} & = q t_{jk} t_{ik} & &\text{for $i < j$} \\
t_{il} t_{jk} & = t_{jk} t_{il} & &\text{for $i  < j$, $k < l$} \\
t_{ik} t_{jl} - t_{jl} t_{ik} & = (q - q^{-1}) t_{il} t_{jk} & &\text{for $i  < j$, $k < l$}
\end{align}
for some $q \in \IC$. One can immediately see that $M_q(n)$ is a
bialgebra with the definitions of coproduct and counit given in
equations \eqref{qmg-coproduct} and \eqref{qmg-counit}. In order to define
the coinverse, one needs to define the inverse of such a matrix and for
this one should define the quantum analogues of the
determinant and the adjoint. For an element $T$ of $M_q(n)$ one defines
the {\it quantum determinant} \cite{krob-leclerc} as:
\beq \label{qdet}
det_q(T) = \sum_{\sigma \in S_n} (-q)^{i(\sigma)} t_{1\sigma(1)} \cdots t_{n\sigma(n)}
\eeq
where $S_n$ is the symmetric group on ${1, \cdots, n}$ and $i(\sigma)$ is the number
of adjacent transpositions in the permutation $\sigma$. Similar to normal matrices,
one can also define the
{\it quantum adjoint} matrix $adj(T) = (a_{ij})$ such that:
\beq \label{qadjoint}
a_{ij} = (-q)^{j - i} det_q(T^{ij})
\eeq
where $T^{ij}$ stands for the $(n-1)$ x $(n-1)$ matrix obtained from $T$ by
deleting the $i$th row and the $j$th column. Just as in classical matrices, one
has:
\beq
T \cdot adj(T)^T = adj(T)^T \cdot T = det_q(T) \I1_n
\eeq
which yields the inverse, and in turn the coinverse, of such a quantum matrix as:
\beq \label{qinverse}
S(T) = T^{-1} = det_q^{-1}(T) adj(T)^T
\eeq
This coinverse is obviously only defined when $det_q(T)$ is invertible for all $T$ in the
quantum matrix group. The quantum subgroup of $M_q(n)$ that satisfies this condition
is called $GL_q(n)$, the quantum general linear group of dimension $n$ and this structure
is a Hopf algebra since the coinverse is now defined. In analogy with
the classical matrix groups, one can further restrict the determinant of such matrices
to be equal to $1$ and obtain the quantum subgroup $SL_q(n)$, the special linear quantum
group of dimension $n$.

\section{Quantum group invariance of an algebra}
A (left)module over the ring $R$ consists of an abelian group $M$ and the scalar
multiplication operation $s: R \otimes M \rightarrow M$, the action of which is
usually written as $s(r, x) = rx$ for some $r$ in $R$ and some $x$ in $M$, and such that:
\begin{align}
r(x+y) &= rx+ry \\
(r+s)x &= rx+sx \\
(rs)x &= r(sx) \\
1x &= x
\end{align}
for all $r,s$ in $R$ and all $x, y$ in $M$. Notice that this definition of a module is the same
as the definition of a vector space except a module is defined over a ring instead of a field.
Thus, every vector space is also a module and a module over $K$ is the same thing as a vector
space over $K$ if $K$ is a field.

The reason why the module is defined here is that they gives a representation of a ring or any
structure that extends the ring structure. On order to see this, one can consider the scalar
multiplication as the action of the ring $R$ on $M$ by sending the element $x$ to $rx$. This action
will be a group endomorphism due to the definition of a module. Thus, if one identifies an element
$r$ in $R$ by its action, then one has defined a map from $R$ to $End(M)$ which respects the
ring structure. Such a map $R \rightarrow End(M)$ is called a representation of the ring $R$ over
the abelian group $M$. If we consider the representations of a vector space, then these representations
also form a vector space such that these representations can be multiplied by the elements
of the underlying field and can be added to generate new representations. Since a Hopf algebra
is an associative algebra, which itself is a vector space, it also has such
representations. However, the interesting fact for Hopf algebras (more specifically bialgebras)
is that if $U$ and $V$ are two representations of the Hopf algebra then $U \otimes V$ is also
a representation for the Hopf algebra due to the nature of the coproduct. The representation
of $A$ in $H$ on $W = U \otimes V$ is given by $\Delta(A) = A_{(1)} \otimes A_{(2)}$ such that
$A_{(1)}$ gives the representation on $U$ and $A_{(2)}$ gives the representation on $V$.
Thus, one can form direct products of representations to form new representations for a Hopf
algebra.

Starting from this interesting fact for Hopf algebras one can arrive at an even more
interesting result. If one were to consider an associative algebra $A$ as a vector representation
of an algebra, then the original algebra structure of $A$ would not be preserved since the
product of two representations don't even form another representation. However, if we consider $A$
as representation for a Hopf algebra $H$ and if the product $m_A$ on $A$ respects the representation map,
then the linear map $\rho: H \otimes A \rightarrow A$ is an algebra representation of the Hopf algebra.
Thus Hopf algebras accept representations which can also form algebras.

Finally, if $A$ is a representation of a Hopf algebra and $X$ is an element of $A$ such that:
\beq
c[X] = \epsilon(c) X
\eeq
for all $c$ in $H$, then $X$ is said to invariant under
$H$. The subset of all such invariant elements of a representation $A$ forms a subrepresentation of $H$.
Thus, given an algebra $A$ which is a representation of $H$, if the invariant elements of $A$ form the
whole of the algebra then $A$ is said to be invariant under the action of the Hopf algebra $H$.
\section{Summary}

This work is divided into four chapters. In the first chapter, an introduction was given to
the basic concepts of boson and fermions and basic mathematical structures related to the
succeeding chapters were introduced.

In the second chapter, the anticommuting spin algebra (ACSA) will be introduced. In that section
it will be shown that the invariance group of ACSA is $SO_{q = -1}(3)$ and that the representations
of ACSA show great similarity to the representations of its sister spin algebra. Finally, the exact
relationship between ACSA and the spin algebra will be examined and a braided Hopf algebra structure
for ACSA will be introduced.

The third chapter deals entirely with the inhomogeneous invariance (quantum) groups of the
boson and fermion algebras. The bosonic inhomogeneous symplectic quantum group and the
fermionic inhomogeneous orthogonal group will be introduced to describe these invariance
conditions. In the subsections of this chapter, the sub(quantum)groups and contractions of
these new quantum groups will be studied. Finally, the fermionic inhomogeneous orthogonal
group will defined in odd number of dimensions and its sub(quantum)groups will be studied.

The fourth and final chapter is reserved for concluding remarks about the body of work that
has been introduced in this study.


\chapter{THE ANTICOMMUTING SPIN ALGEBRA}

%% Put the following text in the intro somewhere
\begin{comment}
The algebra of observables in quantum theory plays a fundamental
role. When classical systems are quantized, their classical
symmetry algebra acting on a set of physical observables, in
simplest examples, remains the same. For some completely
integrable non-linear models, consistent quantization requires
that the classical symmetry group be replaced by a quantum group
\cite{frt,drinfeld,woronowicz,manin} via a deformation parameter
$q = 1 + O(\hbar)$. In recent years quantum groups involving
fermions have received widespread attention. These include
deformed fermion algebras \cite{jx,xh,sm,chung}, spin chains
\cite{nt,gppr,bnnpsw} and Fermi gases \cite{ubriaco}. At the same
time, some quantum systems, most notably fermionic quantum systems
do not have any classical analogues. Nevertheless, fermions are
perhaps the most important sector of quantum phenomena. Motivated
by these considerations, we define a fermionic version of the
angular momentum algebra by the relations
\end{comment}
%%


\section{Defining relations}

\bea
\anti{J_1}{J_2} & = & J_3 \label{eqn:defrel1} \\
\anti{J_2}{J_3} & = & J_1 \label{eqn:defrel2} \\
\anti{J_3}{J_1} & = & J_2 \label{eqn:defrel3}
\eea
where $J_1$, $J_2$, $J_3$ are hermitian generators of the algebra. We will name this algebra ACSA, the anticommutator spin algebra. In these expressions the curly bracket denotes the anticommutator \beq
\anti{A}{B} \equiv AB + BA \eeq so (\ref{eqn:defrel1}-\ref{eqn:defrel3}) should be taken as the
definition of an associative algebra. This proposed algebra does
not fall into the category of superalgebras in the sense of
Berezin-Kac axioms. In particular, the algebra is consistent
without grading and there are no (graded) Jacobi relations. As it
is defined this algebra falls into the category of a
(non-exceptional) Jordan algebra where the Jordan product is
defined by: \beq A \circ B \equiv \frac12 (AB + BA) \quad . \eeq A
formal Jordan algebra, in addition to a commutative Jordan
product, also satisfies $A^2\circ(B\circ A) = (A^2\circ B)\circ
A$. When the Jordan product is given in terms of an anticommutator
this relation is automatically satisfied. Just as a Lie algebra
where the Lie bracket as defined by the commutator leads to an
enveloping associative algebra, a Jordan algebra defined in terms
of the above product leads to an enveloping associative algebra
which we consider as an algebra of observables.

The physical properties of this system turn out to be similar to
those of the angular momentum algebra yet exhibit remarkable
differences. Since the angular momentum algebra is used to
describe various internal symmetries, ACSA could be relevant in
describing those symmetries.

In section 2 we will show that ACSA is invariant under the action
of the quantum group $SO_q(3)$ with $q=-1$. Here, $SO_q(3)$ is
defined as the quantum subgroup of $SU_q(3)$ where each of the
(non-commuting) matrix elements of the $3$x$3$ matrix is
hermitian. We note that this defines a quantum group only for
$q=\pm1$. For $q=1$ one has the real orthogonal group $SO(3)$.

In section 3, we will construct all representations of ACSA and
show that the representations can be labelled by a quantum number
$j$ corresponding to the eigenvalue of $J_3$ whose absolute value
is maximum. For integer $j$, spectrum of $J_3$ is given by $j,
j-1, \ldots, -j$ whereas for half-integer $j$ there are two
representations. These two representations are such that for $j =
2k\pm\frac12$ spectrum of $J_3$ is respectively given by $j, j-2,
\ldots, \pm\frac12$ and $-j, j+2, \ldots, \mp\frac12$. Section 4
is reserved for conclusions and discussion.

\section{The invariance quantum group $SO_{q = -1}(3)$}

In order to find the invariance quantum group of this algebra, we
transform the generators $J_i$ to $J'_i$ by: \beq \label{trans}
J'_i = \sum_j\alpha_{ij} J_j \quad . \eeq The matrix elements
$\alpha_{ij}$ are hermitian since $J_i$'s are hermitian and they
commute with $J_i$'s but do not commute with each other. For the
transformed operators to obey the original relations, there should
exist some conditions on the $\alpha$'s which define the
invariance quantum group of the algebra. It is very convenient at
this moment to switch to an index notation that encompasses all
three defining relations of the algebra in one index equation. For
the angular momentum algebra this is possible by defining the
totally anti-symmetric rank 3 pseudo-tensor $\epsilon_{ijk}$. A
similar object for ACSA which we will call the fermionic
Levi-Civita tensor, $u_{ijk}$, is defined as:
\beq
u_{ijk} =
\left\{
\begin{array}{cl}
1, & i \neq j \neq k, \\
0, & \mbox{otherwise.}
\end{array}
\right.
\eeq
Thus the defining relations (\ref{eqn:defrel1}-\ref{eqn:defrel3}) become:
\beq
\anti{J_i}{J_j} = \sum_k  u_{ijk}\,J_k + 2\delta_{ij}\,J_i^2
\eeq
The second term on the right is needed since when $i = j$ the left-hand side becomes $2J_i^2$. When we apply the transformation (\ref{trans})
on this relation we get:
\beq
\anti{J'_k}{J'_m} = \sum_p
u_{kmp}J'_p = \sum_{p,\; j}
u_{kmp}\,\alpha_{pj}\,J_j\quad\quad\mbox{for}\quad k\neq m.
\eeq
However, substituting the transformation equations into the
left-hand side, we have:
\beq
\anti{J'_k}{J'_m} =
\sum_{i,\;j}\left([\alpha_{ki}, \alpha_{mj}]J_iJ_j +
\alpha_{mj}\,\alpha_{ki}\,u_{ijn}\,J_n +
2\alpha_{mj}\,\alpha_{kj}\,J_j^2\right)
\eeq
These two equations yield the following relations among $\alpha_{ij}$ when $k\neq m$:
\bea
\anti{\alpha_{mj}}{\alpha_{kj}} & = & 0 \label{invrel1} \\
\left[\alpha_{ki}, \alpha_{mj}\right]& = & 0 \quad\quad \mbox{for}\quad i \neq j \label{invrel2} \\
\sum_{i\; j}\alpha_{mj}\,\alpha_{ki}\,u_{ijn}  & = & \sum_p
u_{mkp}\,\alpha_{pn} \label{invrel3}
\eea

Now we will define the quantum group $SO_q(3)$ and show that the
relations above correspond to the case $q=-1$. The quantum group
$SO_q(3)$ can be defined as the quantum subgroup of $SL_q(3, C)$
where an element is given by:
\beq
A =
\left(
\begin{array}{ccc}
\alpha_{11} & \alpha_{12} & \alpha_{13} \\
\alpha_{21} & \alpha_{22} & \alpha_{23} \\
\alpha_{31} & \alpha_{32} & \alpha_{33}
\end{array}
\right)
\eeq
where
\beq
\alpha_{ij}^* = \alpha_{ij}
\eeq
\beq
A^T= A^{-1}
\eeq
and
\beq
\label{gl2}
\left(
\begin{array}{cc}
\alpha_{mj} & \alpha_{mi} \\
\alpha_{kj} & \alpha_{ki}
\end{array}
\right) \in GL_{q}(2)\quad\mbox{for}\quad k \neq m, i \neq j.
\eeq

The quantum group $SO_q(3)$ is equivalent to the quantum group
\break\mbox{$SL_q(3, R) \cap SU_q(3)$}. However one can show for
$SL_q(3)$ that $q= e^{i\beta}$ for some $\beta \in R$ and
similarly for $SU_q(3)$ that $q \in R$. Thus one finds that $q =
\pm 1$ for $SO_q(3)$. When $q = 1$ the quantum group becomes the
usual $SO(3)$ group; the interesting case is when $q = -1$ which,
as we will show, is the invariance quantum group of ACSA.

Equations (\ref{invrel1}) and (\ref{invrel2}) are easily shown to
be satisfied by the matrix $A \in SO_{q=-1}(3)$ by recognizing
that the quantities involved belong to a submatrix that is an
element of $GL_{q=-1}(2)$, as in equation (\ref{gl2}). For a
general matrix $M \in GL_q(2)$ where:
\[
M = \left(
\begin{array}{cc}
a & b \\
c & d
\end{array}
\right)
\]
we have the relations:
\bea
a c & = & q c a \label{ac} \\
a d - q b c & = & d a - q^{-1} c b \label{det} \\
b c & = & c b \label{bc}
\eea
The relation (\ref{ac}) implies that
$\alpha_{mj}\;\alpha_{kj} = (-1) \alpha_{kj}\;\alpha_{mj}$, which
proves equation (\ref{invrel1}) is satisfied, and the relations
(\ref{det}) and (\ref{bc}) show $ a d = d a$ for $q=-1$ which
implies $\alpha_{ki}\;\alpha_{mj} = \alpha_{mj}\;\alpha_{ki}$ thus
proving that equation (\ref{invrel2}) is satisfied by the elements
of an $SO_{q=-1}(3)$ matrix.

It is a little harder to show that equation (\ref{invrel3}) is
satisfied by elements of $SO_{q=-1}(3)$ matrices. However, if one
writes out the indices of the equation, one finds that equation
(\ref{invrel3}) implies that each matrix element is equal to the
$GL_{q=-1}(2)$-determinant of its minor. This fact is indeed
satisfied by $SO_{q=-1}(3)$ matrices since $\mathrm{det} A = 1$
and ${A^{-1}}^T = A$, one can show that $A = Co(A)$ which itself
means that every element is equal to the determinant of its minor.
Note that since $q = -1$, the cofactor of an element is always
equal to the minor without any alternation of signs. This type of
determinant with no alternation of signs is also called a
permanent.

Thus, we have found that the invariance quantum group of ACSA is
the quantum group $SO_q(3)$ with $q=-1$. Strictly speaking, ACSA
is a module of the $q$-deformed $SO(3)$ quantum algebra with $q =
-1$. It is very interesting to note that the invariance group of
the angular momentum algebra is also $SO_q(3)$ but with $q=1$.

\section{Representations}

The Anticommutator Spin Algebra is defined by the relations
(\ref{eqn:defrel1}-\ref{eqn:defrel3}). In order to find the
representations of this algebra we define the operators:
\bea
J_+ & = & J_1 + J_2 \\
J_- & = & J_1 - J_2 \\
J^2 & = & J_1^2 + J_2^2 + J_3^2
\eea
which obey the following relations:
\bea
\anti{J_+}{J_3} & = & J_3 \\
\anti{J_-}{J_3} & = & -J_3 \\
J_+^2 & = & J^2 - J_3^2 + J_3 \label{jp^2}\\
J_-^2 & = & J^2 - J_3^2 - J_3 \label{jm^2}
\eea
Furthermore, it can easily be shown that $J^2$ is central in the algebra, i.e. that it commutes with all the elements of the algebra, by first observing that:
\bea
J^2_j \; J_i &=& J_j (J_k - J_i \; J_j) \nonumber \\
          &=& J_j J_k - (J_k - J_i \; J_j) J_j \nonumber \\
          &=& (J_j \; J_k - J_k \; J_j) + J_i \; J^2_j \nonumber \\
          &=& (2J_j \; J_k - J_i) + J_i \; J^2_j \quad \mbox{for}\quad  i \neq j \neq k \neq i.
\eea
Using this relation and the fact that $J^2 = \sum_j J_j$, we can see:
\bea
J^2 \; J_i &=& \sum_j J^2_j \; J_i \nonumber \\
        &=& J^3_i + \sum_{j \neq j} J^2_j \; J_i \nonumber \\
        &=& J^3_i + \sum_{j \neq i} (2J_j \; J_k - J_i + J_i \; J^2_j)
\eea
However, in the final form of this expression the sum only contains
two terms where the two indices $j$ and $k$ are symmetric. Thus the
whole expression can be written as:
\bea
J^2 \; J_i &=& J^3_i + \sum_{j \neq i} (2J_j \; J_k - J_i + J_i \; J^2_j) \nonumber \\
        &=& J^3_i - 2J_i + 2(J_j \; J_k + J_k \; J_j) + J_i \; J^2_j + J_i \; J^2_k \nonumber \\
        &=& J_i \; J^2 - 2J_i + 2J_i \nonumber \\
        &=& J_i \; J^2 \quad \mbox{for}\quad  i \neq j \neq k \neq i,
\eea
and therefore showing that $J^2$ commutes with all the elements of the algebra.

For this reason, we can label the states in our representation with the eigenvalues of $J^2$ and $J_3$:
\bea
J^2 \ket{\lambda, \mu} & = & \lambda \ket{\lambda, \mu} \\
J_3 \ket{\lambda, \mu} & = & \mu \ket{\lambda, \mu}
\eea
The action of $J_+$ and $J_-$ on the states such defined is easily
shown to be:
\bea
J_+ \ket{\lambda, \mu} & = & f(\lambda, \mu) \ket{\lambda,- \mu + 1} \label{jplus}\\
J_- \ket{\lambda, \mu} & = & g(\lambda, \mu) \ket{\lambda,- \mu -
1} \label{jminus}
\eea
It is enough to look at the norm of the states $J_+ \ket{\lambda, \mu}$ and $J_- \ket{\lambda, \mu}$ to find $f(\lambda, \mu)$ and $g(\lambda, \mu)$. Thus:
\bea
\bra{\lambda, \mu} J_+^2 \ket{\lambda, \mu} & = & |f(\lambda, \mu)|^2 \\
\bra{\lambda, \mu} J^2 - J_3^2 + J_3 \ket{\lambda, \mu} & = & |f(\lambda, \mu)|^2 \\
\lambda - \mu^2 + \mu & = & |f(\lambda, \mu)|^2 \\
f(\lambda, \mu) & = & \sqrt{\lambda - \mu^2 + \mu}
\eea
and, similarly, $g(\lambda, \mu) = \sqrt{\lambda - \mu^2 - \mu}$. These coefficients must be real due to the fact that $J_+$ and $J_-$ are hermitian operators. This constraint imposes the following
conditions on $\lambda$ and $\mu$:
\bea
\lambda - \mu^2 + \mu & \geq & 0 \\
\lambda - \mu^2 - \mu & \geq & 0
\eea
which can be satisfied by letting $\lambda = j(j+1)$ for some $j$ with:
\beq
j \geq \mu \geq -j. \label{muspec}
\eeq

Note that equation (\ref{jplus}) shows that the action of $J_+$ is
composed of a reflection which changes sign of $\mu$, the
eigenvalue of $J_3$, followed by raising by one unit. Similarly,
equation (\ref{jminus}) shows that $J_-$ reflects and lowers. Thus
the highest state $\mu = j$ is annihilated by $J_-$ and "lowered"
by $J_+$. Applying $J_+$ or $J_-$ twice to any state gives back
the same state due to relations (\ref{jp^2}) and (\ref{jm^2}).
Thus starting from the highest state we apply $J_-$ and $J_+$
alternately to get the spectrum: \beq j, -j+1, j-2, -j+3, ... \eeq
This sequence ends so as to satisfy equation (\ref{muspec}) only
for integer or half-integer $j$. For integer $j$, it terminates,
after an even number of steps, at $-j$ and visits every integer in
between only once. For half-integer $j=2k\pm\frac{1}{2}$ it ends
at $j=\pm\frac{1}{2}$ having visited only half the states with
$\mu$ half-odd integer between $j$ and $-j$. The rest of the
states cannot be reached from these but are obtained by starting
from the $\mu= -j$ state and applying $J_-$ and $J_+$ alternately;
starting with $J_-$.
\\
We now give a few examples:
\begin{itemize}
  \item
For $\mathbf{j=2}$ the states follow the sequence:
\[
\mathbf{\mu = 2, -1, 0, 1, -2}\quad.
\]
  \item
For $\mathbf{j=\frac{3}{2}}$ there exist two irreducible
representations one with:
\[
\mathbf{\mu = \frac{3}{2}, -\frac{1}{2}}\quad,
\] and the other with:
\[
\mathbf{\mu = -\frac{3}{2}, \frac{1}{2}}\quad.
\]

  \item
For $\mathbf{j=\frac{5}{2}}$ the two representations are given by:
\[
\mathbf{\mu = \frac{5}{2}, -\frac{3}{2}, \frac{1}{2}}\quad,
\] and by:
\[
\mathbf{\mu = -\frac{5}{2}, \frac{3}{2}, -\frac{1}{2}}\quad.
\]

\end{itemize}

\section{Hopf Algebra Structure with braiding}

One natural question to ask having considered this associative algebra is whether or not it has a Hopf algebra structure. On the surface, this algebra shares a lot with its sister algebra, the $SU(2)$ Lie algebra, which has a Hopf algebra structure and one would expect ACSA to similarly have one. It turns out, however, that naively trying the same coproduct rule for ACSA does not work due to the symmetric nature of the product defined on ACSA since the product is defined in terms of anticommutators. As was noted in the Introduction of this work, the coproduct of the Lie algebra requires the product on the Lie algebra to be anti-symmetric. For this reason, the coproduct of the $SU(2)$ Lie algebra is not suitable for ACSA.

In our quest for a Hopf algebra structure for ACSA, it would be more fruitful to understand the nature of the relationship of ACSA with the $SU(2)$ Lie algebra. If one names the generators of the $SU(2)$ algebra $I_i$, then it can easily be shown that $\tilde{J_i}$ defined as
\beq
\tilde{J}_i = - I_i \otimes \sigma_i
\eeq
satisfy the defining relation of ACSA since:
\bean
\tilde{J}_i \tilde{J}_j + \tilde{J}_j \tilde{J}_i &=& I_i I_j \otimes \sigma_i \sigma_j + I_j I_i \otimes \sigma_j \sigma_i\\
&=& I_i I_j \otimes i \sigma_k + I_j I_i \otimes -i \sigma_k \\
&=& i(I_i I_j - I_j I_i) \otimes \sigma_k \\
&=& i(iI_k) \otimes \sigma_k \\
&=& - I_k \otimes \sigma_k \\
&=& \tilde{J}_k
\quad \mbox{for}\quad  i \neq j \neq k \neq i.
\eean

Similarly, the generators satisfying the $SU(2)$ Lie algebra can be written in terms of ACSA generators as:
\beq
\tilde{I}_i = J_i \otimes \sigma_i
\eeq
since:
\bean
\tilde{I}_i \tilde{I}_j - \tilde{I}_j \tilde{I}_i &=& J_i J_j \otimes \sigma_i \sigma_j - J_j J_i \otimes \sigma_j \sigma_i\\
&=& J_i J_j \otimes i \sigma_k - J_j J_i \otimes -i \sigma_k \\
&=& i(J_i J_j + J_j J_i) \otimes \sigma_k \\
&=& i(J_k) \otimes \sigma_k \\
&=& \tilde{I}_k
\quad \mbox{for}\quad  i \neq j \neq k \neq i.
\eean

These two relations show that the $SU(2)$ algebra and ACSA are so closely related that it is not even possible to identify which one of these algebras is more fundamental. Both of them can be written in terms of the generators of the other and their algebraic structure can be derived from the structure of the other one. However, as mentioned, the Hopf algebra structure of ACSA cannot be derived from the Hopf algebra structure of the $SU(2)$ algebra. Specifically, ACSA does not admit a coproduct defined in a normal way using the usual tensor products. Such a coproduct can be defined if one were to extend the permutation operator $\tau$ used in the connecting relation. Normally the operation of $\tau$ is defined as:
\beq
\tau(A \otimes B) = B \otimes A \quad ,
\eeq
however, if one considers the algebra to be graded and one were to define a degree operator ($deg$) which is $0$ for bosonic variables and is $1$ for fermionic variables, then the natural redefinition of the $\tau$ operator is
\beq
\tau(A \otimes B) = (-1)^{deg\;A\; deg\;B\;} B \otimes A \quad .
\eeq
Using this redefined permutation operator, one can still write down the Hopf algebra relations and only the connecting relation will be the one that will be redefined; thus, we arrive at a braided Hopf algebra structure.

When the permutation operator is redefined in this way, the product of two tensor product terms is given by $(A\otimes B) (C\otimes D) = (-1)^{deg\;B\; deg\;C\;} (AC \otimes BD)$ where the $-1$ factor comes in because of the reordering of the $B$ and $C$ terms. Using this rule and defining the degree of $1$ as $0$ and the degrees of $J_1, J_2, J_3$ as $1$, we can see that the coproduct defined as:
\bea
\Delta(J_i) &=& 1 \otimes J_i + J_i \otimes 1 \\
\Delta(1) &=& 1 \otimes 1
\eea
satisfies the algebra structure relations since:
\bean
\Delta(J_i)\Delta(J_j)
&=& (1 \otimes J_i + J_i \otimes 1)(1 \otimes J_j + J_j \otimes 1) \\
&=& 1 \otimes J_i J_j - 1 J_j \otimes J_i 1 + J_i 1 \otimes 1 J_j + J_i J_j \otimes 1 \\
&=& 1 \otimes J_i J_j - J_j \otimes J_i + J_i \otimes J_j + J_i J_j \otimes 1
\eean
and
\bean
\Delta(J_i)\Delta(J_j) + \Delta(J_j)\Delta(J_i)
&=& 1 \otimes J_i J_j - J_j \otimes J_i + J_i \otimes J_j + J_i J_j \otimes 1 \\
& & + 1 \otimes J_j J_i - J_i \otimes J_j + J_j \otimes J_i + J_j J_i \otimes 1 \\
&=& 1 \otimes J_i J_j + J_i J_j \otimes 1 \\
&=& 1 \otimes J_k + J_k \otimes 1 \\
&=& \Delta(J_k)
\quad \mbox{for}\quad  i \neq j \neq k \neq i.
\eean
The counit and coinverse are simpler and they match with the definitions for the normal Lie algebra, i.e.:
\bea
\epsilon(J_i) &=& 0 \\
S(J_i) &=& -J_i
\eea
These definitions of the coproduct, the counit and the coinverse give us a braided Hopf algebra structure for ACSA.

\chapter{QUANTUM GROUPS ASSOCIATED WITH INVARIANCE OF NON-DEFORMED OSCILLATORS}


The concepts of bosons and fermions lie at the heart of
microscopic physics. They are described in terms of creation and
annihilation operators of the corresponding particle algebra: \bea
c_i c_j \mp c_j c_i & = & 0 \\
c_i c^*_j \mp c^*_j c_i & = & \delta_{ij} \eea where the upper
sign is for the boson algebra $BA(d)$ and the lower sign is for
the fermion algebra $FA(d)$.

It has been realized that quantum algebras play an important role
in the description of physical phenomena. Some classical physical
systems which are invariant under a classical Lie group, when
quantized, are invariant under a quantum group \cite{fadeev}. The
quantum groups thus considered turn out to be $q$-deformations of
the classical semisimple groups. On the other hand, inhomogeneous
quantum groups \cite{sww,pw} are perhaps more interesting since
classical inhomogeneous groups such as the Poincar\'e group are
more important in physics.

In this paper we will investigate an important class of
inhomogeneous quantum groups which are related to the boson
algebra $BA(d)$ and the fermion algebra $FA(d)$. Although $BA(d)$
and $FA(d)$ themselves are not quantum groups, by considering
quantum group versions of symmetry transformations acting on these
algebras, one can arrive at these inhomogeneous quantum groups.
Mathematically speaking we are thus interested in constructing
left modules of these algebras such that these modules have Hopf
algebra structure.

Traditionally the boson algebra has the symmetry group $ISp(2d,
\IR)$, the inhomogeneous symplectic group, which transforms
creation and annihilation operators as: \beq c_i \rightarrow
\alpha_{ij} c_j + \beta_{ij} c^*_j + \gamma_i \quad . \eeq In this
transformation $\alpha_{ij}, \beta_{ij}, \gamma_i$ are complex
numbers satisfying the constraints required by the group $ISp(2d,
\IR)$. One should note that this symmetry group is also the group
of linear canonical transformations of a classical dynamical
system. An important physical application of this transformation
is the Bogoliubov transformation which is crucial in the
explanation of many quantum mechanical effects such as the Unruh
Effect \cite{unruh}
 and Hawking Radiation \cite{hawking}.
In the case of the Hawking Radiation, the physical
reinterpretation of the transformed operators imply that the
future vacuum state is annihilated by the transformed annihilation
operator, which is related to the initial creation and
annihilation operators by a Bogoliubov transformation.

Similar to the boson algebra, the fermion algebra has the
classical symmetry group $O(2d)$ with the transformation law: \beq
c_i \rightarrow \alpha_{ij} c_j + \beta_{ij} c^*_j \quad . \eeq
however, unlike its bosonic counterpart this algebra is not
inhomogeneous. This fact is the primary motivation for the
generalization that we are going to offer. By relaxing the
conditions on the transformation coefficients such as
commutativity, one can come up with inhomogeneous invariance
(quantum)groups for fermions and for bosons alike. The explicit
$R$-matrices utilizing the quantum group properties of these
structures have already been presented \cite{agy, ab}. In this
paper, after a brief definition of these quantum groups \FIO, the
fermionic inhomogeneous orthogonal quantum group, and \BISp, the
bosonic inhomogeneous symplectic quantum group, in Section 1, we
will investigate their sub(quantum)groups and also study the
(quantum)groups obtained by their contractions. In the last
section, $FIO(2d + 1, \IR)$, the fermionic inhomogeneous quantum
orthogonal group in odd number of dimensions, will also be defined
and its properties examined.

A general transformation of a particle algebra can be described in
the following way: \beq \left(
\begin{array}{c}
c' \\
{c^{*}}' \\
1
\end{array}
\right) = \left(
\begin{array}{ccc}
\alpha & \beta & \gamma \\
\beta^* & \alpha^* & \gamma^* \\
0 & 0 & 1
\end{array}
\right) \dot{\otimes} \left(
\begin{array}{c}
c \\
c^* \\
1
\end{array}
\right) \eeq where $c, c^*, \gamma, \gamma^*$ are column matrices
and $\alpha, \beta, \alpha^*, \beta^*$ are $d\times d$ matrices.
Thus, in index notation the transformation is given by: \bea
c'_i &=& \alpha_{ij} \otimes c_j + \beta_{ij} \otimes c^*_j + \gamma_i \otimes 1 \quad , \\
{c^{*}}'_i &=& \alpha^*_{ij} \otimes c^*_j + \beta^*_{ij} \otimes
c_j + \gamma^*_i \otimes 1 \quad . \eea

Given this transformation, we look for an algebra $\mathcal{A}$
generated by these matrix elements such that the particle algebra
remains invariant. Thus, we first write the transformation matrix
in the above equation in the following way: \beq M = \left(
\begin{array}{cc|c}
\alpha & \beta & \gamma \\
\beta^* & \alpha^* & \gamma^* \\
\hline 0 & 0 & 1
\end{array}
\right)
 =
\left(
\begin{array}{c|c}
A & \Gamma \\
\hline 0 & 1
\end{array}
\right) \quad . \label{M-matrix} \eeq

We assume that $\alpha_{ij}, \beta_{ij}, \gamma_i$ belong to a
possibly noncommutative algebra on which a hermitian conjugation
denoted by $*$ is defined.

\section{The Bosonic Inhomogeneous Symplectic Quantum Group \BISp}

If we consider the transformation matrix \eqref{M-matrix} being applied to the boson algebra given by:
\bea
c_i c_j - c_j c_i & = & 0 \\
c_i c^*_j - c^*_j c_i & = & \delta_{ij}
\eea
then we require that the transformed operators $c'_i$ and ${c^*}'_i$ are required to satisfy the same algebra in order for the transformation to be an algebra invariance. Thus we require that:
\bea
c'_i c'_j - c'_j c'_i & = & 0 \\
c'_i {c^*}'_j - {c^*}'_j c'_i & = & \delta_{ij}
\eea
Explicitly writing out the transformed operators, these relations become:
\begin{align}
\begin{split}
(\alpha_{ik} \otimes c_k + \beta_{ik} \otimes c^*_k + \gamma_i \otimes 1)
(\alpha_{jl} \otimes c_l + \beta_{jl} \otimes c^*_l + \gamma_j \otimes 1) \\
- (\alpha_{jl} \otimes c_l + \beta_{jl} \otimes c^*_l + \gamma_j \otimes 1)
(\alpha_{ik} \otimes c_k + \beta_{ik} \otimes c^*_k + \gamma_i \otimes 1)
&=  0
\end{split}
\\
\begin{split}(\alpha_{ik} \otimes c_k + \beta_{ik} \otimes c^*_k + \gamma_i \otimes 1)
(\alpha^*_{jl} \otimes c^*_l + \beta^*_{jl} \otimes c_l + \gamma^*_j \otimes 1) \\
- (\alpha^*_{jl} \otimes c^*_l + \beta^*_{jl} \otimes c_l + \gamma^*_j \otimes 1)
(\alpha_{ik} \otimes c_k + \beta_{ik} \otimes c^*_k + \gamma_i \otimes 1)
& =  \delta_{ij}
\end{split}
\end{align}
which gives us:
\beq
\begin{split}
& [\alpha_{ik}, \alpha_{jl}]c_l c_k + [\beta_{ik}, \beta_{jl}]c^*_l c^*_k \\
& + [\alpha_{ik}, \gamma_j] c_k + [\beta_{ik}, \gamma_j] c^*_k \\
& + [\gamma_i, \alpha_{jl}] c_l + [\gamma_i, \beta_{jl}] c^*_l \\
& + [\alpha_{ik}, \beta_{jl}] c_k c^*_l + [\beta_{ik}, \alpha_{jl}] c^*_k c_l \\
& + (\alpha_{jk} \beta_{ik} - \beta_{jk} \alpha_{ik} + [\gamma_i, \gamma_j]) =  0 \quad ,
\end{split}
\eeq
and
\beq
\begin{split}
& [\alpha_{ik}, \beta^*_{jl}]c_l c_k + [\beta_{ik}, \alpha^*_{jl}] c^*_l c^*_k \\
& + [\alpha_{ik}, \gamma^*_j] c_k + [\beta_{ik}, \gamma^*_j] c^*_k \\
& + [\gamma_i, \beta^*_{jl}] c_l + [\gamma_i, \alpha_{jl}] c^*_l \\
& + [\alpha_{ik}, \alpha^*_{jl}] c_k c^*_l + [\beta_{ik}, \beta^*_{jl}] c^*_k c_l \\
& + (\alpha^*_{jk} \alpha_{ik} - \beta^*_{jk} \beta_{ik} + [\gamma_i, \gamma^*_j]) = \delta_{ij} \quad .
\end{split}
\eeq

In the first of these relations, for the equality to be satisfied, it is sufficient for
the coefficients of all the terms on the left hand side to be equal to zero.
In the second one, however, we only have a term that is a multiple of the unit
element of the boson algebra on the right hand side, thus the coefficient of that term
should be equal on both sides and it is sufficient for the coefficients of the
other terms on the left hand side to be separately equal to zero.

Thus we have the following relations between the transformation elements:
\bea
\gamma_i \gamma^*_j - \gamma^*_j \gamma_i &=& \delta_{ij} - \alpha_{ik}\alpha^*_{jk} + \beta_{ik} \beta^*_{jk} \label{rel1bos} \\
\gamma_i \gamma_j - \gamma_j \gamma_i &=& \beta_{ik} \alpha_{jk} - \alpha_{ik} \beta_{jk} \label{rel2bos} \\
\alpha_{ij} \gamma_k - \gamma_k \alpha_{ij} & = & 0 \label{rel3bos} \\
\beta_{ij} \gamma_k - \gamma_k \beta_{ij} & = & 0 \label{rel4bos} \\
\alpha_{ij} \gamma^*_k - \gamma^*_k \alpha_{ij} & = & 0 \label{rel5bos} \\
\beta_{ij} \gamma^*_k - \gamma^*_k \beta_{ij} & = & 0 \label{rel6bos}
\eea
and any two elements from the set
$\alpha_{ij}, \beta_{ij}, \alpha^*_{ij}, \beta^*_{ij}$ commute.

The set of matrices $M$ obeying the above relations form the group
of inhomogeneous transformations of bosons. We name this quantum group
as the bosonic inhomogeneous symplectic quantum group \BISp
since it is an inhomogeneous extension of the symplectic group where
the inhomogeneous part exhibits bosonic behavior. This
symmetry group, however, is not a classical group like the symplectic group
but is in fact a quantum group with a Hopf algebra structure. As shown in
\cite{ab}, this Hopf algebra has an explicit $R$-matrix
representation and the coproduct, counit and coinverse are defined
as:
\bea
\Delta(M) & = & M \dot{\otimes} M \label{coproductbos} \\
\epsilon(M) & = & I \label{counitbos} \\
S(M) & = & M^{-1} \label{antipodebos} \quad . \eea In Equation
\eqref{coproductbos}, the symbol $\dot{\otimes}$ denotes the usual
matrix multiplication where when elements of the matrices are
multiplied, tensor multiplication is used.

The inverse of the matrix $M$ can be defined as: \beq M^{-1} =
\left(
\begin{array}{cc}
A^{-1} & -A^{-1} \Gamma \\
0 & 1
\end{array}
\right) \eeq where $A^{-1}$ is defined in the standard way since
matrix elements of $A$ are shown to be commutative.

\subsection{Subgroups}
After having shown that the inhomogeneous transformations of the boson
algebra forms the symmetry quantum group \BISp,
one important question is what sub(quantum)groups does this quantum
group have. For example, we know that the group $ISp(2d, \IR)$ is
an important special subgroup of \BISp and other
sub(quantum)groups could turn out to have similarly important
physical applications. While searching for sub(quantum)groups, we
would also like to find new (quantum)groups allowed by suitable
contractions \cite{inonu} of these quantum groups as well.

The sub(quantum)groups are obtained by imposing
additional relations on the matrix elements of $M$ which obey the
relations (\ref{rel1bos}) - (\ref{rel6bos}). The additional relations
that we will impose are:
\begin{subequations}
\bea
\delta_{ij} - \alpha_{ik}\alpha^*_{jk} + \beta_{ik} \beta^*_{jk} = \beta_{ik} \alpha_{jk} - \alpha_{ik} \beta_{jk} = 0 \label{suba} \\
\gamma_i = 0  \label{subb} \\
\beta_{ij} = 0  \label{subc} \\
\alpha_{ij} = 0  \label{subd}
\eea
\end{subequations}

We would like to study the implication of each relation one by one
in the following subsections.

\subsubsection{Inhomogeneous Subgroup}

The relation \eqref{suba}:
\[
\delta_{ij} - \alpha_{ik}\alpha^*_{jk} + \beta_{ik} \beta^*_{jk}
= \beta_{ik} \alpha_{jk} - \alpha_{ik} \beta_{jk} = 0
\]
by virtue of \eqref{rel1bos} and \eqref{rel2bos} implies that $\gamma_i
\gamma^*_j - \gamma^*_j \gamma_i = 0$ and $\gamma_i \gamma_j -
\gamma_j \gamma_i = 0$, i.e. that the inhomogeneous transformation
parameters are commutative variables.

For bosonic particles, the fact that the inhomogeneous elements of the
quantum group are commutative elements coupled with the fact that the
remaining relations between the transformation elements are already
commutative gives us a symmetry transformation of the boson algebra
where all the elements commute. However, we know that such a transformation
is nothing but the classical symmetry group of the bosonic particle algebra $ISp(2d,
\IR)$ in which all the parameters, including the inhomogeneous
elements, are commutative.

\subsubsection{Homogeneous Subgroup}

The relation \eqref{subb}:
\[
\gamma_i = 0
\]
practically gets rid of the inhomogeneous part of the
transformation and also implies the relation \eqref{suba} considered
in the previous subsection. Since the previous relation is implied
the resulting group will be a subgroup of $ISp(2d,
\IR)$ and since the group is not inhomogeneous anymore the resulting
subgroup is the classical symplectic group $Sp(2d, \IR)$.

\subsubsection{Bosonic Inhomogeneous Unitary Quantum Group}
The relation \eqref{subc}:
\[
\beta_{ij} = 0
\]
applied to the transformation gets rid of the off-diagonal members
of the homogeneous part of it and leaves us with the following
relation:
\bea
\gamma_i\gamma^*_j - \gamma^*_j\gamma_i &=& \delta_{ij} - \alpha_{ik}\alpha^*_{jk} \\
\gamma_i\gamma_j - \gamma_j\gamma_i &=& 0
\eea

This equation implies for the homogeneous
part of the transformation the relation:
\beq
\delta_{ij} = \alpha_{ik}\alpha^*_{jk}
\eeq
which tells us that the submatrices
$\alpha$ and $\alpha^*$ in equation \eqref{M-matrix} are
members of $U(d)$. The subgroup we have arrived at thus is an
inhomogeneous quantum group extension to the classical homogeneous
group $U(d)$. Since the inhomogeneous elements of the resulting group
obeys the same relations as \BISp, we will name this quantum group
 $BIU(d)$, the bosonic inhomogeneous quantum group.

\subsubsection{Boson Algebra}

The relation \eqref{subd}:
\[
\alpha_{ij} = 0
\]
applied alone onto the transformation gets rid of the diagonal
members of the homogeneous part and prevents such transformations
from forming a (quantum)group since the homogeneous parts of these
set of transformations can never include the identity
transformation.

However, if this relation is applied together with the previous
one, relation \eqref{subc}, the resulting relation gets rid of the
whole homogeneous part of the transformation leaving only the
inhomogeneous part and leaves us with two relations:
\bea
\gamma_i\gamma^*_j - \gamma^*_j\gamma_i &=& \delta_{ij} \\
\gamma_i\gamma_j - \gamma_j\gamma_i &=& 0
\eea
which
gives us back the boson algebra, $BA(d)$.

We should note, however, that after this condition is applied, the
resulting set of matrices $M$, which now form $BA(d)$, is
no longer a quantum or classical group since the antipode defined
in equation \eqref{antipodebos} no longer exits. For this reason,
the boson algebra can be considered to be a boundary for the
sub(quantum)groups of \BISp.

\subsubsection{Sub(quantum)group Diagram}
As a result of the above discussion, we get the sub(quantum)group
diagram:
\[
\begin{CD}
{BISp(2d, \IR)} @>\eqref{suba}>> {ISp(2d, \IR)} @>\eqref{subb}>> {Sp(2d, \IR)} \\
@V\eqref{subc}VV @V\eqref{subc}VV @V\eqref{subc}VV \\
{BIU(d)} @>\eqref{suba}>> {IU(d)} @>\eqref{subb}>> {U(d)} \\
@V\eqref{subd})VV \\
{BA(d)}
\end{CD}
\]
for the sub(quantum)groups of the \BISp we have introduced in this
section.

\subsection{Contractions}
In order to explore the new (quantum)groups that will come about
as the result of a contraction, we replace $\gamma_i$ by
$\gamma_i/\sqrt{\hbar}\;$ so that we may consider the case $\hbar
\rightarrow 0$. After this replacement, the equations \eqref{rel1bos}
and \eqref{rel2bos} become:
\bea
\gamma_i \gamma^*_j - \gamma^*_j \gamma_i &=& \hbar(\delta_{ij} - \alpha_{ik}\alpha^*_{jk} + \beta_{ik} \beta^*_{jk}) \\
\gamma_i \gamma_j - \gamma_j \gamma_i &=& \hbar(\beta_{ik} \alpha_{jk} - \alpha_{ik} \beta_{jk})
\eea
When we consider the
case $\hbar \rightarrow 0$, we get the relations:
\bea
\gamma_i \gamma^*_j - \gamma^*_j \gamma_i &=& 0 \\
\gamma_i \gamma_j - \gamma_j \gamma_i &=& 0
\eea
which imply
that the inhomogeneous part of the transformation form
ordinary complex numbers. What makes this case different
from the previous
case of subgroups is that the homogeneous part of this
transformation forms a matrix $A$ with non-zero determinant. We
can transform such a matrix $A$ with a similarity transformation
given by the unitary matrix:
\beq
U = \frac{1}{\sqrt{2}}
\left(
\begin{array}{cc}
1 & 1 \\
i & -i
\end{array}
\right)
\eeq
to put it in a real form. The transformation gives:
\beq
\begin{split}
A'
&= U A U^\dagger \\
&= \frac12
    \begin{pmatrix}
      1 & 1 \\
      i & -i
    \end{pmatrix}
    \begin{pmatrix}
      \alpha & \beta \\
      \beta^* & \alpha^*
    \end{pmatrix}
    \begin{pmatrix}
      1 & -i \\
      1 & i
    \end{pmatrix}
    \\
&=  \begin{pmatrix}
      Re(\alpha) + Re(\beta) & Im(\alpha) - Im(\beta) \\
      - Im(\alpha) - Im(\beta) & Re(\alpha) - Re(\beta)
    \end{pmatrix}
\end{split}
\eeq
which is a real matrix that is a member of the
general linear group $GL(2d, \IR)$. Thus we
have the group $IGL(2d, \IR)$, the inhomogeneous general linear group.

If we consider the contraction of the subgroups as well then we
should examine the $\hbar \rightarrow 0$ limit after the relations
\eqref{subc} and \eqref{subd} are applied.

After we apply relation \eqref{subc}, we get the subgroup $BIU(d)$
as discussed previously. After the contraction,
again, the inhomogeneous part of this group become
complex numbers. However, if we apply the previous similarity
transformation on the homogeneous part, we get:
\beq
\begin{split}
A'
&=  U A U^\dagger \\
&=  \frac12
      \begin{pmatrix}
        1 & 1 \\
        i & -i
      \end{pmatrix}
      \begin{pmatrix}
        \alpha & 0 \\
        0 & \alpha^*
      \end{pmatrix}
      \begin{pmatrix}
        1 & -i \\
        1 & i
      \end{pmatrix}
    \\
&=
    \begin{pmatrix}
      Re(\alpha) & Im(\alpha) \\
      - Im(\alpha) & Re(\alpha)
    \end{pmatrix}
    \\
&= Re(\alpha) \I1 + Im(\alpha) \II
\end{split}
\eeq
where $\I1$ stands for the identity matrix and
$\II$ stands for the matrix the square of which is minus the
identity matrix. This way we can see that the matrix $A'$ is a
actually member of $GL(d, \IC)$. This gives us $IGL(d, \IC)$
as the group we arrive at as the contraction of $BIU(d)$.

We have previously shown that we get the boson
algebra after applying both of the relations \eqref{subc} and
\eqref{subd}. We have also discussed that in this case only the
inhomogeneous part of the transformation survives. After
applying the contraction, the surviving inhomogeneous part of the
transformation turns into complex variables. Thus
in this case, the contraction of $BA(d)$ gives us $\IC^d$.

As a summary, for the contraction considered in this section
applied onto the subgroups obtained in the previous subsection
we get the following group diagram:
\[
\begin{CD}
{BISp(2d, \IR)} @>\hbar \rightarrow 0>> {IGL(2d, \IR)} \\
@V\eqref{subc}VV @V\eqref{subc}VV \\
{BIU(d)} @>\hbar \rightarrow 0>> {IGL(d, \IC)} \\
@V\eqref{subd}VV @V\eqref{subd}VV \\
{BA(d)} @>\hbar \rightarrow 0>> {\IC^d}
\end{CD}
\]

\section{The Fermionic Inhomogenous Group \FIO}

Similarly to how it was done in the bosonic case one can also
consider the transformation matrix
\eqref{M-matrix} being applied to the fermion algebra given by:
\bea
c_i c_j + c_j c_i & = & 0 \\
c_i c^*_j + c^*_j c_i & = & \delta_{ij}
\eea
and then require that the transformed operators $c'_i$ and ${c^*}'_i$ satisfy
the same algebra in order for the transformation to be an algebra invariance.
Thus the requirement is that:
\bea
c'_i c'_j + c'_j c'_i & = & 0 \\
c'_i {c^*}'_j + {c^*}'_j c'_i & = & \delta_{ij}
\eea
Explicitly writing out the transformed operators, these relations become:
\begin{align}
\begin{split}
(\alpha_{ik} \otimes c_k + \beta_{ik} \otimes c^*_k + \gamma_i \otimes 1)
(\alpha_{jl} \otimes c_l + \beta_{jl} \otimes c^*_l + \gamma_j \otimes 1) \\
+(\alpha_{jl} \otimes c_l + \beta_{jl} \otimes c^*_l + \gamma_j \otimes 1)
(\alpha_{ik} \otimes c_k + \beta_{ik} \otimes c^*_k + \gamma_i \otimes 1)
& = 0
\end{split} \\
\begin{split}
(\alpha_{ik} \otimes c_k + \beta_{ik} \otimes c^*_k + \gamma_i \otimes 1)
(\alpha^*_{jl} \otimes c^*_l + \beta^*_{jl} \otimes c_l + \gamma^*_j \otimes 1) \\
+(\alpha^*_{jl} \otimes c^*_l + \beta^*_{jl} \otimes c_l + \gamma^*_j \otimes 1)
(\alpha_{ik} \otimes c_k + \beta_{ik} \otimes c^*_k + \gamma_i \otimes 1)
& = \delta_{ij}
\end{split}
\end{align}
which gives us:
\beq
\begin{split}
&[\alpha_{jl}, \alpha_{ik}]c_l c_k + [\beta_{jl}, \beta_{ik}]c^*_l c^*_k  \\
&+ \{\alpha_{ik}, \gamma_j\} c_k + \{\beta_{ik}, \gamma_j\} c^*_k  \\
&+ \{\gamma_i, \alpha_{jl}\} c_l + \{\gamma_i, \beta_{jl}\} c^*_l  \\
&+ [\alpha_{ik}, \beta_{jl}] c_k c^*_l + [\beta_{ik}, \alpha_{jl}] c^*_k c_l  \\
&+ (\alpha_{jk} \beta_{ik} + \beta_{jk} \alpha_{ik} + \{\gamma_i, \gamma_j\}) = 0 \quad ,
\end{split}
\eeq
and
\beq
\begin{split}
&[\beta^*_{jl}, \alpha_{ik}]c_l c_k + [\alpha^*_{jl}, \beta_{ik}] c^*_l c^*_k \\
&+ \{\alpha_{ik}, \gamma^*_j\} c_k + \{\beta_{ik}, \gamma^*_j\} c^*_k \\
&+ \{\gamma_i, \beta^*_{jl}\} c_l + \{\gamma_i, \alpha_{jl}\} c^*_l \\
&+ [\alpha_{ik}, \alpha^*_{jl}] c_k c^*_l + [\beta_{ik}, \beta^*_{jl}] c^*_k c_l \\
&+ (\alpha^*_{jk} \alpha_{ik} + \beta^*_{jk} \beta_{ik} + \{\gamma_i, \gamma^*_j\}) = \delta_{ij} \quad .
\end{split}
\eeq

In the first of these relations, for the equality to be satisfied, it is sufficient for
the coefficients of all the terms on the left hand side to be equal to zero.
In the second one, however, we only have a term that is a multiple of the unit
element of the boson algebra on the right hand side, thus the coefficient of that term
should be equal on both sides and it is sufficient for the coefficients of the
other terms on the left hand side to be separately equal to zero.

Thus we have the following relations between the transformation elements:
\bea
\gamma_i \gamma^*_j + \gamma^*_j \gamma_i &=& \delta_{ij} - \alpha_{ik}\alpha^*_{jk} - \beta_{ik} \beta^*_{jk} \label{rel1fer} \\
\gamma_i \gamma_j + \gamma_j \gamma_i &=& - \beta_{ik} \alpha_{jk} - \alpha_{ik} \beta_{jk} \label{rel2fer} \\
\alpha_{ij} \gamma_k + \gamma_k \alpha_{ij} & = & 0 \label{rel3fer} \\
\beta_{ij} \gamma_k + \gamma_k \beta_{ij} & = & 0 \label{rel4fer} \\
\alpha_{ij} \gamma^*_k + \gamma^*_k \alpha_{ij} & = & 0 \label{rel5fer} \\
\beta_{ij} \gamma^*_k + \gamma^*_k \beta_{ij} & = & 0 \label{rel6fer}
\eea
and any two elements from the set
$\alpha_{ij}, \beta_{ij}, \alpha^*_{ij}, \beta^*_{ij}$ commute.

The set of matrices $M$ obeying the above relations form the group
of inhomogeneous transformations of fermions. We call this quantum group
the fermionic inhomogeneous orthogonal quantum group \FIO
since it is an inhomogeneous extension of the orthogonal group where
the inhomogeneous part exhibits fermionic behavior. This
symmetry group, like its sister \BISp, is not a classical group
but is a quantum group with a Hopf algebra structure. Similar to
the case with \BISp, this Hopf algebra has an explicit $R$-matrix
representation and the coproduct, counit and coinverse are defined
as:
\bea
\Delta(M) & = & M \dot{\otimes} M \label{coproductfer} \\
\epsilon(M) & = & I \label{counitfer} \\
S(M) & = & M^{-1} \label{antipodefer} \quad . \eea

\subsection{Subgroups}
We have shown that there is a rich sub(quantum)group structure for
\BISp and it should naturally follow that there should be a similarly
rich sub(quantum)group structure for the fermionic counterpart \FIO.

In this subsection this sub(quantum)group structure will be explored using
relations similar to the ones considered for \BISp:
\begin{subequations}
\bea
\delta_{ij} - \alpha_{ik}\alpha^*_{jk} - \beta_{ik} \beta^*_{jk} = - \beta_{ik} \alpha_{jk} - \alpha_{ik} \beta_{jk} &=& 0 \label{subafer} \\
\gamma_i &=& 0  \label{subbfer} \\
\beta_{ij} &=& 0  \label{subcfer} \\
\alpha_{ij} &=& 0  \quad. \label{subdfer}
\eea
\end{subequations}
The implication of each of these relations will be explored in
the following subsections.

\subsubsection{Inhomogeneous Subsupergroup}
The relation \eqref{subafer}:
\[
\delta_{ij} - \alpha_{ik}\alpha^*_{jk} - \beta_{ik} \beta^*_{jk}
= - \beta_{ik} \alpha_{jk} - \alpha_{ik} \beta_{jk} = 0
\]
by virtue of \eqref{rel1fer} and \eqref{rel2fer} implies that $\gamma_i
\gamma^*_j + \gamma^*_j \gamma_i = 0$ and $\gamma_i \gamma_j +
\gamma_j \gamma_i = 0$, i.e. that the inhomogeneous transformation
parameters are anticommutative variables.

Thus we end up with an
inhomogeneous orthogonal algebra where the inhomogeneous
parameters are grassmanian variables giving us the Grassmanian
inhomogeneous orthogonal group, $GrIO(2d, \IR)$, as the resulting
subgroup of \FIO. This subgroup of \FIO can also be considered
as an inhomogeneous supergroup. Actually, more generally, the
transformation elements $\alpha_{ij}$, $\beta_{ij}$, $\alpha^*_{ij}$
and $\beta^*_{ij}$ anticommute with
$\gamma_i$, $\gamma^*_i$ and the \FIO matrices $M$ are multiplied
with each other using the standard tensor product. One can also show
that $\alpha_{ij}$, $\beta_{ij}$, $\alpha^*_{ij}$ and $\beta^*_{ij}$ can
be taken to commute with $\gamma_i$, $\gamma^*_i$ provided that
the matrices $M$ are multiplied with a braided \cite{majid} tensor product,
eg. \beq (A \otimes B)(C \otimes D) = - AC \otimes BD \eeq whenever $B$ and
$C$ are both fermionic. This approach is similar to the approach that was
taken with ACSA to obtain a Hopf algebra structure. As a result of this
redefinition the treatment of the transformation elements corresponds to
the usual superalgebra approach, i.e. that the elements $\alpha_{ij}$,
$\beta_{ij}$, $\alpha^*_{ij}$ and $\beta^*_{ij}$ are bosonic and the elements
$\gamma_i$, $\gamma^*_i$ are fermionic.

\subsubsection{Homogeneous Subgroup}
The relation \eqref{subbfer}:
\[
\gamma_i = 0
\]
practically gets rid of the inhomogeneous part of the
transformation and also implies the relation \eqref{subafer} considered
in the previous subsection.

This gives us the subgroup in the previous subsection
without the inhomogeneous part which is basically the
classical orthogonal group $O(2d, \IR)$.

\subsubsection{Fermionic Inhomogeneous Unitary Quantum Group}
The relation \eqref{subcfer}:
\[
\beta_{ij} = 0
\]
applied to the transformation gets rid of the off-diagonal members
of the homogeneous part of it and leaves us with the following
relation:
\bea
\gamma_i\gamma^*_j + \gamma^*_j\gamma_i &=& \delta_{ij} - \alpha_{ik}\alpha^*_{jk} \\
\gamma_i\gamma_j + \gamma_j\gamma_i &=& 0
\eea

For the homogeneous
part of the transformation, this equation implies $ \delta_{ij} =
\alpha_{ik}\alpha^*_{jk} $ which tells us that the submatrices
$\alpha$ and $\alpha^*$ in equation \eqref{M-matrix} are both
members of $U(d)$. The subgroup we have arrived at thus is an
inhomogeneous quantum group whose homogeneous part is $U(d)$.
For fermions we will name this group the fermionic
inhomogeneous quantum group, $FIU(d)$, since the inhomogeneous part of
the transformation exhibits fermionic behavior.

\subsubsection{Fermion Algebra}
The relation \eqref{subdfer}:
\[
\alpha_{ij} = 0
\]
applied alone onto the transformation gets rid of the diagonal
members of the homogeneous part and prevents such transformations
from forming a (quantum)group since the homogeneous parts of these
set of transformations can never include the identity
transformation.

However, if this relation is applied together with the previous
one, relation \eqref{subcfer}, the resulting relation gets rid of the
whole homogeneous part of the transformation leaving only the
inhomogeneous part and gives us a single relation:
\bea
\gamma_i\gamma^*_j + \gamma^*_j\gamma_i &=& \delta_{ij} \\
\gamma_i\gamma_j + \gamma_j\gamma_i &=& 0
\eea
which
gives us back the fermion algebra, $FA(d)$.

It should be noted that the fermion algebra in $d$ dimensions is isomorphic
to the Clifford algebra in $2d$ dimensions. If one considers operators $\psi_i$
defined as:
\bea
\psi_i =
\begin{cases}
   i (c_{i/2} - c^*_{i/2}) & \text{when $i$ is even} \\
   (c_{(i + 1)/2} + c^*_{(i + 1)/2}) & \text{when $i$ is odd} \\
\end{cases}
\eea
where $i = 1, 2,  \ldots, 2d$ and $c_j$ are elements of the fermion algebra,
then one can show that $\psi_i$ satisfy the Clifford algebra rule:
\beq
\{\psi_i, \psi_j\} = 2\delta_{ij}
\eeq
In order to show this, one only needs to prove the cases:
\beq
\{\psi_i, \psi_j\} =
\begin{cases}
  2\delta_{ij} & \text{$i$ and $j$ odd} \\
  2\delta_{ij} & \text{$i$ and $j$ even} \\
  0            & \text{$i$ odd, $j$ even} \\
\end{cases}
\eeq

For $i$ and $j$ both odd, $\{\psi_i, \psi_j\}$ becomes:
\beq
\begin{split}
\{\psi_i, \psi_j\}
&= (c_{(i + 1)/2} + c^*_{(i + 1)/2}) (c_{(j + 1)/2} + c^*_{(j + 1)/2}) \\
& \phantom{=} + (c_{(j + 1)/2} + c^*_{(j + 1)/2}) (c_{(i + 1)/2} + c^*_{(i + 1)/2}) \\
&= \{c_{(i + 1)/2}, c_{(j + 1)/2}\} + \{c^*_{(i + 1)/2}, c^*_{(j + 1)/2}\} \\
& \phantom{=} + \{c_{(i + 1)/2}, c^*_{(j + 1)/2}\} + \{c^*_{(i + 1)/2}, c_{(j + 1)/2}\} \\
&= 0 + 0 + \delta_{ij} + \delta_{ij} \\
&= 2 \delta_{ij}
\end{split}
\eeq
Similarly, when $i$ and $j$ are both even, one gets:
\beq
\begin{split}
\{\psi_i, \psi_j\}
&= i (c_{i/2} - c^*_{i/2}) i (c_{j/2} - c^*_{j/2}) \\
& \phantom{=} + i (c_{j/2} - c^*_{j/2}) i (c_{i/2} - c^*_{i/2}) \\
&= - \{c_{i/2}, c_{j/2}\} - \{c^*_{i/2}, c^*_{j/2}\} \\
& \phantom{=} + \{c_{i/2}, c^*_{j/2}\} + \{c^*_{i/2}, c_{j/2}\} \\
&= - 0 - 0 + \delta_{ij} + \delta_{ij} \\
&= 2 \delta_{ij}
\end{split}
\eeq
Finally, when $i$ is odd and $j$ is even, the anticommutator becomes:
\beq
\begin{split}
\{\psi_i, \psi_j\}
&= (c_{(i + 1)/2} + c^*_{(i + 1)/2}) i (c_{j/2} - c^*_{j/2}) \\
&  \phantom{=}+ i (c_{j/2} - c^*_{j/2}) (c_{(i + 1)/2} + c^*_{(i + 1)/2}) \\
&= i \{c_{(i + 1)/2}, c_{j/2}\} - i \{c^*_{(i + 1)/2}, c^*_{j/2}\} \\
&  \phantom{=}- i \{c_{(i + 1)/2}, c^*_{j/2}\} + i \{c^*_{(i + 1)/2}, c_{j/2}\} \\
&= 0 - 0 - \delta_{i+1,j} + \delta_{i+1,j} \\
&= 0
\end{split}
\eeq
thereby completing the proof that the operators $\psi_i$ form the elements of a Clifford algebra
of $2d$ dimensions. If one also considers that fact that the definition of $\psi_i$
is invertible and that $c_i$ can also be defined in terms of $\psi_i$, one can
conclude that the algebras $FA(d)$ and $Cliff(2d)$ are isomorphic.

\subsubsection{Sub(quantum)group Diagram}
As a result of the above discussion, we get the sub(quantum)group
diagram:
\[
\begin{CD}
{FIO(2d, \IR)}   @> \eqref{subafer} >> {GrIO(2d, \IR)} @> \eqref{subbfer} >> {O(2d, \IR)} \\
@V\eqref{subcfer}VV                    @V\eqref{subcfer}VV                   @V\eqref{subcfer}VV \\
{FIU(d)}         @> \eqref{subafer} >> {GrIU(d)}       @> \eqref{subbfer} >> {U(d)} \\
@V\eqref{subdfer}VV \\
{FA(d) \approx Cliff(2d)}
\end{CD}
\]
for the for the sub(quantum)groups of \FIO that have been introduced in
this section.

\subsection{Contractions}

It was observed for \BISp that using a suitable contraction one can
obtain new sub(quantum)groups. This should also be possible for \FIO and
the resulting structures will be examined in this section.

Similar to the bosonic treatment, we replace $\gamma_i$ by
$\gamma_i/\sqrt{\hbar}\;$ so that we may consider the case $\hbar
\rightarrow 0$. After this replacement, the equations \eqref{rel1fer}
and \eqref{rel2fer} become:
\bea
\gamma_i \gamma^*_j + \gamma^*_j \gamma_i &=& \hbar(\delta_{ij} - \alpha_{ik}\alpha^*_{jk} - \beta_{ik} \beta^*_{jk}) \\
\gamma_i \gamma_j + \gamma_j \gamma_i &=& \hbar(-\beta_{ik} \alpha_{jk} - \alpha_{ik} \beta_{jk})
\eea
and we consider the
case $\hbar \rightarrow 0$, we get the relations:
\bea
\gamma_i \gamma^*_j + \gamma^*_j \gamma_i &=& 0 \\
\gamma_i \gamma_j + \gamma_j \gamma_i &=& 0
\eea
which imply
that the inhomogeneous part of the transformation are Grassmanian
elements. What makes this case different
from the previous case of subgroups is that the homogeneous part of this
transformation forms a matrix $A$ with non-zero determinant. We
have previously shown that this matrix can be put in real form that
is a member of the general linear group $GL(2d, \IR)$. Since the
homogeneous part of the transformation is the general linear group
and the inhomogeneous part is Grassmanian, we
have the group $GrIGL(2d, \IR)$, the Grassmanian inhomogeneous general linear group
where the inhomogeneous part of the group are Grassmanian.

If we consider the contraction of the subgroups as well then we
should examine the $\hbar \rightarrow 0$ limit after the relations
\eqref{subcfer} and \eqref{subdfer} are applied.

After we apply relation \eqref{subcfer}, we get the subgroup $FIU(d)$
as discussed previously. After the contraction,
again, the inhomogeneous part of this group become
Grassmanian variables. However, as was shown during the contraction
of \BISp, if we apply the previous similarity
transformation on the homogeneous part of the resulting transformation
matrix after contraction, one can see that the
homogeneous part of the transformation is a member of the general
linear group $GL(d, \IC)$. Similarly this gives us $GrIGL(d, \IC)$.

We have previously shown that we get the fermion
algebra after applying both of the relations \eqref{subcfer} and
\eqref{subdfer}. We have also discussed that in this case only the
inhomogeneous part of the transformation survives. After
applying the contraction, the surviving inhomogeneous part of the
transformation turns into Grassmanian variables. Thus
in this case, the contraction of $FA(d)$ gives us $Gr(d, \IC)$,
the $d$ dimensional Grassman algebra.

As a summary, for the contraction considered combined with the
remaining subgroup relations we get the table:
\[
\begin{CD}
{FIO(2d, \IR)}            @> \hbar \rightarrow 0   >> {GrIGL(2d, \IR)} \\
@V\eqref{subcfer}VV                                    @V\eqref{subcfer}VV \\
{FIU(d)}                  @> \hbar \rightarrow 0   >> {GrIGL(d, \IC)} \\
@V\eqref{subdfer}VV                                    @V\eqref{subdfer}VV \\
{FA(d) \approx Cliff(2d)} @> \hbar \rightarrow 0    >> {Gr(d, \IC)}
\end{CD}
\]

\section{The Fermionic Inhomogeneous
Orthogonal Quantum Group of Odd Dimension}

The bosonic transformation quantum group \BISp can only be defined
in even dimensions and it is not possible to extend this definition
to odd dimension. However, as will be shown in this section, it is
possible to define the fermionic inhomogeneous orthogonal quantum
group of odd dimension. In order to show this, one should first
consider a unitary transformation of the \FIO matrix:
\[
M \rightarrow UMU^{-1}
\]
using the unitary matrix:
\beq U = \left(
\begin{tabular}{cc|c}
$\frac{1}{\sqrt{2}}$ & $\frac{1}{\sqrt{2}}$ & $0$ \\
$\frac{i}{\sqrt{2}}$ & $\frac{-i}{\sqrt{2}}$ & $0$ \\
\hline $0$ & $0$ & $1$
\end{tabular}
\right)
\eeq
If one applies this unitary transformation, it can be seen that
$M$ can be put in the real form:
\beq
\left(
\begin{tabular}{cc|c}
$Re(\alpha + \beta)$ & $Im(\alpha - \beta)$ & $\sqrt{2}Re(\gamma)$ \\
$-Im(\alpha + \beta)$ & $Re(\alpha - \beta)$ & $-\sqrt{2}Im(\gamma)$ \\
\hline $0$ & $0$ & $1$
\end{tabular}
\right) = \left(
\begin{tabular}{c|c}
$A$ & $\Gamma$ \\
\hline $0$ & $1$
\end{tabular}
\right)
\eeq
where $A$ and $\Gamma$ matrices are defined as in
\eqref{M-matrix}, and $Re$ and $Im$ denote the hermitian and
anti-hermitian parts.

Using this form it is not too hard to show that for \FIO,
the transformation relations \eqref{rel1fer} and \eqref{rel2fer}
together transform into the single equation:
\beq
\{\Gamma_i, \Gamma_j\} = \delta_{ij} - A_{ik} A_{jk} \quad,\quad i, j = 1, 2, \ldots , 2d .
\label{relcombined}
\eeq
By extending the range of the indices in this relation to
odd-dimensions it is possible
to define $FIO(2d + 1, \IR)$, the fermionic inhomogeneous
orthogonal algebra of odd dimension.

In order to show the validity of \eqref{relcombined}, one needs
to consider the three cases:
\[
\{\Gamma_i, \Gamma_j\} =
\begin{cases}
  2\{Re(\gamma_i), Re(\gamma_j) \} & \text{for $1 \leq i,j \leq d$} \\
  2\{Im(\gamma_i), Im(\gamma_j) \} & \text{for $d + 1 \leq i,j \leq 2d$} \\
  - 2\{Re(\gamma_i), Im(\gamma_j) \} & \text{for $1 \leq i \leq d$ and $d + 1 \leq j \leq 2d$} \\
\end{cases}
\]
For the case when $1 \leq i,j \leq d$, the above form becomes:
\beq \label{Gamma-case1}
\begin{split}
\{\Gamma_i, \Gamma_j\} &= 2\{Re(\gamma_i), Re(\gamma_j) \} \\
&= \frac12 \left[\{\gamma_i,\gamma_j\} + \{\gamma_i,\gamma^*_j\} + \{\gamma^*_i,\gamma_j\} + \{\gamma^*_i,\gamma^*_j\}\right] \\
&= \frac12 \left[(-\beta_{ik}\alpha_{jk}-\alpha_{ik}\beta_{jk}) + (\delta_{ij} - \alpha_{ik}\alpha^*_{jk} - \beta_{ik}\beta^*_{jk}) \right.\\
&  \phantom{{=} \frac12 \left[\right.}\left. + (\delta_{ij} - \alpha^*_{ik}\alpha_{jk} - \beta^*_{ik}\beta_{jk}) + (-\beta^*_{ik}\alpha^*_{jk}-\alpha^*_{ik}\beta^*_{jk})\right] \\
&= \frac12 \left[2\delta_{ij} - (\alpha_{ik} + \beta^*_{ik})(\alpha^*_{jk} + \beta_{jk}) - (\beta_{ik} + \alpha^*_{ik})(\alpha_{jk} + \beta_{jk})\right] \\
&= \delta_{ij} - \frac12 \left[(\alpha_{ik} + \beta^*_{ik})(\alpha^*_{jk} + \beta_{jk}) + (\beta_{ik} + \alpha^*_{ik})(\alpha_{jk} + \beta_{jk})\right]
\end{split}
\eeq
however, for this case, the form of $A_{ik}A_{jk}$ is:
\beq
\begin{split}
\sum^{2d}_{k = 1} A_{ik} A_{jk} &= \sum^{d}_{k = 1} A_{ik} A_{jk} + \sum^{2d}_{k = d + 1} A_{ik} A_{jk} \\
&= \sum^{d}_{k = 1} \left[Re(\alpha_{ik} + \beta_{ik})Re(\alpha_{jk} + \beta_{jk}) + Im(\alpha_{ik} - \beta_{ik})Im(\alpha_{jk} - \beta_{jk})\right] \\
&= \frac14 \left[(\alpha_{ik} + \beta_{ik} + \alpha^*_{ik} + \beta^*_{ik})(\alpha_{jk} + \beta_{jk} + \alpha^*_{jk} + \beta^*_{jk}) \right. \\
&  \phantom{{=} \frac14 \left[\right.}\left. -(\alpha_{ik} - \beta_{ik} - \alpha^*_{ik} + \beta^*_{ik})(\alpha_{jk} - \beta_{jk} - \alpha^*_{jk} + \beta^*_{jk}) \right] \\
&= \frac14 \left[(\alpha_{ik} + \beta^*_{ik})(\alpha^*_{jk} + \beta_{jk}) + (\alpha_{ik} + \beta^*_{ik})(\alpha_{jk} + \beta^*_{jk}) \right. \\
&  \phantom{{=} \frac14 \left[\right.}\left.+(\alpha^*_{ik} + \beta_{ik})(\alpha_{jk} + \beta_{jk}) + (\alpha^*_{ik} + \beta_{ik}) (\alpha^*_{jk} + \beta^*_{jk}) \right. \\
&  \phantom{{=} \frac14 \left[\right.}\left.+(\alpha_{ik} + \beta^*_{ik})(\alpha^*_{jk} + \beta_{jk}) - (\alpha_{ik} + \beta^*_{ik}) (\alpha_{jk} + \beta^*_{jk}) \right. \\
&  \phantom{{=} \frac14 \left[\right.}\left.+(\alpha^*_{ik} + \beta_{ik})(\alpha_{jk} + \beta_{jk}) - (\alpha^*_{ik} + \beta_{ik}) (\alpha^*_{jk} + \beta^*_{jk}) \right] \\
&= \frac12 \left[(\alpha_{ik} + \beta^*_{ik})(\alpha^*_{jk} + \beta_{jk}) + (\alpha^*_{ik} + \beta_{ik})(\alpha_{jk} + \beta_{jk}) \right]
\end{split}
\eeq
and using this in equation \eqref{Gamma-case1} gives us:
\beq
\begin{split}
\{\Gamma_i, \Gamma_j\} &= \delta_{ij} - \frac12 \left[(\alpha_{ik} + \beta^*_{ik})(\alpha^*_{jk} + \beta_{jk}) + (\beta_{ik} + \alpha^*_{ik})(\alpha_{jk} + \beta_{jk})\right] \\
&= \delta_{ij} - A_{ik}A_{jk} \quad \quad \text{for $1 \leq i,j \leq d$}
\end{split}
\eeq
For the case when $d + 1 \leq i,j \leq 2d$, the above form becomes:
\beq
\begin{split} \label{Gamma-case2}
\{\Gamma_i, \Gamma_j\} &= 2\{Im(\gamma_i), Im(\gamma_j) \} \\
&= - \frac12 \left[\{\gamma_i,\gamma_j\} - \{\gamma_i,\gamma^*_j\} - \{\gamma^*_i,\gamma_j\} + \{\gamma^*_i,\gamma^*_j\}\right] \\
&= - \frac12 \left[(-\beta_{ik}\alpha_{jk}-\alpha_{ik}\beta_{jk}) - (\delta_{ij} - \alpha_{ik}\alpha^*_{jk} - \beta_{ik}\beta^*_{jk}) \right.\\
&    \phantom{{=} - \frac12 \left[\right.}\left. - (\delta_{ij} - \alpha^*_{ik}\alpha_{jk} - \beta^*_{ik}\beta_{jk}) + (-\beta^*_{ik}\alpha^*_{jk}-\alpha^*_{ik}\beta^*_{jk})\right] \\
&= - \frac12 \left[-2\delta_{ij} + (\alpha^*_{ik} - \beta_{ik})(\alpha_{jk} - \beta^*_{jk}) + (\beta^*_{ik} - \alpha_{ik})(\beta_{jk} - \alpha^*_{jk})\right] \\
&= \delta_{ij} - \frac12 \left[(\alpha^*_{ik} - \beta_{ik})(\alpha_{jk} - \beta^*_{jk}) + (\alpha_{ik} - \beta^*_{ik})(\alpha^*_{jk} - \beta_{jk})\right]
\end{split}
\eeq
however, for this case, the form of $A_{ik}A_{jk}$ is:
\beq
\begin{split}
\sum^{2d}_{k = 1} A_{ik} A_{jk} &= \sum^{d}_{k = 1} A_{ik} A_{jk} + \sum^{2d}_{k = d + 1} A_{ik} A_{jk} \\
&= \sum^{d}_{k = 1} \left[Im(\alpha_{ik} + \beta_{ik})Im(\alpha_{jk} + \beta_{jk}) + Re(\alpha_{ik} - \beta_{ik})Re(\alpha_{jk} - \beta_{jk})\right] \\
&= \frac14 \left[-(\alpha_{ik} + \beta_{ik} - \alpha^*_{ik} - \beta^*_{ik})(\alpha_{jk} + \beta_{jk} - \alpha^*_{jk} - \beta^*_{jk}) \right.\\
&  \phantom{{=} \frac14 \left[\right.}\left.+(\alpha_{ik} - \beta_{ik} + \alpha^*_{ik} - \beta^*_{ik})(\alpha_{jk} - \beta_{jk} + \alpha^*_{jk} - \beta^*_{jk}) \right] \\
&= \frac14 \left[(\alpha^*_{ik} - \beta_{ik})(\alpha_{jk} - \beta^*_{jk}) - (\alpha^*_{ik} - \beta_{ik})(\alpha^*_{jk} - \beta_{jk}) \right. \\
&  \phantom{{=} \frac14 \left[\right.}\left.+(\alpha_{ik} - \beta^*_{ik})(\alpha^*_{jk} - \beta_{jk}) - (\alpha_{ik} - \beta^*_{ik})(\alpha_{jk} - \beta^*_{jk}) \right. \\
&  \phantom{{=} \frac14 \left[\right.}\left.+(\alpha^*_{ik} - \beta_{ik})(\alpha_{jk} - \beta^*_{jk}) + (\alpha^*_{ik} - \beta_{ik})(\alpha^*_{jk} - \beta_{jk}) \right. \\
&  \phantom{{=} \frac14 \left[\right.}\left.+(\alpha_{ik} - \beta^*_{ik})(\alpha^*_{jk} - \beta_{jk}) + (\alpha_{ik} - \beta^*_{ik})(\alpha_{jk} - \beta^*_{jk}) \right] \\
&= \frac12 \left[(\alpha^*_{ik} - \beta_{ik})(\alpha_{jk} - \beta^*_{jk}) + (\alpha_{ik} - \beta^*_{ik})(\alpha^*_{jk} - \beta_{jk})\right]
\end{split}
\eeq
and using this in equation \eqref{Gamma-case2} gives us:
\beq
\begin{split}
\{\Gamma_i, \Gamma_j\} &= \delta_{ij} - \frac12 \left[(\alpha^*_{ik} - \beta_{ik})(\alpha_{jk} - \beta^*_{jk}) + (\alpha_{ik} - \beta^*_{ik})(\alpha^*_{jk} - \beta_{jk})\right] \\
&= \delta_{ij} - A_{ik}A_{jk} \quad \quad \text{for $d + 1 \leq i,j \leq 2d$}
\end{split}
\eeq
Finally, for the case when $1 \leq i \leq d$ and $d + 1 \leq j \leq 2d$, the above form becomes:
\beq
\begin{split} \label{Gamma-case3}
\{\Gamma_i, \Gamma_j\} &= - 2\{Re(\gamma_i), Im(\gamma_j) \} \\
&= - \frac1{2i} \left[\{\gamma_i,\gamma_j\} - \{\gamma_i,\gamma^*_j\} + \{\gamma^*_i,\gamma_j\} - \{\gamma^*_i,\gamma^*_j\}\right] \\
&= - \frac1{2i} \left[(-\beta_{ik}\alpha_{jk}-\alpha_{ik}\beta_{jk}) - (\delta_{ij} - \alpha_{ik}\alpha^*_{jk} - \beta_{ik}\beta^*_{jk}) \right.\\
&    \phantom{{=} - \frac1{2i} \left[\right.}\left. + (\delta_{ij} - \alpha^*_{ik}\alpha_{jk} - \beta^*_{ik}\beta_{jk}) - (-\beta^*_{ik}\alpha^*_{jk}-\alpha^*_{ik}\beta^*_{jk})\right] \\
&= - \frac1{2i} \left[(\alpha_{ik} + \beta^*_{ik})(\alpha^*_{jk} - \beta_{jk}) - (\alpha^*_{ik} + \beta_{ik})(\alpha_{jk} - \beta^*_{jk})\right]
\end{split}
\eeq
however, for this case, the form of $A_{ik}A_{jk}$ is:
\beq
\begin{split}
\sum^{2d}_{k = 1} A_{ik} A_{jk} &= \sum^{d}_{k = 1} A_{ik} A_{jk} + \sum^{2d}_{k = d + 1} A_{ik} A_{jk} \\
&= \sum^{d}_{k = 1} \left[-Re(\alpha_{ik} + \beta_{ik})Im(\alpha_{jk} + \beta_{jk}) + Im(\alpha_{ik} - \beta_{ik})Re(\alpha_{jk} - \beta_{jk})\right] \\
&= \frac1{4i} \left[-(\alpha_{ik} + \beta_{ik} + \alpha^*_{ik} + \beta^*_{ik})(\alpha_{jk} + \beta_{jk} - \alpha^*_{jk} - \beta^*_{jk}) \right.\\
&  \phantom{{=} \frac1{4i} \left[\right.}\left.+(\alpha_{ik} - \beta_{ik} - \alpha^*_{ik} + \beta^*_{ik})(\alpha_{jk} - \beta_{jk} + \alpha^*_{jk} - \beta^*_{jk}) \right] \\
&= \frac1{4i} \left[(\alpha_{ik} + \beta^*_{ik})(\alpha^*_{jk} - \beta_{jk}) - (\alpha_{ik} + \beta^*_{ik})(\alpha_{jk} - \beta^*_{jk}) \right. \\
&  \phantom{{=} \frac1{4i} \left[\right.}\left.-(\alpha^*_{ik} + \beta_{ik})(\alpha_{jk} - \beta^*_{jk}) + (\alpha^*_{ik} + \beta_{ik})(\alpha^*_{jk} - \beta_{jk}) \right. \\
&  \phantom{{=} \frac1{4i} \left[\right.}\left.+(\alpha_{ik} + \beta^*_{ik})(\alpha^*_{jk} - \beta_{jk}) + (\alpha_{ik} + \beta^*_{ik})(\alpha_{jk} - \beta^*_{jk}) \right. \\
&  \phantom{{=} \frac1{4i} \left[\right.}\left.-(\alpha^*_{ik} + \beta_{ik})(\alpha_{jk} - \beta^*_{jk}) - (\alpha^*_{ik} + \beta_{ik})(\alpha^*_{jk} - \beta_{jk}) \right] \\
&= \frac12    \left[(\alpha_{ik} + \beta^*_{ik})(\alpha^*_{jk} - \beta_{jk}) - (\alpha^*_{ik} + \beta_{ik})(\alpha_{jk} - \beta^*_{jk})\right]
\end{split}
\eeq
and using this in equation \eqref{Gamma-case3} gives us:
\beq
\begin{split}
\{\Gamma_i, \Gamma_j\} &= - \frac1{2i} \left[(\alpha_{ik} + \beta^*_{ik})(\alpha^*_{jk} - \beta_{jk}) - (\alpha^*_{ik} + \beta_{ik})(\alpha_{jk} - \beta^*_{jk})\right] \\
&= - A_{ik}A_{jk} \quad \quad \text{for $1 \leq i \leq d$ and $d + 1 \leq j \leq 2d$}
\end{split}
\eeq
which completes the derivation of equation \eqref{relcombined}.

Similar to the analysis that went into finding the
sub(quantum)groups of \FIO, we can also investigate the
sub(quantum)groups of $FIO(2d + 1, \IR)$. The sub(quantum)group
relations in this case, however, are more restricted owing to the
fact that the algebra is not described anymore by submatrices of
the $A$ matrix but is rather described by the whole matrix itself.
Thus, we cannot set $\alpha$ or $\beta$ to zero on their own, we
can only restrict the algebra by setting the whole of $A$ to zero.
Thus the resulting sub(quantum)algebra relations are:
\begin{subequations}
\bea
\delta_{ij} - A_{ik}A_{jk} = 0 \label{subaferodd} \\
\Gamma_i = 0 \label{subbferodd} \\
A_{ij} = 0 \label{subcferodd}
\eea
\end{subequations}
which, through a similar analysis to the even dimensional case,
gives us the following sub(quantum)group diagram:
\[
\begin{CD}
{FIO(2d + 1, \IR)} @>\eqref{subaferodd}>> {GrIO(2d + 1, \IR)} @>\eqref{subbferodd}>> {O(2d + 1, \IR)} \\
@V\eqref{subcferodd}VV\\
{Cliff(2d + 1)}
\end{CD}
\]


\chapter{CONCLUSIONS}

%% TODO: Write this

%% Conclusions from ACSA
\begin{comment}
\section{Conclusions}

The Anticommutator Spin Algebra, which is a special Jordan
algebra, has many implications. The first of these is the fact
that this algebra is a consistent fermionic algebra which is not a
superalgebra.
% The
%bosonic and fermionic generators $B, F$ in a superalgebra obey:
%\bea
%[B,B] & = & B \nonumber \\ \nonumber
%[F,B] & = & F \\ \nonumber
%\{F,F\} & = & B \nonumber
%\eea
%whereas in ACSA this relation is of the form:
%\[
%\{F,F\} = F
%\]
%which shows how different it is from a superalgebra.
For possible physical applications the right-hand side of the
defining relations (\ref{eqn:defrel1}-\ref{eqn:defrel3}) must also
be supplied with an $\hbar$. In a superalgebra approach where the
$J_i$ are regarded as odd operators, the $\hbar$ on the right-hand
side should also be regarded as an operator anticommuting with the
$J_i$. These models \cite{leites,batalin} result from the
quantization of the odd Poisson bracket. In our approach however,
the concept of grading and therefore an underlying Poisson bracket
formalism does not exist. In particular, there is no Jacobi
identity. Nevertheless, the associative algebra we consider is
consistent with quantum mechanics where physical observables
correspond to hermitian operators and their eigenvalues to
possible results of physical measurement of these observables. It
is for this reason that ACSA suggests a new kind of statistics
which, we believe, will be useful in physics.

The second implication is the important role of quantum groups in
mathematical physics. As we have shown in this paper, the
invariance group of ACSA turns out to be a quantum group. Given
the fact that ACSA is very similar to normal spin algebra and that
the invariance group of spin algebra plays an important role in
physics, the invariance quantum group of ACSA, $SO_{q=-1}(3)$,
becomes a prime example of how central quantum groups have become
in mathematical physics. It is also interesting to note that more
algebras like ACSA can be constructed where the commutators of the
original Lie algebra are turned into anticommutators and that such
algebras might also have invariance quantum groups that is the
same as the invariance group of the original Lie algebra with
$q=-1$. This possibility is open to investigation in a more
general framework.
\end{comment}

% From BISP and FIO paper
\begin{comment}
\section{Discussion}
As we have shown, the boson and fermion algebras can be obtained
as a limit of the inhomogeneous quantum groups \BISp and \FIO. We
can understand why these boson and fermion algebras are not
quantum groups from this construction, since in this limit the
quantum group becomes singular and the antipode does not exist.
Thus we can consider these quantum groups as deformations with a
Hopf algebra structure of their respective particle algebras. This
construction is similar to $q$-deforming the bosonic oscillator to
obtain Pusz-Woronowicz \cite{puszwor} oscillators and then
constructing the $q$-deformed quantum unitary groups as their left
modules. Similarly, in that construction, the $q$-deformed
oscillator can be reobtained as a limit of these $q$-deformed
quantum unitary groups. However, unlike that construction the
quantum groups presented in this paper are inhomogeneous quantum
groups.

Finally, we would like to remark that the widely used field
theoretical generalization achieved by extending the discrete
indices $i, j, k$ to continuous variables together with a
replacement of the Kr\"onecker deltas to Dirac delta functions is
also applicable to the quantum groups we have presented. In this
respect, these quantum groups are also different from the
Pusz-Woronowicz oscillators which cannot be extended to continuous
indices.

We believe that the establishment of these and similar quantum
groups in field theory will be helpful in generalizing methods of
quantization. These approaches will yield a more consistent
approach to interacting field theory and will be the subject of
further investigations.

\end{comment}



%\appendix      % Starts appendices
%\chapter{A FERMION LIKE ALGEBRA}
%%% TODO: Write this


\begin{thebibliography}{9}

\bibitem{sweedler}
M.E. Sweedler, \textit{Hopf Algebras}, Benjamin (1969)

\bibitem{krob-leclerc}
D. Krob, B. Leclerc, \textit{Commun. Math. Phys.}, {\bf 169} (1995) 1-23

%% References for Chapter 2

\bibitem{sklyanin}
E.K. Sklyanin, \textit{Funct. Anal. Appl.}, {\bf 16} (1982) 263

\bibitem{frt}
L.D. Fadeev, N.Y. Reshetikhin, L.A. Takhtajan,
\textit{Quantization of Lie Groups and Lie Algebras}, preprint
LOMI, {\bf E-14-87} (1987)

\bibitem{drinfeld}
V.G. Drinfeld, \textit{Quantum Groups}, Proc. Int. Congr. Math.,
Berkeley 1, 798 (1986)

\bibitem{woronowicz}
S.L. Woronowicz, \textit{Commun. Math. Phys.}, {\bf 111} (1987)
613

\bibitem{manin}
Yu I. Manin, \textit{Quantum Groups and Non-commutative Geometry}
(CRM, Montreal University Press, 1988)

\bibitem{jx}
S. Jing, J. Xu, \textit{J. Phys. A: Math. Gen.}, {\bf 24} (1991)
L891

\bibitem{xh}
W.Y. Chan, C.L. Ho, \textit{J. Phys. A: Math. Gen.}, {\bf 26}
(1993) 4827

\bibitem{sm}
A. Solomon, R. McDermott, \textit{J. Phys. A: Math. Gen.}, {\bf
27} (1994) 2619

\bibitem{chung}
W.S. Chung, \textit{Phys. Lett. A}, {\bf 259} (1999) 437

\bibitem{nt}
T Nassar and O Tirkkonen, \textit{J. Phys. A: Math. Gen.}, {\bf31}
(1998) 9983

\bibitem{gppr}
H Grosse, S Pallua, P Prester and E Raschhofer, \textit{J. Phys.
A: Math. Gen.}, {\bf27} (1994) 4761

\bibitem{bnnpsw} J. Balog, M. Niedermaier, F. Niedermayer, A. Patrascioiu, E. Seiler and P. Weisz,
\textit{Nuclear Physics B}, {\bf618} (2001) 315

\bibitem{ubriaco}
M. R. Ubriaco, \textit{Phys. Rev. E}, {\bf58} (1998) 4191

\bibitem{leites}
D. A. Leites, I. M. Shchepochkina, \textit{Theor. Math. Phys.},
{\bf 3}, (2001) 281

\bibitem{batalin}
I. Batalin, I. Tyutin, �Generalized field�antifield formalism,�,
Topics in Statistical and Theoretical Physics (R. Dobrushin et
al., eds.) (Trans. Am. Math. Soc., Ser. 2, Vol. 177), Am. Math.
Soc., Providence, RI (1996), 23

%% References for Chapter 3

\bibitem{fadeev}
V. G. Drinfeld, \textit{Proc. Int. Congr. Math.}, Berkeley {\bf
1}, 798 (1986)

M. Jimbo, \textit{Lett. Math. Phys.} {\bf 11}, 247 (1986)

S. L. Woronowicz, \textit{Commun. Math. Phys.} {\bf 111}, 613
(1987)

L.D. Fadeev, N.Y. Reshetikhin, L.A. Takhtajan, "Quantization of
Lie Groups and Lie Algebras", \textit{Leningrad Math. J.} {\bf 1},
193 (1987)

\bibitem{sww}
M. Schlieker, W. Weich and R. Weixler, \textit{Z. Phys. C} {\bf
53}, 79 (1992)

\bibitem{pw}
P. Podles and S. L. Woronowicz, \textit{Commun. Math. Phys.} {\bf
185}, 325 (1997)

\bibitem{unruh}
W. G. Unruh, \textit{Phys. Rev. D} {\bf14}, 870 (1976)

\bibitem{hawking}
S. W. Hawking, \textit{Phys. Rev. D} {\bf 14}, 2460 (1976)

\bibitem{agy}
M. Arik, S. Gun, A. Yildiz, \textit{Eur. Phys. J. C} {\bf 27}, 453
(2003)

\bibitem{ab}
M. Arik, A. Baykal, \textit{J. Math. Phys.} {\bf 45}, 4207 (2004)

\bibitem{inonu}
E.Inonu and E.P.Wigner, \textit{Proc. Natl. Acad. Sci. U.S.A.}
{\bf 39}, 510 (1953)

\bibitem{majid}
S. Majid, "Braided groups and algebraic quantum field theories",
\textit{Letters in Math. Phys.} {\bf 22}, 167 (1991)

\bibitem{puszwor}
W. Pusz and S.L. Woronowicz, \textit{Rep. Math. Phys.} {\bf 27},
231 (1989)

M. Arik, \textit{Z. Phys. C} {\bf 51}, 627 (1991)

%% References for Chapter 4



\end{thebibliography}
%
% You can also try harvardbibliography, bibnotcited and
% harvardbibnotcited environments for other types of bibliographical
% lists.
%
\end{document}
