%
% This is a template for the fbe_tez.sty, v4.1 (and maybe later).
%
%
\documentclass[12pt]{report}

% Main Thesis Layout style
\usepackage{fbe_tez}

% The following are needed for chapter3
\usepackage{amscd}
\usepackage{amsmath}
\usepackage{amssymb}

% For multi-line comments
\usepackage{verbatim}

% For commutative diagrams
\usepackage{xypic}
\xyoption{all}

\newcommand{\beq}{\begin{equation}}
\newcommand{\eeq}{\end{equation}}
\newcommand{\bea}{\begin{eqnarray}}
\newcommand{\eea}{\end{eqnarray}}
\newcommand{\bean}{\begin{eqnarray*}}
\newcommand{\eean}{\end{eqnarray*}}

\newcommand{\adag}{a^{\dagger}}
\newcommand{\bdag}{b^{\dagger}}
\newcommand{\ket}[1]{\mid #1\;\rangle}
\newcommand{\bra}[1]{\langle\; #1\mid}
\newcommand{\braket}[2]{\langle\; #1\mid #2\;\rangle}
\newcommand{\delfn}[1]{\delta(#1)}
\newcommand{\kro}[1]{\delta_{#1}}
\newcommand{\anti}[2]{\{#1, #2\}}


\def\IC{\mathbb{C}}
\def\IR{\mathbb{R}}
\def\I1{\mathbb{I}}
\def\II{\mathbb{J}}
\def\FIO{$FIO(2d, \IR)\;$}
\def\BISp{$BISp(2d, \IR)\;$}

%
% LaTeX 2.09 users: Replace the above two lines with the following line
%\documentstyle[12pt,fbe_tez]{report}
%
% Declarations
%

\title{QUANTUM GROUP STRUCTURES ASSOCIATED WITH INVARIANCES OF SOME PHYSICAL ALGEBRAS}

\turkcebaslik{BAZI F\.{I}Z\.{I}KSEL CEB\.{I}RLER\.{I}N
  DE\~G\.{I}\c{S}MEZL\.{I}\~G\.{I} \.{I}LE \.{I}LG\.{I}L\.{I} KUANTUM GRUP YAPILARI}

\degree{
B.S. in Physics, Bo\~ gazi\c ci University, 1997 \\
B.S. in Mathematics, Bo\~ gazi\c ci University, 1997
}
\author{Ufuk Kayserilio\~ glu}

\program{Physics}
\subyear{2005}

%
% Declare relevant examiners
%
\supervisor{Prof. Dr. Metin Ar\i k}
%\cosuperi{Title and Name of Cosupervisor I}
%\cosuperii{Title and Name of Cosupervisor II}
\examineri{Prof. Dr. \"Omer Faruk Day{\i}}
\examinerii{Prof. Dr. Fahr\"{u}nisa Neyzi}
\examineriii{Prof. Dr. Cihan Sa\c{c}l{\i}o\~{g}lu}
\examineriv{Do\c{c}. Dr. Teoman Turgut}
%\examinerv{Title and Name of Examiner}
\dateofapproval{22.07.2005}

\begin{document}
\pagenumbering{roman}

\makephdtitle      % For Ph.D. theses

\makeapprovalpage

\begin{acknowledgements}
This work is dedicated to my loving wife \textit{Emi} for all her support and understanding;
to my wise and patient mentor \textit{Metin Ar{\i}k} for teaching me a lot of what I know
and to \textit{my parents} who have shown me how to think scientifically about nature.
\end{acknowledgements}

\begin{abstract}
In this study, the anticommuting spin algebra is introduced and it is
shown to be invariant under the action of the quantum group $SO_{q = -1}(3)$.
Furthermore, its representations and Hopf algebra structure are studied and
found to be closely resemble the similar results for the angular momentum
algebra. The invariance properties of the bosonic and fermionic oscillator
algebras under inhomogeneous transformations are also studied. The bosonic
inhomogeneous symplectic group, \BISp, and the fermionic
inhomogeneous orthogonal group, \FIO, are defined as the inhomogeneous invariance
quantum groups of these algebras. The sub(quantum)groups and contractions of these
quantum groups are studied as a source for new quantum groups. Finally, the
fermionic inhomogeneous orthogonal quantum group is defined for odd number of
dimensions and its sub(quantum)groups and contractions are studied.
inho
\end{abstract}
%
% The usage of the "foreword" and "preface" environments are similar
% the "abstract" and "acknowledgements". See FBE manual for the
% correct order of these pages in the thesis.
%
\begin{ozet}
Bu \c{c}al{\i}\c{s}mada, ters-de\~{g}i\c{s}meli spin cebri tan{\i}mlanm{\i}\c{s} ve bu cebrin $SO_{q = -1}(3)$
kuantum grubu alt{\i}nda de\~{g}i\c{s}mezli\~{g}i g\"{o}sterilmi\c{s}tir. Bunun \"{o}tesinde, bu cebrin temsilleri
ve Hopf cebir yap{\i}s{\i} incelenmi\c{s} ve a\c{c}{\i}sal momentum cebri i\c{c}in bulunmu\c{s} olanlara
\c{c}ok benzer yap{\i}lara var{\i}lm{\i}\c{s}t{\i}r. Bozonik ve fermiyonik osilat\"{o}r cebirlerinin homojen
olmayan de\~{g}i\c{s}mezlik \"{o}zellikleri incelenmi\c{s} ve bunlar{\i}n sonucunda bozonik inhomojen
simplektik group, \BISp, ve fermiyonik inhomojen ortogonal group, \FIO,
de\~{g}i\c{s}mezlik kuantum gruplar{\i} olarak tan{\i}mlanm{\i}\c{s}t{\i}r. Bu kuantum gruplar{\i}n{\i}n alt(kuantum)gruplar{\i}
ve b\"{u}z\"{u}lmeleri yeni kuantum grup kaynaklar{\i} olarak incelenmi\c{s}tir. Son olarak, fermiyonik
inhomojen ortogonal kuantum grup tek boyutlarda da tan{\i}mlanm{\i}\c{s} ve bu kuantum grubunun da
alt(kuantum)gruplar{\i} ve b\"{u}z\"{u}lmeleri incelenmi\c{s}tir.
\end{ozet}

\tableofcontents
\listoffigures
%\listoftables

\begin{symabbreviations}
% The title will be typeset as "LIST OF SYMBOLS/ABBREVIATIONS".
% If you prefer it as "LIST OF SYMBOLS" or "LIST OF ABBREVIATIONS"
% use the environments "symbols" or "abbreviations", respectively.
%
% Use a separate \sym command for each symbols definition.
% First Latin symbols in alphabetical order
\sym{$a$}{Annihilation operator}
\sym{$a^\dagger$}{Creation operator}
\sym{$id$}{Identity map}
\sym{$m$}{Multiplication map}
\sym{$N$}{Number operator}
\sym{$S$}{Coinverse map}

\sym{}{}       % Separating line between symbols and abbreviations

% Then Greek symbols in alphabetical order
\sym{$\Delta$}{Coproduct map}
\sym{$\epsilon$}{Counit map}
\sym{$\eta$}{Unit map}
\sym{$\tau$}{Permutation map}
\sym{$\sigma_i$}{Pauli sigma matrices}
\sym{}{}       % Separating line between symbols and abbreviations
\sym{$\otimes$}{Tensor product}
\sym{$\dot{\otimes}$}{Matrix multiplication with tensor product}
\sym{$\circ$}{Function composition}

\sym{}{}       % Separating line between symbols and abbreviations

\sym{ACSA}{The Anticommuting Spin Algebra}
\sym{BISp}{The Bosonic Inhomogeneous Symplectic Quantum Group}
\sym{FIO}{The Fermionic Inhomogeneous Orthogonal Quantum Group}
\end{symabbreviations}

\chapter{INTRODUCTION}
\pagenumbering{arabic}

The concept of bosonic and fermionic particles is one of the most important concepts
in modern quantum physics. The behavior of large scale matter, from chemical properties
of elements to superconductivity and superfluidity can mostly be understood by
referring to the fermionic or bosonic nature of the quantum mechanical particles
involved in such processes. It is for this reason that understanding the symmetry
properties of these phenomena and, motivated by their importance, trying to find
other behavior that mimic them is very meaningful.

Furthermore, while bosonic behavior has a classical counterpart, the concept of a fermionic
particle is one that can only exist in the quantum domain. This fact makes the study
of such behavior even more important. However, what could be more interesting is the
study of other such constructs that cannot have a classical counterpart. These constructs
would thus belong solely in the quantum domain and could help us understand phenomena that
are strictly quantum mechanical in nature.

There is a strong relation between the spin properties of a particle and the particle being
a boson or a fermion. In fact, it is a proven fact of quantum physics that integer spin
particles are bosons and half-integer spin particles are fermions. This is most often referred
to as the "spin-statistics theorem" in quantum mechanics and is a very interesting fact since
it implies a relationship between two concepts that seems to be totally unrelated. This
strong relation between the bosonic/fermionic nature of a particle and its spin make the
angular momentum algebra also very central in quantum physics.

Before we start investigating such matters, it would be apt to give an overview of the
state of bosons and fermions and the angular momentum algebra as it has been studied up to
now.

When the harmonic oscillator is studied in a quantum mechanical manner, one arrives
at the relation:
\beq
a \adag - \adag a = 1
\eeq
for the system the Hamiltonian of which is given by $\hbar \omega (a \adag + \adag a)$. The spectrum
of this Hamiltonian, which in turn gives us the allowable energy levels of the quantum
harmonic oscillator, can be obtained easily by introducing the hermitian operator
$N = \adag a$ which has the following relations with $a$ and $\adag$:
\bea
 [ N , \adag ] &=& \adag \\ [0pt]
 [ N , a ]     &=& a
\eea
where $[\;,\; ]$ denotes the usual commutator. By observing the fact that the
Hamiltonian is nothing but $N + \frac12$, one can see that one can get the energy levels
states as eigenvectors $\ket{n}$, of the operator $N$. The action of $\adag$ and $a$ on such an
eigenvector $\ket{n}$ can be found to be:
\bea
\adag \ket{n} &=& \sqrt{n + 1} \ket{n + 1} \\
a \ket{n} &=& \sqrt{n} \ket{n - 1}
\eea
which in turn implies that the values of $n$, the eigenvalues of the operator $N$,
begin from $0$ and increase by $1$ every time $\adag$ is applied on the relevant
state. As a result of this study one finds that the energy levels of the quantum
harmonic oscillator is given by $\hbar \omega (n + \frac12)$ and that the operators
$\adag$ and $a$ are operators that create and destroy, respectively,
one quanta of energy. For this reason they are usually called creation and annihilation operators.

Even though this operator algebra seems to only describe the quantum harmonic oscillator,
when one studies quantum field theory, this algebra comes up as the algebra of
of the Fourier coefficients of the field operator of a bosonic particle. In that setting,
the operators $\adag_p$ and $a_p$, which now carry a continuous momentum index, are
interpreted as the operators that create and destroy, respectively, one bosonic
particle of such a field with momentum $p$.

For fermionic particle fields as similar treatment again yields creation and annihilation
operators as the Fourier coefficients of the field operator; however, the algebra obeyed
by these operators is not the same as the algebra of the quantum harmonic oscillator. Instead,
the defining relations of this algebra are:
\bea
a \adag + \adag a &=& 1 \\
a a &=& 0
\eea
which, through a similar treatment, yields the eigenvalues $0$ and $1$ for the operator
$N$.


\chapter{THE ANTICOMMUTING SPIN ALGEBRA}

%% Put the following text in the intro somewhere
\begin{comment}
The algebra of observables in quantum theory plays a fundamental
role. When classical systems are quantized, their classical
symmetry algebra acting on a set of physical observables, in
simplest examples, remains the same. For some completely
integrable non-linear models, consistent quantization requires
that the classical symmetry group be replaced by a quantum group
\cite{frt,drinfeld,woronowicz,manin} via a deformation parameter
$q = 1 + O(\hbar)$. In recent years quantum groups involving
fermions have received widespread attention. These include
deformed fermion algebras \cite{jx,xh,sm,chung}, spin chains
\cite{nt,gppr,bnnpsw} and Fermi gases \cite{ubriaco}. At the same
time, some quantum systems, most notably fermionic quantum systems
do not have any classical analogues. Nevertheless, fermions are
perhaps the most important sector of quantum phenomena. Motivated
by these considerations, we define a fermionic version of the
angular momentum algebra by the relations
\end{comment}
%%


\section{Defining relations}

\bea
\anti{J_1}{J_2} & = & J_3 \label{eqn:defrel1} \\
\anti{J_2}{J_3} & = & J_1 \label{eqn:defrel2} \\
\anti{J_3}{J_1} & = & J_2 \label{eqn:defrel3}
\eea
where $J_1$, $J_2$, $J_3$ are hermitian generators of the algebra. We will name this algebra ACSA, the anticommutator spin algebra. In these expressions the curly bracket denotes the anticommutator \beq
\anti{A}{B} \equiv AB + BA \eeq so (\ref{eqn:defrel1}-\ref{eqn:defrel3}) should be taken as the
definition of an associative algebra. This proposed algebra does
not fall into the category of superalgebras in the sense of
Berezin-Kac axioms. In particular, the algebra is consistent
without grading and there are no (graded) Jacobi relations. As it
is defined this algebra falls into the category of a
(non-exceptional) Jordan algebra where the Jordan product is
defined by: \beq A \circ B \equiv \frac12 (AB + BA) \quad . \eeq A
formal Jordan algebra, in addition to a commutative Jordan
product, also satisfies $A^2\circ(B\circ A) = (A^2\circ B)\circ
A$. When the Jordan product is given in terms of an anticommutator
this relation is automatically satisfied. Just as a Lie algebra
where the Lie bracket as defined by the commutator leads to an
enveloping associative algebra, a Jordan algebra defined in terms
of the above product leads to an enveloping associative algebra
which we consider as an algebra of observables.

The physical properties of this system turn out to be similar to
those of the angular momentum algebra yet exhibit remarkable
differences. Since the angular momentum algebra is used to
describe various internal symmetries, ACSA could be relevant in
describing those symmetries.

In section 2 we will show that ACSA is invariant under the action
of the quantum group $SO_q(3)$ with $q=-1$. Here, $SO_q(3)$ is
defined as the quantum subgroup of $SU_q(3)$ where each of the
(non-commuting) matrix elements of the $3$x$3$ matrix is
hermitian. We note that this defines a quantum group only for
$q=\pm1$. For $q=1$ one has the real orthogonal group $SO(3)$.

In section 3, we will construct all representations of ACSA and
show that the representations can be labelled by a quantum number
$j$ corresponding to the eigenvalue of $J_3$ whose absolute value
is maximum. For integer $j$, spectrum of $J_3$ is given by $j,
j-1, \ldots, -j$ whereas for half-integer $j$ there are two
representations. These two representations are such that for $j =
2k\pm\frac12$ spectrum of $J_3$ is respectively given by $j, j-2,
\ldots, \pm\frac12$ and $-j, j+2, \ldots, \mp\frac12$. Section 4
is reserved for conclusions and discussion.

\section{The invariance quantum group $SO_{q = -1}(3)$}

In order to find the invariance quantum group of this algebra, we
transform the generators $J_i$ to $J'_i$ by:
\beq \label{trans}
J'_i = \sum_j\alpha_{ij} J_j \quad .
\eeq
The matrix elements
$\alpha_{ij}$ are hermitian since $J_i$'s are hermitian and they
commute with $J_i$'s but are not assumed to commute with each other. For the
transformed operators to obey the original relations, there should
exist some conditions on the $\alpha$'s which define the
invariance quantum group of the algebra. It is very convenient at
this moment to switch to an index notation that encompasses all
three defining relations of the algebra in one index equation. For
the angular momentum algebra this is possible by defining the
totally anti-symmetric rank 3 pseudo-tensor $\epsilon_{ijk}$. A
similar object for ACSA which we will call the fermionic
Levi-Civita tensor, $u_{ijk}$, is defined as:
\beq
u_{ijk} =
  \begin{cases}
    1, & \text{for $i \neq j \neq k \neq i$,} \\
    0, & \text{otherwise.}
  \end{cases}
\eeq
Thus the defining relations (\ref{eqn:defrel1}-\ref{eqn:defrel3}) become:
\beq
\anti{J_i}{J_j} = \sum_k  u_{ijk}\,J_k + 2\delta_{ij}\,J_i^2
\eeq
The second term on the right is needed since when $i = j$ the left-hand side becomes $2J_i^2$.
Upon transformation (\ref{trans}) we require the algebra relations to remain invariant which means:
\beq
\anti{J'_i}{J'_j} = J'_k\quad\quad\text{for $i \neq j \neq k \neq i$}.
\eeq
However, substituting the transformation equations into the
left-hand side, we have:
\beq
\anti{J'_i}{J'_j} =
\sum_{k,\;m} \left(\alpha_{ik} \alpha_{jm} J_k J_m + \alpha_{jm} \alpha_{ik} J_m J_k \right)
\eeq
If one considers the quadratic forms in the universal enveloping algebra of ACSA, then
one can see that the symmetric part of these forms resolve to linear forms owing to
the defining relations of the algebra. Thus the independent quadratic forms in the
algebra are the antisymmetric forms, $[J_m, J_k]$ where $m \neq k$, and the square forms,
$J^2_k$. Using this observation we put the above relation in the form of a
linear sum over independent algebra elements:
\beq
\begin{split}
\anti{J'_i}{J'_j}
& = \sum_{n,\;m} \left(\alpha_{in} \alpha_{jm} J_n J_m + \alpha_{jm} \alpha_{in} J_m J_n \right)\\
& = \frac12 \sum_{n,\;m} \left(\alpha_{in} \alpha_{jm} (\anti{J_n}{J_m} + [J_n, J_m])
  + \alpha_{jm} \alpha_{in} (\anti{J_m}{J_n} + [J_m, J_n]) \right) \\
& = \frac12 \sum_{\substack{n,\;m \\n \neq m}} \left(\sum_{l}(\alpha_{in} \alpha_{jm} + \alpha_{jm} \alpha_{in}) u_{nml} J_l
    + (\alpha_{in} \alpha_{jm} - \alpha_{jm} \alpha_{in}) [J_n, J_m]\right) \\
& \quad + \sum_{n} \left(\alpha_{in} \alpha_{jn} + \alpha_{jn} \alpha_{in}\right) J^2_n
\end{split}
\eeq
which should be equal to $J'_k$ for $i \neq j \neq k \neq i$, which in turn gives:
\begin{gather}
\anti{J'_i}{J'_j} = J'_k    \\
\begin{split}
&\sum_{n} \left(\alpha_{in} \alpha_{jn} + \alpha_{jn} \alpha_{in}\right) J^2_n \\
& \quad + \frac12 \sum_{\substack{n,\;m \\n \neq m}} \left(\sum_{l}(\alpha_{in} \alpha_{jm} + \alpha_{jm} \alpha_{in}) u_{nml} J_l
    + (\alpha_{in} \alpha_{jm} - \alpha_{jm} \alpha_{in}) [J_n, J_m]\right) \\
& \qquad \qquad = \sum_{l} \alpha_{kl} J_l \qquad \text{for $i \neq j \neq k \neq i$.}\\
\end{split}
\end{gather}
This final equation yields the following relations among $\alpha_{ij}\;$:
\begin{align}
\alpha_{in} \alpha_{jn} + \alpha_{jn} \alpha_{in} & = 0  && \text{for $i \neq j$} \label{invrel1'} \\
\alpha_{in} \alpha_{jm} - \alpha_{jm} \alpha_{in}& = 0 && \text{for $i \neq j$ and $n \neq m$} \label{invrel2'} \\
\frac12 \sum_{\substack{n,\;m \\n \neq m}}(\alpha_{in} \alpha_{jm} + \alpha_{jm} \alpha_{in}) u_{nml} & = \alpha_{kl} && \text{for $i \neq j \neq k \neq i$} \label{invrel3'}
\end{align}
However, by virtue of \eqref{invrel2'} and the fact that $u_{ijk} = 0$ if
any two indices are the same, the relation \eqref{invrel3'} can be written as:
\beq
\begin{split}
\alpha_{kl} & = \frac12 \sum_{n,\;m}(\alpha_{in} \alpha_{jm} + \alpha_{jm} \alpha_{in}) u_{nml} \\
& = \frac12 \sum_{n,\;m}2 \alpha_{in} \alpha_{jm} u_{nml} \\
& = \sum_{n,\;m}\alpha_{in} \alpha_{jm} u_{nml} \quad \text{for $i \neq j \neq k \neq i$}
\end{split}
\eeq
Therefore the resulting relations between the $\alpha_{ij}$ that define the invariance
group of this algebra becomes:
\begin{align}
\alpha_{in} \alpha_{jn} + \alpha_{jn} \alpha_{in} & = 0  && \text{for $i \neq j$} \label{invrel1} \\
\alpha_{in} \alpha_{jm} - \alpha_{jm} \alpha_{in}& = 0 && \text{for $i \neq j$ and $n \neq m$} \label{invrel2} \\
\sum_{n,\;m} \alpha_{in} \alpha_{jm} u_{nml} & = \alpha_{kl} && \text{for $i \neq j \neq k \neq i$} \label{invrel3}
\end{align}

Before we define the quantum group $SO_q(3)$ and show that the
relations above correspond to the case $q=-1$, we first define the quantum
general linear group $GL_{q}(2)$. This group is defined by the elements:
\beq
M =
  \begin{pmatrix}
    a & b \\
    c & d
  \end{pmatrix}
\eeq
such that the matrix elements satisfy the following relations:
\bea
a b & = & q b a \label{ab} \\
b d & = & q d b \label{bd} \\
a c & = & q c a \label{ac} \\
c d & = & q d c \label{cd} \\
a d - q b c & = & d a - q^{-1} c b \label{det} \\
b c & = & c b \label{bc}
\eea
Using this definition of $GL_q(2)$, we first define $GL_q(3)$ as the set of
matrices:
\beq
A =
  \begin{pmatrix}
    A_{11} & A_{12} & A_{13} \\
    A_{21} & A_{22} & A_{23} \\
    A_{31} & A_{32} & A_{33}
  \end{pmatrix} \in GL_q(3)
\eeq
where
\beq
\label{gl2}
\begin{pmatrix}
  A_{in} & A_{im} \\
  A_{jn} & A_{jm}
\end{pmatrix}
\in GL_{q}(2) \quad \text{for $i \neq j$ and $m \neq n$}\quad .
\eeq
From $GL_q(3)$, one can obtain the quantum special linear group in
3-dimensions, $SL_q(3)$ by imposing the condition:
\beq
det_{q}(A) = 1
\eeq
where the quantum determinant is defined as presented in the Introduction.
Furthermore, on $SL_q(3)$, one can impose the "reality" condition:
\beq
A_{ij} = A^*_{ij}
\eeq
thus ending up the quantum group $SL_q(3, \IR)$. On the other hand, one
can impose the unitarity condition:
\beq
A^\dagger = A^{-1}
\eeq
on $SL_q(3)$ and
obtain the quantum group $SU_q(3)$. The quantum group $SO_q(3)$
is equivalent to the quantum group
\mbox{$SL_q(3, \IR) \cap SU_q(3)$}. However one can show for
$SL_q(3, \IR)$ that $q= e^{i\beta}$ for some $\beta \in \IR$ and
similarly for $SU_q(3)$ that $q \in \IR$. Thus one finds that $q =
\pm 1$ for $SO_q(3)$. When $q = 1$ the quantum group becomes the
usual $SO(3)$ group; the interesting case is when $q = -1$ which,
as we will show, is the invariance quantum group of ACSA.

By virtue of the relation \eqref{gl2} and the relations \eqref{ac} - \eqref{bc}
between the matrix elements of a $GL_q(2)$ matrix, we can see that for
the case when $q = -1$, we have the following relations between the matrix
elements of $SO_q(3)$:
\bea
A_{in} A_{jn} & = & - A_{jn} A_{in} \label{gl2-res1} \\
A_{in} A_{jm} + A_{im} A_{jn} & = & A_{jm} A_{in} + A_{jn} A_{im} \label{gl2-res2} \\
A_{im} A_{jn} & = & A_{jn} A_{im} \label{gl2-res3}
\eea
for $i \neq j$ and $n \neq m$. Furthermore, using \eqref{gl2-res3} in \eqref{gl2-res2} one finds:
\beq \label{gl2-res4}
A_{in} A_{jm} = A_{jm} A_{in}
\eeq
again for $i \neq j$ and $n \neq m$.

Thus for a matrix $A \in SO_{q=-1}(3)$, the transformation invariance relation
\eqref{invrel1} is shown to be satisfied by virtue of relation \eqref{gl2-res1}.
Similarly, the elements of such a matrix satisfy the relation \eqref{invrel2}
by virtue of the $GL_q(2)$ relations \eqref{gl2-res3} and \eqref{gl2-res4}.

It is a little harder to show that equation \eqref{invrel3} is
satisfied by elements of $SO_{q=-1}(3)$ matrices. However, if one considers
a particular choice of $k$ and $l$ on the right hand side of this equation, one
can see that on the left hand side one has the freedom to choose $i$ and $j$ in two
different ways. This implies that a particular $\alpha_{kl}$ is
equal to two separate forms. Given explicitly, for a given choice of $k$ and
$l$, we get:
\bea
\alpha_{kl} &=& \sum_{r,\; q} \alpha_{ir} \alpha_{jq} u_{rql} \\
\alpha_{kl} &=& \sum_{r,\; q} \alpha_{jr} \alpha_{iq} u_{rql}
\eea
for a particular choice of $i$ and $j$ such that $i, j, k$ are all different.
In each of these sums only two terms survive, one where $r = n$, $q = m$
and the other one where $r = m$, $q = n$, such that $n, m, l$ all different. This
is due to the nature of $u_{ijk}$ which is non-zero only if all the indices are
different. Finally we arrive at the explicit form of relation \eqref{invrel3}:
\beq
\alpha_{kl} = \alpha_{in} \alpha_{jm} + \alpha_{im} \alpha_{jn} = \alpha_{jm} \alpha_{in} +  \alpha_{jn} \alpha_{im}
\eeq
for $i, j, k$ all different and $n, m, l$ all different. Written in this form, it is obvious
that due to the $GL_q(2)$ relation \eqref{gl2-res2}, part of the above equality is satisfied by the
matrix elements of a matrix in $SO_{q=-1}(3)$. The fact that both sides of this relation
is equal to another matrix element does not rise form the $GL_q(2)$ relations but is
due to the fact that the matrix is special and orthogonal, i.e. it is due to the fact that
$A^T = A^{-1}$ and that $\det_{q=-1}A = 1$. In order to show this, we should first note that
$\det_q$ where $q = -1$ is the same as the normal determinant except there is no alternation
of signs as there is in the normal determinant; this type of
determinant with no alternation of signs is also called a
permanent. Given this fact, one can notice that the $GL_{q=-1}(2)$
relation \eqref{gl2-res2} is equal to the determinant of the $GL_{q=-1}(2)$ submatrix and
is nothing but the statement that this determinant is defined and unique.
The inverse of a matrix, $A^{-1}$, is defined as:
\beq \label{matinverse}
A^{-1} = \frac{1}{\det A} adj(A)^T
\eeq
where $adj(A)$ stands for the adjoint matrix where each matrix element is equal to the
cofactor of the same position element in the original matrix A.
For a matrix $A \in SO_{q=-1}(3)$; however, we have the fact that $A^{-1} = A^T$ and
$\det_{q = -1} A = 1$, thus for such a matrix, the relation \eqref{matinverse} becomes:
\beq
A^T = adj(A)^T \qquad \Rightarrow \qquad A = adj(A)
\eeq
which implies that the matrix elements of $A$ and $adj(A)$ are equal. This further implies
that each matrix element of $A$ is equal to the cofactor of itself. For $SO_{q=-1}(3)$
matrices, the cofactor of a matrix element is equal to the $q = -1$ determinant of its
$GL_{q=-1}(2)$ minor submatrix.
Thus, as a result of this argument, we have:
\beq
A_{kl} = cof_{kl}(A) = A_{in} A_{jm} + A_{im} A_{jn} = A_{jm} A_{in} +  A_{jn} A_{im} \quad ,
\eeq
thereby, showing that matrices which are elements of $SO_{q=-1}(3)$ fully satisfy the
transformation relations that leave ACSA invariant.

Thus, we have found that the invariance quantum group of ACSA is
the quantum group $SO_q(3)$ with $q=-1$. Strictly speaking, ACSA
is a module of the $q$-deformed $SO(3)$ quantum algebra with $q =
-1$. It is very interesting to note that the invariance group of
the angular momentum algebra is also $SO_q(3)$ but with $q=1$.

\section{Representations}

The Anticommutator Spin Algebra is defined by the relations
(\ref{eqn:defrel1}-\ref{eqn:defrel3}). In order to find the
representations of this algebra we define the operators:
\bea
J_+ & = & J_1 + J_2 \\
J_- & = & J_1 - J_2 \\
J^2 & = & J_1^2 + J_2^2 + J_3^2
\eea
which obey the following relations:
\bea
\anti{J_+}{J_3} & = & J_3 \\
\anti{J_-}{J_3} & = & -J_3 \\
J_+^2 & = & J^2 - J_3^2 + J_3 \label{jp^2}\\
J_-^2 & = & J^2 - J_3^2 - J_3 \label{jm^2}
\eea
Furthermore, it can easily be shown that $J^2$ is central in the algebra, i.e. that it commutes with all the elements of the algebra, by first observing that:
\bea
J^2_j \; J_i &=& J_j (J_k - J_i \; J_j) \nonumber \\
          &=& J_j J_k - (J_k - J_i \; J_j) J_j \nonumber \\
          &=& (J_j \; J_k - J_k \; J_j) + J_i \; J^2_j \nonumber \\
          &=& (2J_j \; J_k - J_i) + J_i \; J^2_j \quad \mbox{for}\quad  i \neq j \neq k \neq i.
\eea
Using this relation and the fact that $J^2 = \sum_j J_j$, we can see:
\bea
J^2 \; J_i &=& \sum_j J^2_j \; J_i \nonumber \\
        &=& J^3_i + \sum_{j \neq j} J^2_j \; J_i \nonumber \\
        &=& J^3_i + \sum_{j \neq i} (2J_j \; J_k - J_i + J_i \; J^2_j)
\eea
However, in the final form of this expression the sum only contains
two terms where the two indices $j$ and $k$ are symmetric. Thus the
whole expression can be written as:
\bea
J^2 \; J_i &=& J^3_i + \sum_{j \neq i} (2J_j \; J_k - J_i + J_i \; J^2_j) \nonumber \\
        &=& J^3_i - 2J_i + 2(J_j \; J_k + J_k \; J_j) + J_i \; J^2_j + J_i \; J^2_k \nonumber \\
        &=& J_i \; J^2 - 2J_i + 2J_i \nonumber \\
        &=& J_i \; J^2 \quad \mbox{for}\quad  i \neq j \neq k \neq i,
\eea
and therefore showing that $J^2$ commutes with all the elements of the algebra.

For this reason, we can label the states in our representation with the eigenvalues of $J^2$ and $J_3$:
\bea
J^2 \ket{\lambda, \mu} & = & \lambda \ket{\lambda, \mu} \\
J_3 \ket{\lambda, \mu} & = & \mu \ket{\lambda, \mu}
\eea
The action of $J_+$ and $J_-$ on the states such defined is easily
shown to be:
\bea
J_+ \ket{\lambda, \mu} & = & f(\lambda, \mu) \ket{\lambda,- \mu + 1} \label{jplus}\\
J_- \ket{\lambda, \mu} & = & g(\lambda, \mu) \ket{\lambda,- \mu -
1} \label{jminus}
\eea
It is enough to look at the norm of the states $J_+ \ket{\lambda, \mu}$ and $J_- \ket{\lambda, \mu}$ to find $f(\lambda, \mu)$ and $g(\lambda, \mu)$. Thus:
\bea
\bra{\lambda, \mu} J_+^2 \ket{\lambda, \mu} & = & |f(\lambda, \mu)|^2 \\
\bra{\lambda, \mu} J^2 - J_3^2 + J_3 \ket{\lambda, \mu} & = & |f(\lambda, \mu)|^2 \\
\lambda - \mu^2 + \mu & = & |f(\lambda, \mu)|^2 \\
f(\lambda, \mu) & = & \sqrt{\lambda - \mu^2 + \mu}
\eea
and, similarly, $g(\lambda, \mu) = \sqrt{\lambda - \mu^2 - \mu}$. These coefficients must be real due to the fact that $J_+$ and $J_-$ are hermitian operators. This constraint imposes the following
conditions on $\lambda$ and $\mu$:
\bea
\lambda - \mu^2 + \mu & \geq & 0 \\
\lambda - \mu^2 - \mu & \geq & 0
\eea
which can be satisfied by letting $\lambda = j(j+1)$ for some $j$ with:
\beq
j \geq \mu \geq -j. \label{muspec}
\eeq

Note that equation (\ref{jplus}) shows that the action of $J_+$ is
composed of a reflection which changes sign of $\mu$, the
eigenvalue of $J_3$, followed by raising by one unit. Similarly,
equation (\ref{jminus}) shows that $J_-$ reflects and lowers. Thus
the highest state $\mu = j$ is annihilated by $J_-$ and "lowered"
by $J_+$. Applying $J_+$ or $J_-$ twice to any state gives back
the same state due to relations (\ref{jp^2}) and (\ref{jm^2}).
Thus starting from the highest state we apply $J_-$ and $J_+$
alternately to get the spectrum: \beq j, -j+1, j-2, -j+3, ... \eeq
This sequence ends so as to satisfy equation (\ref{muspec}) only
for integer or half-integer $j$. For integer $j$, it terminates,
after an even number of steps, at $-j$ and visits every integer in
between only once. For half-integer $j=2k\pm\frac{1}{2}$ it ends
at $j=\pm\frac{1}{2}$ having visited only half the states with
$\mu$ half-odd integer between $j$ and $-j$. The rest of the
states cannot be reached from these but are obtained by starting
from the $\mu= -j$ state and applying $J_-$ and $J_+$ alternately;
starting with $J_-$.
\\
We now give a few examples:

\begin{figure}[!h]
  \[
  \xymatrix@R=20pt@C=100pt{
    \ar@(ul,dl)[]_*+{J_+} \ar@{-}[r]^-*+{0} & \ar@(ur,dr)[]^*+{J_-}
  }
  \]
  \caption{State diagram for $j=0$}
  \label{J0Diagram}
\end{figure}
\begin{figure}[!h]
  \[
  \xymatrix@R=20pt@C=100pt{
    \ar@/_/[d]_*+{J_+} \ar@{-}[r]^-*+{+1} & \\
    \ar@/_/[d]_*+{J_-} \ar@{-}[r]^-*+{0} & \\
    \ar@{-}[r]^-*+{-1} &  \\
  }
  \]
  \caption{State diagram for $j=1$}
  \label{J1Diagram}
\end{figure}
\begin{figure}[!h]
  \[
  \xymatrix@R=20pt@C=100pt{
    \ar@(dl,ul)[ddd]_<<<<<*+{J_+} \ar@{-}[r]^-*+{+2} & \\
    \ar@(dl,ul)[ddd]_>>>>>*+{J_-} \ar@{-}[r]^-*+{+1} & \\
    \ar@{-}[r]^-*+{0} & \ar@/_/[u]_-*+{J_+} \\
    \ar@{-}[r]^-*+{-1} & \ar@/_/[u]_-*+{J_-} \\
    \ar@{-}[r]^-*+{-2} &
  }
  \]
  \caption{State diagram for $j=2$}
  \label{J2Diagram}
\end{figure}
\begin{figure}[!h]
  \[
  \xymatrix@R=20pt@C=50pt{
    \ar@(ul,dl)[]_*+{J_+} \ar@{-}[rr]^-*+{\frac12} & & \\
                                                  & \ar@{-}[rr]^-*+{-\frac12} & & \ar@(ur,dr)[]^*+{J_-} \\
  }
  \]
  \caption{State diagram for $j=\frac12$}
  \label{J1/2Diagram}
\end{figure}
\begin{figure}[!h]
  \[
  \xymatrix@R=20pt@C=50pt{
    \ar@/_/[dd]_*+{J_+} \ar@{-}[rr]^-*+{\frac32} &  & \\
                                                 & \ar@{-}[rr]^-*+{\frac12} &   & \\
    \ar@{-}[rr]^-*+{-\frac12} & *\txt<2pc>{}      &  & \\
                                                 & \ar@{-}[rr]^-*+{-\frac32} &   & \ar@/_/[uu]^*+{J_-}\\
  }
  \]
  \caption{State diagram for $j=\frac32$}
  \label{J3/2Diagram}
\end{figure}
\pagebreak
\begin{comment}
\begin{itemize}
  \item
For $\mathbf{j=2}$ the states follow the sequence:
\[
\mathbf{\mu = 2, -1, 0, 1, -2}\quad.
\]
  \item
For $\mathbf{j=\frac{3}{2}}$ there exist two irreducible
representations one with:
\[
\mathbf{\mu = \frac{3}{2}, -\frac{1}{2}}\quad,
\] and the other with:
\[
\mathbf{\mu = -\frac{3}{2}, \frac{1}{2}}\quad.
\]

  \item
For $\mathbf{j=\frac{5}{2}}$ the two representations are given by:
\[
\mathbf{\mu = \frac{5}{2}, -\frac{3}{2}, \frac{1}{2}}\quad,
\] and by:
\[
\mathbf{\mu = -\frac{5}{2}, \frac{3}{2}, -\frac{1}{2}}\quad.
\]

\end{itemize}
\end{comment}

\section{Hopf Algebra Structure with braiding}

One natural question to ask having considered this associative algebra is whether or not it has a Hopf algebra structure. On the surface, this algebra shares a lot with its sister algebra, the $SU(2)$ Lie algebra, which has a Hopf algebra structure and one would expect ACSA to similarly have one. It turns out, however, that naively trying the same coproduct rule for ACSA does not work due to the symmetric nature of the product defined on ACSA since the product is defined in terms of anticommutators. As was noted in the Introduction of this work, the coproduct of the Lie algebra requires the product on the Lie algebra to be anti-symmetric. For this reason, the coproduct of the $SU(2)$ Lie algebra is not suitable for ACSA.

In our quest for a Hopf algebra structure for ACSA, it would be more fruitful to understand the nature of the relationship of ACSA with the $SU(2)$ Lie algebra. If one names the generators of the $SU(2)$ algebra $I_i$, then it can easily be shown that $\tilde{J_i}$ defined as
\beq
\tilde{J}_i = - I_i \otimes \sigma_i
\eeq
satisfy the defining relation of ACSA since:
\bean
\tilde{J}_i \tilde{J}_j + \tilde{J}_j \tilde{J}_i &=& I_i I_j \otimes \sigma_i \sigma_j + I_j I_i \otimes \sigma_j \sigma_i\\
&=& I_i I_j \otimes i \sigma_k + I_j I_i \otimes -i \sigma_k \\
&=& i(I_i I_j - I_j I_i) \otimes \sigma_k \\
&=& i(iI_k) \otimes \sigma_k \\
&=& - I_k \otimes \sigma_k \\
&=& \tilde{J}_k
\quad \mbox{for}\quad  i \neq j \neq k \neq i.
\eean

Similarly, the generators satisfying the $SU(2)$ Lie algebra can be written in terms of ACSA generators as:
\beq
\tilde{I}_i = J_i \otimes \sigma_i
\eeq
since:
\bean
\tilde{I}_i \tilde{I}_j - \tilde{I}_j \tilde{I}_i &=& J_i J_j \otimes \sigma_i \sigma_j - J_j J_i \otimes \sigma_j \sigma_i\\
&=& J_i J_j \otimes i \sigma_k - J_j J_i \otimes -i \sigma_k \\
&=& i(J_i J_j + J_j J_i) \otimes \sigma_k \\
&=& i(J_k) \otimes \sigma_k \\
&=& \tilde{I}_k
\quad \mbox{for}\quad  i \neq j \neq k \neq i.
\eean

These two relations show that the $SU(2)$ algebra and ACSA are so closely related that it is not even possible to identify which one of these algebras is more fundamental. Both of them can be written in terms of the generators of the other and their algebraic structure can be derived from the structure of the other one. However, as mentioned, the Hopf algebra structure of ACSA cannot be derived from the Hopf algebra structure of the $SU(2)$ algebra. Specifically, ACSA does not admit a coproduct defined in a normal way using the usual tensor products. Such a coproduct can be defined if one were to extend the permutation operator $\tau$ used in the connecting relation. Normally the operation of $\tau$ is defined as:
\beq
\tau(A \otimes B) = B \otimes A \quad ,
\eeq
however, if one considers the algebra to be graded and one were to define a degree operator ($deg$) which is $0$ for bosonic variables and is $1$ for fermionic variables, then the natural redefinition of the $\tau$ operator is
\beq
\tau(A \otimes B) = (-1)^{deg\;A\; deg\;B\;} B \otimes A \quad .
\eeq
Using this redefined permutation operator, one can still write down the Hopf algebra relations and only the connecting relation will be the one that will be redefined; thus, we arrive at a braided Hopf algebra structure.

When the permutation operator is redefined in this way, the product of two tensor product terms is given by $(A\otimes B) (C\otimes D) = (-1)^{deg\;B\; deg\;C\;} (AC \otimes BD)$ where the $-1$ factor comes in because of the reordering of the $B$ and $C$ terms. Using this rule and defining the degree of $1$ as $0$ and the degrees of $J_1, J_2, J_3$ as $1$, we can see that the coproduct defined as:
\bea
\Delta(J_i) &=& 1 \otimes J_i + J_i \otimes 1 \\
\Delta(1) &=& 1 \otimes 1
\eea
satisfies the algebra structure relations since:
\bean
\Delta(J_i)\Delta(J_j)
&=& (1 \otimes J_i + J_i \otimes 1)(1 \otimes J_j + J_j \otimes 1) \\
&=& 1 \otimes J_i J_j - 1 J_j \otimes J_i 1 + J_i 1 \otimes 1 J_j + J_i J_j \otimes 1 \\
&=& 1 \otimes J_i J_j - J_j \otimes J_i + J_i \otimes J_j + J_i J_j \otimes 1
\eean
and
\bean
\Delta(J_i)\Delta(J_j) + \Delta(J_j)\Delta(J_i)
&=& 1 \otimes J_i J_j - J_j \otimes J_i + J_i \otimes J_j + J_i J_j \otimes 1 \\
& & + 1 \otimes J_j J_i - J_i \otimes J_j + J_j \otimes J_i + J_j J_i \otimes 1 \\
&=& 1 \otimes J_i J_j + J_i J_j \otimes 1 \\
&=& 1 \otimes J_k + J_k \otimes 1 \\
&=& \Delta(J_k)
\quad \mbox{for}\quad  i \neq j \neq k \neq i.
\eean
The counit and coinverse are simpler and they match with the definitions for the normal Lie algebra, i.e.:
\bea
\epsilon(J_i) &=& 0 \\
S(J_i) &=& -J_i
\eea
These definitions of the coproduct, the counit and the coinverse give us a braided Hopf algebra structure for ACSA. 

\chapter{QUANTUM GROUPS ASSOCIATED WITH INVARIANCE OF NON-DEFORMED OSCILLATORS}


The concepts of bosons and fermions lie at the heart of
microscopic physics. They are described in terms of creation and
annihilation operators of the corresponding particle algebra: \bea
c_i c_j \mp c_j c_i & = & 0 \\
c_i c^*_j \mp c^*_j c_i & = & \delta_{ij} \eea where the upper
sign is for the boson algebra $BA(d)$ and the lower sign is for
the fermion algebra $FA(d)$.

It has been realized that quantum algebras play an important role
in the description of physical phenomena. Some classical physical
systems which are invariant under a classical Lie group, when
quantized, are invariant under a quantum group \cite{fadeev}. The
quantum groups thus considered turn out to be $q$-deformations of
the classical semisimple groups. On the other hand, inhomogeneous
quantum groups \cite{sww,pw} are perhaps more interesting since
classical inhomogeneous groups such as the Poincar\'e group are
more important in physics.

In this paper we will investigate an important class of
inhomogeneous quantum groups which are related to the boson
algebra $BA(d)$ and the fermion algebra $FA(d)$. Although $BA(d)$
and $FA(d)$ themselves are not quantum groups, by considering
quantum group versions of symmetry transformations acting on these
algebras, one can arrive at these inhomogeneous quantum groups.
Mathematically speaking we are thus interested in constructing
left modules of these algebras such that these modules have Hopf
algebra structure.

Traditionally the boson algebra has the symmetry group $ISp(2d,
\IR)$, the inhomogeneous symplectic group, which transforms
creation and annihilation operators as: \beq c_i \rightarrow
\alpha_{ij} c_j + \beta_{ij} c^*_j + \gamma_i \quad . \eeq In this
transformation $\alpha_{ij}, \beta_{ij}, \gamma_i$ are complex
numbers satisfying the constraints required by the group $ISp(2d,
\IR)$. One should note that this symmetry group is also the group
of linear canonical transformations of a classical dynamical
system. An important physical application of this transformation
is the Bogoliubov transformation which is crucial in the
explanation of many quantum mechanical effects such as the Unruh
Effect \cite{unruh}
 and Hawking Radiation \cite{hawking}.
In the case of the Hawking Radiation, the physical
reinterpretation of the transformed operators imply that the
future vacuum state is annihilated by the transformed annihilation
operator, which is related to the initial creation and
annihilation operators by a Bogoliubov transformation.

Similar to the boson algebra, the fermion algebra has the
classical symmetry group $O(2d)$ with the transformation law: \beq
c_i \rightarrow \alpha_{ij} c_j + \beta_{ij} c^*_j \quad . \eeq
however, unlike its bosonic counterpart this algebra is not
inhomogeneous. This fact is the primary motivation for the
generalization that we are going to offer. By relaxing the
conditions on the transformation coefficients such as
commutativity, one can come up with inhomogeneous invariance
(quantum)groups for fermions and for bosons alike. The explicit
$R$-matrices utilizing the quantum group properties of these
structures have already been presented \cite{agy, ab}. In this
paper, after a brief definition of these quantum groups \FIO, the
fermionic inhomogeneous orthogonal quantum group, and \BISp, the
bosonic inhomogeneous symplectic quantum group, in Section 1, we
will investigate their sub(quantum)groups and also study the
(quantum)groups obtained by their contractions. In the last
section, $FIO(2d + 1, \IR)$, the fermionic inhomogeneous quantum
orthogonal group in odd number of dimensions, will also be defined
and its properties examined.

\section{The Bosonic Inhomogeneous Symplectic Quantum Group \BISp}

\section{The Fermionic Inhomogenous Group \FIO}
A general transformation of a particle algebra can be described in
the following way: \beq \left(
\begin{array}{c}
c' \\
{c^{*}}' \\
1
\end{array}
\right) = \left(
\begin{array}{ccc}
\alpha & \beta & \gamma \\
\beta^* & \alpha^* & \gamma^* \\
0 & 0 & 1
\end{array}
\right) \dot{\otimes} \left(
\begin{array}{c}
c \\
c^* \\
1
\end{array}
\right) \eeq where $c, c^*, \gamma, \gamma^*$ are column matrices
and $\alpha, \beta, \alpha^*, \beta^*$ are $d\times d$ matrices.
Thus, in index notation the transformation is given by: \bea
c'_i &=& \alpha_{ij} \otimes c_j + \beta_{ij} \otimes c^*_j + \gamma_i \otimes 1 \quad , \\
{c^{*}}'_i &=& \alpha^*_{ij} \otimes c^*_j + \beta^*_{ij} \otimes
c_j + \gamma^*_i \otimes 1 \quad . \eea

Given this transformation, we look for an algebra $\mathcal{A}$
generated by these matrix elements such that the particle algebra
remains invariant. Thus, we first write the transformation matrix
in the above equation in the following way: \beq M = \left(
\begin{array}{cc|c}
\alpha & \beta & \gamma \\
\beta^* & \alpha^* & \gamma^* \\
\hline 0 & 0 & 1
\end{array}
\right)
 =
\left(
\begin{array}{c|c}
A & \Gamma \\
\hline 0 & 1
\end{array}
\right) \quad . \label{M-matrix} \eeq

We assume that $\alpha_{ij}, \beta_{ij}, \gamma_i$ belong to a
possibly noncommutative algebra on which a hermitian conjugation
denoted by $*$ is defined. We also assume that the matrix elements
of $A$ form a commutative subalgebra.

Applying this transformation and requiring that the
bosonic/fermionic particle algebra remains invariant after the
transformation, we arrive at the following relations that the
transformation parameters should obey: \bea
\gamma_i \gamma^*_j \mp \gamma^*_j \gamma_i &=& \delta_{ij} - \alpha_{ik}\alpha^*_{jk} \pm \beta_{ik} \beta^*_{jk} \label{rel1} \\
\gamma_i \gamma_j \mp \gamma_j \gamma_i &=& \pm \beta_{ik} \alpha_{jk} - \alpha_{ik} \beta_{jk} \label{rel2} \\
\alpha_{ij} \gamma_k \mp \gamma_k \alpha_{ij} & = & 0 \label{rel3} \\
\beta_{ij} \gamma_k \mp \gamma_k \beta_{ij} & = & 0 \label{rel4} \\
\alpha_{ij} \gamma^*_k \mp \gamma^*_k \alpha_{ij} & = & 0 \label{rel5} \\
\beta_{ij} \gamma^*_k \mp \gamma^*_k \beta_{ij} & = & 0
\label{rel6} \eea together with the $*$-conjugates of these
relations. In these relations the upper and lower signs are for
the transformation of bosons and fermions, respectively.
Furthermore according to our assumption above, the set
$\alpha_{ij}, \beta_{i,j}, \alpha^*_{ij}, \beta^*_{ij}$ forms a
commutative algebra.

The set of matrices $M$ obeying the above relations form the group
of inhomogeneous transformations of bosons and fermions. For
bosons, we name this group \BISp and for fermions \FIO. These
symmetry groups, however, are not classical groups but are in fact
quantum groups with a Hopf algebra structure. As shown in
\cite{ab}, this Hopf algebra has an explicit $R$-matrix
representation and the coproduct, counit and antipode are defined
as: \bea
\Delta(M) & = & M \dot{\otimes} M \label{coproduct} \\
\epsilon(M) & = & I \label{counit} \\
S(M) & = & M^{-1} \label{antipode} \quad . \eea In Equation
\ref{coproduct}, the symbol $\dot{\otimes}$ denotes the usual
matrix multiplication where when elements of the matrices are
multiplied, tensor multiplication is used.

The inverse of the matrix $M$ can be defined as: \beq M^{-1} =
\left(
\begin{array}{cc}
A^{-1} & -A^{-1} \Gamma \\
0 & 1
\end{array}
\right) \eeq where $A^{-1}$ is defined in the standard way since
matrix elements of $A$ were assumed to be commutative.

\section{Subgroups and Contractions}
After having shown that the inhomogeneous transformations form the
symmetry groups \FIO and \BISp and that they are quantum groups,
one important question is what sub(quantum)groups do these quantum
groups have. For example, we know that the group $ISp(2d, \IR)$ is
an important special subgroup of \BISp and other
sub(quantum)groups could turn out to have similarly important
physical applications. While searching for sub(quantum)groups, we
would also like to find new (quantum)groups allowed by suitable
contractions \cite{inonu} of these quantum groups as well.

The sub(quantum)groups of these algebras are obtained by imposing
additional relations on the matrix elements of $M$ which obey the
relations (\ref{rel1}) - (\ref{rel6}). The additional relations
that we will impose are:
\begin{enumerate}
\renewcommand{\labelenumi}{\bf(\alph{enumi})}
\item $\delta_{ij} - \alpha_{ik}\alpha^*_{jk} \pm \beta_{ik} \beta^*_{jk} = \pm \beta_{ik} \alpha_{jk} - \alpha_{ik} \beta_{jk} = 0$
\item $\gamma_i = 0$
\item $\beta_{ij} = 0$
\item $\alpha_{ij} = 0$
\end{enumerate}

We would like to study the implication of each relation one by one
in the bosonic and the fermionic case.

\subsection{Inhomogeneous Sub(super)groups}
The relation {\bf(a)}:
\[
\delta_{ij} - \alpha_{ik}\alpha^*_{jk} \pm \beta_{ik} \beta^*_{jk}
= \pm \beta_{ik} \alpha_{jk} - \alpha_{ik} \beta_{jk} = 0
\]
by virtue of (\ref{rel1}) and (\ref{rel2}) implies that $\gamma_i
\gamma^*_j \mp \gamma^*_j \gamma_i = 0$ and $\gamma_i \gamma_j \mp
\gamma_j \gamma_i = 0$, i.e. that the inhomogeneous transformation
parameters are commutative or anticommutative variables.

In the case of the fermionic particles, we end up with an
inhomogeneous orthogonal algebra where the inhomogeneous
parameters are grassmanian variables thus giving us the group
$GrIO(2d, \IR)$ as the resulting subgroup of \FIO. This group can
be considered to be an inhomogeneous supergroup.

More generally, in the fermionic case, $\alpha_{ij}$,
$\beta_{ij}$, $\alpha^*_{ij}$, $\beta^*_{ij}$ anticommute with
$\gamma_i$, $\gamma^*_i$ and the \FIO matrices $M$ are multiplied
with each other using the standard tensor product. It can
additionally be shown that $\alpha_{ij}$, $\beta_{ij}$,
$\alpha^*_{ij}$, $\beta^*_{ij}$ can be taken to commute with
$\gamma_i$, $\gamma^*_i$ provided that the matrices $M$ are
multiplied with a braided \cite{majid} tensor product, eg. \beq (A
\otimes B)(C \otimes D) = - AC \otimes BD \eeq whenever $B$ and
$C$ are both fermionic. This also corresponds to the usual
superalgebra approach.

For bosonic particles, applying the relation {\bf(a)} gives us the
classical symmetry group of the bosonic particle algebra $ISp(2d,
\IR)$ in which all the parameters of the inhomogeneous
transformation are commutative.

\subsection{Homogeneous Subgroups}
The relation {\bf(b)}:
\[
\gamma_i = 0
\]
practically gets rid of the inhomogeneous part of the
transformation and also implies the relation {\bf(a)} considered
in the previous subsection.

For the fermionic particles, this gives us the subgroup in the
previous subsection without the inhomogeneous part which is the
classical orthogonal group $O(2d, \IR)$ and similarly, for bosonic
particles, gives us the classical symplectic group $Sp(2d, \IR)$.

\subsection{Fermionic and Bosonic Inhomogeneous Unitary Quantum Groups}
The relation {\bf(c)}:
\[
\beta_{ij} = 0
\]
applied to the transformation gets rid of the off-diagonal members
of the homogeneous part of it and leaves us with the following
relation: \beq \gamma_i\gamma^*_j \mp \gamma^*_j\gamma_i =
\delta_{ij} - \alpha_{ik}\alpha^*_{jk} \eeq For the homogeneous
part of the transformation, this equation implies $ \delta_{ij} =
\alpha_{ik}\alpha^*_{jk} $ which tells us that the submatrices
$\alpha$ and $\alpha^*$ in equation (\ref{M-matrix}) are both
members of $U(d)$. The subgroup we have arrived at thus is an
inhomogeneous quantum group whose homogeneous part is $U(d)$. For
fermions we will name this group $FIU(d)$, the fermionic
inhomogeneous quantum group, and for bosons we will similarly name
it $BIU(d)$, the bosonic inhomogeneous quantum group.

\subsection{Fermion and Boson Algebra}
The relation {\bf(d)}:
\[
\alpha_{ij} = 0
\]
applied alone onto the transformation gets rid of the diagonal
members of the homogeneous part and prevents such transformations
from forming a (quantum)group since the homogeneous parts of these
set of transformations can never include the identity
transformation.

However, if this relation is applied together with the previous
one, relation {\bf(c)}, the resulting relation gets rid of the
whole homogeneous part of the transformation leaving only the
inhomogeneous part and gives us a single relation: \beq
\gamma_i\gamma^*_j \mp \gamma^*_j\gamma_i = \delta_{ij} \eeq which
gives us back the fermion algebra, $FA(d)$, for the fermionic case
and the boson algebra, $BA(d)$, for the bosonic case.

We should note, however, that after this condition is applied, the
resulting set of matrices $M$, which form $FA(d)$ or $BA(d)$, are
no longer quantum or classical groups since the antipode defined
in equation (\ref{antipode}) no longer exits. For this reason,
these algebras can be considered to be a boundary for the
subgroups of their corresponding quantum groups.

\subsection{Sub(quantum)group Diagram}
As a result of the above discussion, we get the sub(quantum)group
diagram:
\[
\begin{CD}
{FIO(2d, \IR)} @>{\bf (a)}>> {GrIO(2d, \IR)} @>{\bf (b)}>> {O(2d, \IR)} \\
@V{\bf (c)}VV @V{\bf (c)}VV @V{\bf (c)}VV \\
{FIU(d)}  @>{\bf (a)}>> {GrIU(d)} @>{\bf (b)}>> {U(d)} \\
@V{\bf (d)}VV \\
{FA(d)}
\end{CD}
\]
for the fermionic case and the diagram:
\[
\begin{CD}
{BISp(2d, \IR)} @>{\bf (a)}>> {ISp(2d, \IR)} @>{\bf (b)}>> {Sp(2d, \IR)} \\
@V{\bf (c)}VV @V{\bf (c)}VV @V{\bf (c)}VV \\
{BIU(d)} @>{\bf (a)}>> {IU(d)} @>{\bf (b)}>> {U(d)} \\
@V{\bf (d)}VV \\
{BA(d)}
\end{CD}
\]
for the bosonic case.

\subsection{Contractions of \FIO and \BISp}
In order to explore the new (quantum)groups that will come about
as the result of a contraction, we replace $\gamma_i$ by
$\gamma_i/\hbar$ so that we may consider the case $\hbar
\rightarrow 0$. After this replacement, the equations (\ref{rel1})
and (\ref{rel2}) become: \bea
\gamma_i \gamma^*_j \mp \gamma^*_j \gamma_i &=& \hbar(\delta_{ij} - \alpha_{ik}\alpha^*_{jk} \pm \beta_{ik} \beta^*_{jk}) \\
\gamma_i \gamma_j \mp \gamma_j \gamma_i &=& \hbar(\pm \beta_{ik}
\alpha_{jk} - \alpha_{ik} \beta_{jk}) \eea When we consider the
case $\hbar \rightarrow 0$, we get the relations: \bea
\gamma_i \gamma^*_j \mp \gamma^*_j \gamma_i &=& 0 \\
\gamma_i \gamma_j \mp \gamma_j \gamma_i &=& 0 \eea which imply
that the inhomogeneous part of the transformation form grassmanian
variables for the fermionic case and ordinary complex numbers for
the bosonic case. What makes this case different from the previous
case of sub(super)groups is that the homogeneous part of this
transformation forms a matrix $A$ with non-zero determinant. We
can transform such a matrix $A$ with a similarity transformation
given by the unitary matrix: \beq U = \frac{1}{\sqrt{2}} \left(
\begin{array}{cc}
1 & 1 \\
i & -i
\end{array}
\right) \eeq to put it in a real form. The transformation gives:
\bea
A' & = & U A U^\dagger \\
& = & \frac12 \left(
\begin{array}{cc}
1 & 1 \\
i & -i
\end{array}
\right) \left(
\begin{array}{cc}
\alpha & \beta \\
\beta^* & \alpha^*
\end{array}
\right) \left(
\begin{array}{cc}
1 & -i \\
1 & i
\end{array}
\right) \\
& = & \left(
\begin{array}{cc}
Re(\alpha) + Re(\beta) & Im(\alpha) - Im(\beta) \\
- Im(\alpha) - Im(\beta) & Re(\alpha) - Re(\beta)
\end{array}
\right) \eea which is a real matrix that is a member of the
general linear group $GL(2d, \IR)$. Thus for the fermionic case we
have the group $GrIGL(2d, \IR)$, the grassmanian inhomogeneous
general linear group, and for the bosonic case we have $IGL(2d,
\IR)$, the inhomogeneous general linear group.

If we consider the contraction of the subgroups as well then we
should examine the $\hbar \rightarrow 0$ limit after the relations
{\bf(c)} and {\bf(d)} are applied.

After we apply relation {\bf(c)}, we get the subgroups, $FIU(d)$
and $BIU(d)$ as discussed previously. After, the contraction,
again the inhomogeneous part of these groups become grassmanian
variables and complex numbers for the fermionic and bosonic cases
respectively. However, if we apply the previous similarity
transformation on the homogeneous part, we get: \bea
A' & = & U A U^\dagger \\
& = & \frac12 \left(
\begin{array}{cc}
1 & 1 \\
i & -i
\end{array}
\right) \left(
\begin{array}{cc}
\alpha & 0 \\
0 & \alpha^*
\end{array}
\right) \left(
\begin{array}{cc}
1 & -i \\
1 & i
\end{array}
\right) \\
& = & \left(
\begin{array}{cc}
Re(\alpha) & Im(\alpha) \\
- Im(\alpha) & Re(\alpha)
\end{array}
\right) \\
& = & Re(\alpha) \I1
 +
Im(\alpha) \II \eea where $\I1$ stands for the identity matrix and
$\II$ stands for the matrix the square of which is minus the
identity matrix. This way we can see that the matrix $A'$ is a
actually member of $GL(d, \IC)$. This gives us $GrIGL(d, \IC)$ as
the group for the contraction in the fermionic case and $IGL(d,
\IC)$ for the bosonic case.

We have previously shown that we get the fermionic and bosonic
algebras after applying both of the relations {\bf(c)} and
{\bf(d)}. We have also discussed that in this case only the
inhomogeneous part of the transformation survives and after
applying the contraction the inhomogeneous part of the
transformation turns into grassmanian or complex variables. Thus
in this case, the contraction of $FA(d)$ gives us $Gr(d, \IC)$ and
the contraction of $BA(d)$ gives us $\IC^d$.


As a summary, for the contraction considered combined with the
remaining subgroup relations we get the tables:
\[
\begin{CD}
{FIO(2d, \IR)} @>{\bf (e)}>> {GrIGL(2d, \IR)} \\
@V{\bf (c)}VV @V{\bf (c)}VV \\
{FIU(d)} @>{\bf (e)}>> {GrIGL(d, \IC)} \\
@V{\bf (d)}VV @V{\bf (d)}VV \\
{FA(d)} @>{\bf (e)}>> {Gr(d, \IC)}
\end{CD}
\]
for the fermionic case and:
\[
\begin{CD}
{BISp(2d, \IR)} @>{\bf (e)}>> {IGL(2d, \IR)} \\
@V{\bf (c)}VV @V{\bf (c)}VV \\
{BIU(d)} @>{\bf (e)}>> {IGL(d, \IC)} \\
@V{\bf (d)}VV @V{\bf (d)}VV \\
{BA(d)} @>{\bf (e)}>> {\IC^d}
\end{CD}
\]
for the bosonic case.

\section{The Fermionic Inhomogeneous
Orthogonal Quantum Group of Odd Dimension}

When we consider a unitary transformation of the \FIO matrix as:
\[
M \rightarrow UMU^{-1}
\]
using the unitary matrix: \beq U = \left(
\begin{tabular}{cc|c}
$\frac{1}{\sqrt{2}}$ & $\frac{1}{\sqrt{2}}$ & $0$ \\
$\frac{i}{\sqrt{2}}$ & $\frac{-i}{\sqrt{2}}$ & $0$ \\
\hline $0$ & $0$ & $1$
\end{tabular}
\right) \eeq we can see that we can put $M$ in the real form: \beq
\left(
\begin{tabular}{cc|c}
$Re(\alpha + \beta)$ & $Im(\alpha - \beta)$ & $\sqrt{2}Re(\gamma)$ \\
$Im(-\alpha - \beta)$ & $Re(\alpha - \beta)$ & $\sqrt{2}Im(\gamma)$ \\
\hline $0$ & $0$ & $1$
\end{tabular}
\right) = \left(
\begin{tabular}{c|c}
$A$ & $\Gamma$ \\
\hline $0$ & $1$
\end{tabular}
\right) \eeq where $A$ and $\Gamma$ matrices are defined as in
(\ref{M-matrix}), and $Re$ and $Im$ denote the hermitian and
anti-hermitian parts.

Using this form it is found that for \FIO, the transformation
relations (\ref{rel1}) and (\ref{rel2}) together become:

\beq [\Gamma_i, \Gamma_j]_+ = \delta_{ij} - A_{ik} A_{jk}
\quad\quad, i, j = 1, 2, \ldots , 2d . \eeq By extending the range
of the indices in this relation to odd-dimensions it is possible
to define $FIO(2d + 1, \IR)$, the fermionic inhomogeneous
orthogonal algebra of odd dimension.

Similar to the analysis that went into finding the
sub(quantum)groups of \FIO, we can also investigate the
sub(quantum)groups of $FIO(2d + 1, \IR)$. The sub(quantum)group
relations in this case, however, are more restricted owing to the
fact that the algebra is not described anymore by submatrices of
the $A$ matrix but is rather described by the whole matrix itself.
Thus, we cannot set $\alpha$ or $\beta$ to zero on their own, we
can only restrict the algebra by setting the whole of $A$ to zero.
Thus the resulting sub(quantum)algebra relations are:

\begin{enumerate}
\renewcommand{\labelenumi}{\bf(\alph{enumi})}
\item $\delta_{ij} - A_{ik}A_{jk} = 0$
\item $\Gamma_i = 0$
\item $A_{ij} = 0$
\end{enumerate}
which, through a similar analysis to the even dimensional case,
gives us the following sub(quantum)group diagram:

\[
\begin{CD}
{FIO(2d + 1, \IR)} @>{\bf (a)}>> {GrIO(2d + 1, \IR)} @>{\bf (b)}>> {O(2d + 1, \IR)} \\
@V{\bf (c)}VV\\
{Clif(2d + 1)}
\end{CD}
\]



\chapter{CONCLUSIONS}

%% TODO: Write this

%% Conclusions from ACSA
\begin{comment}
\section{Conclusions}

The Anticommutator Spin Algebra, which is a special Jordan
algebra, has many implications. The first of these is the fact
that this algebra is a consistent fermionic algebra which is not a
superalgebra.
% The
%bosonic and fermionic generators $B, F$ in a superalgebra obey:
%\bea
%[B,B] & = & B \nonumber \\ \nonumber
%[F,B] & = & F \\ \nonumber
%\{F,F\} & = & B \nonumber
%\eea
%whereas in ACSA this relation is of the form:
%\[
%\{F,F\} = F
%\]
%which shows how different it is from a superalgebra.
For possible physical applications the right-hand side of the
defining relations (\ref{eqn:defrel1}-\ref{eqn:defrel3}) must also
be supplied with an $\hbar$. In a superalgebra approach where the
$J_i$ are regarded as odd operators, the $\hbar$ on the right-hand
side should also be regarded as an operator anticommuting with the
$J_i$. These models \cite{leites,batalin} result from the
quantization of the odd Poisson bracket. In our approach however,
the concept of grading and therefore an underlying Poisson bracket
formalism does not exist. In particular, there is no Jacobi
identity. Nevertheless, the associative algebra we consider is
consistent with quantum mechanics where physical observables
correspond to hermitian operators and their eigenvalues to
possible results of physical measurement of these observables. It
is for this reason that ACSA suggests a new kind of statistics
which, we believe, will be useful in physics.

The second implication is the important role of quantum groups in
mathematical physics. As we have shown in this paper, the
invariance group of ACSA turns out to be a quantum group. Given
the fact that ACSA is very similar to normal spin algebra and that
the invariance group of spin algebra plays an important role in
physics, the invariance quantum group of ACSA, $SO_{q=-1}(3)$,
becomes a prime example of how central quantum groups have become
in mathematical physics. It is also interesting to note that more
algebras like ACSA can be constructed where the commutators of the
original Lie algebra are turned into anticommutators and that such
algebras might also have invariance quantum groups that is the
same as the invariance group of the original Lie algebra with
$q=-1$. This possibility is open to investigation in a more
general framework.
\end{comment}

% From BISP and FIO paper
\begin{comment}
\section{Discussion}
As we have shown, the boson and fermion algebras can be obtained
as a limit of the inhomogeneous quantum groups \BISp and \FIO. We
can understand why these boson and fermion algebras are not
quantum groups from this construction, since in this limit the
quantum group becomes singular and the antipode does not exist.
Thus we can consider these quantum groups as deformations with a
Hopf algebra structure of their respective particle algebras. This
construction is similar to $q$-deforming the bosonic oscillator to
obtain Pusz-Woronowicz \cite{puszwor} oscillators and then
constructing the $q$-deformed quantum unitary groups as their left
modules. Similarly, in that construction, the $q$-deformed
oscillator can be reobtained as a limit of these $q$-deformed
quantum unitary groups. However, unlike that construction the
quantum groups presented in this paper are inhomogeneous quantum
groups.

Finally, we would like to remark that the widely used field
theoretical generalization achieved by extending the discrete
indices $i, j, k$ to continuous variables together with a
replacement of the Kr\"onecker deltas to Dirac delta functions is
also applicable to the quantum groups we have presented. In this
respect, these quantum groups are also different from the
Pusz-Woronowicz oscillators which cannot be extended to continuous
indices.

We believe that the establishment of these and similar quantum
groups in field theory will be helpful in generalizing methods of
quantization. These approaches will yield a more consistent
approach to interacting field theory and will be the subject of
further investigations.

\end{comment}

The importance of Lie groups in physics arises from the fact that
they are invariance groups of classical physical systems. Thus, for
example, the 3 dimensional position space, the 3 dimensional momentum
space and the 3 dimensional angular momentum space are all transformed
under the same Lie group $SO(3)$. When the classical system is
quantized one realizes that although the resulting quantum system
is invariant under the classical group $SO(3)$, one should also 
perhaps generalize the definition of a Lie group such that the
transformation matrix may have non-commuting entities. This is 
precisely what has been considered in this work. If one considers
the angular momentum algebra and tries to find such a non-commutative
quantum group which leaves the commutation relations of the Lie algebra
invariant, one finds that the elements of the transformation matrix
should be commutative and reobtain the classical group $SO(3)$. On the
other hand, as we have shown, when one considers the anticommuting
spin algebra, its invariance quantum group becomes $SO_{-1}(3)$.
It is also interesting to note that more
algebras like the anticommuting spin algebra can be constructed
where the original Lie algebra is turned into a similar Jordan
algebra. These might also have invariance quantum groups that is the
same as the invariance group of the original Lie algebra in the
limit $q=1$. This possibility is open to investigation in a more
general framework.

As far as the momentum and position are concerned, one realizes that
the Heisenberg algebra inevitably contains the unit operator
and therefore the transformations considered on that algebra should 
be inhomogeneous. It was shown
in this work that this approach indeed makes sense by explicitly
calculating the invariance quantum groups of the bosonic and fermionic
oscillator algebras. In 3 dimensions, the hermitian and antihermitian parts
of the annihilation operator can be identified with the position and 
$i$ times the momentum operator, respectively. For this reason, the
invariance quantum group \BISp that was introduced becomes the invariance
quantum group of the quantum phase space in 3 dimensions. Both
the fermionic and the bosonic inhomogeneous quantum groups considered
in this work are relevant for field theoretical systems; especially since
they can be made infinite dimensional. This is achieved by 
extending the discrete indices $i, j, k$ in \BISp and \FIO 
to continuous  variables together with a replacement of 
the Kr\"onecker deltas to Dirac delta functions. 


As was shown, the boson and fermion algebras can be obtained
as a limit of the inhomogeneous quantum groups \BISp and \FIO. We
can understand why these boson and fermion algebras are not
quantum groups from this construction, since in this limit the
quantum group becomes singular and the antipode does not exist.
Thus we can consider the invariance quantum groups 
as deformations, with a Hopf algebra structure, of their 
respective particle algebras. This
construction is similar to $q$-deforming the bosonic oscillator to
obtain Pusz-Woronowicz \cite{puszwor} oscillators and then
constructing the $q$-deformed quantum unitary groups as their left
modules. Similarly, in that construction, the $q$-deformed
oscillator can be reobtained as a limit of these $q$-deformed
quantum unitary groups. However, in contrast to that construction the
quantum groups presented in this paper are inhomogeneous quantum
groups.

Lastly, we would like to remark on the definition of a quantum group.
Although in most works, quantum groups are defined as noncocommutative
and noncommutative Hopf algebras, this definition does not produce
any physical insight. It makes more sense to define a quantum group as
a Hopf algebra which is a left and/or right module of a physical
algebra obtained by quantizing a classical system. This thesis has directly
dealt with such quantum groups.







\bibliography{references}
\bibliographystyle{unsrt}
%\bibliographystyle{plain}

\end{document}
